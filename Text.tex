\documentclass[8pt]{article}
\usepackage{graphicx} % Required for inserting images
\usepackage{amsthm}
\usepackage{amsmath}
\usepackage{amsfonts}
\usepackage{amsopn}
\usepackage{comment}
\usepackage{amssymb}
\usepackage{hyperref}
\usepackage{bussproofs}
\usepackage[all, 2cell]{xy}
\usepackage[all]{xy}
\usepackage{rotating}
\usepackage{lscape}
\usepackage{minted}


\theoremstyle{definition}
\newtheorem{definition}{Definition}[section]

\theoremstyle{definition}
\newtheorem{theorem}{Theorem}[section]

\theoremstyle{definition}
\newtheorem{claim}{Claim}[section]

\theoremstyle{definition}
\newtheorem{ex}{Example}[section] 

\theoremstyle{definition}
\newtheorem{cons}{Construction}[section] 

\theoremstyle{definition}
\newtheorem{rem}{Remark}[section] 


\theoremstyle{definition}
\newtheorem{prop}{Proposition}[section]

\theoremstyle{definition}
\newtheorem{lemma}{Lemma}[section]

\theoremstyle{definition}
\newtheorem{fact}{Fact}[section]

\theoremstyle{definition}
\newtheorem{remark}{Remark}[section]

\theoremstyle{definition}
\newtheorem{notation}{Notation}[section]

\theoremstyle{definition}
\newtheorem{example}{Example}[section]

\theoremstyle{definition}
\newtheorem{exercise}{Exercise}[section]

\theoremstyle{definition}
\newtheorem{col}{Corollary}[section]

\theoremstyle{question}
\newtheorem{question}{Question}

\let\strokeL\L
\renewcommand\L{\mathbf{L}}

\title{Some Notes on Set Theory, Pt 1}
\author{Daniel Rogozin}
\date{ }

\begin{document}

\maketitle

\tableofcontents

\newpage

\section{Cardinals}

An ordinal number $\alpha$ is a \emph{cardinal number} if no $\beta < \alpha$ such that $|\alpha| = |\beta|$.
Further, we shall use $\kappa, \lambda, \mu$ to denote cardinal numbers.

Let $W$ be a well-ordered set, then there exists an ordinal $\alpha$ such that $|W| = |\alpha|$, so we let:
\begin{center}
  $|W| = \text{the least ordinal $\alpha$ such that $|W| = |\alpha|$}$
\end{center}

An \emph{aleph} is an infinite cardinal number. 

Let $\alpha$ be an ordinal, then $\alpha^+$ is the least cardinal bigger than $\alpha$.

\begin{lemma}
  $ $

  \begin{enumerate}
    \item For every $\alpha$ there is a cardinal number $\kappa$ such that $\kappa > \alpha$.
    \item Let $X$ be a set of cardinal, then $\sup X$ is a cardinal.
  \end{enumerate}
\end{lemma}

\begin{proof}
  $ $

  \begin{enumerate}
    \item Let $X$ be a set, let 
    \begin{center}
      $h(X) = \text{the least $\alpha$ such that no injection from $\alpha$ into $X$}$
    \end{center}

    Consider $X \times X$, so $2^{X \times X}$ is the set of relations on $X$ and there 
    are well-orderings of subsets of $X$ amongst all relations in $2^{X \times X}$, so consider the set
    \begin{center}
      $Y = \{ R \subseteq Y \times Y \: | \: Y \subseteq X \: \& \: \text{$Y$ is a well-ordering }\}$
    \end{center}

    So there is a set of ordinals:
    \begin{center}
      $Ord(Y) = \{ \alpha \in \operatorname{Ord} \: | \: \exists R \in Y \: \text{$\alpha$ is the order type of $Y$}  \}$
    \end{center}
    Note that $Ord(Y)$ is a set and take the least element ordinal $\beta$ does not belong to $Ord(Y)$. So $h(X) = \beta$.
    To be more precise, we have:
    \begin{center}
      $\beta = \sup Ord(Y)$
    \end{center}

    Then $|\alpha| < h(\alpha)$ for each ordinal $\alpha$.

    \item Let $\alpha = \sup X$. Let $f$ be a one-to-one function from $\alpha$ onto some $\beta < \alpha$.
    Let $\kappa$ be a cardinal such that $\beta < \kappa \leq \alpha$, then $|\kappa| = |\{ f(\xi) \: | \: \xi < \kappa \}| \leq \beta$, 
    so contradiction and $\alpha$ is a cardinal.
  \end{enumerate}
\end{proof}

The enumeration of all alephs is defined by transfinite induction:
\begin{itemize}
  \item $\aleph_0 = \omega$
  \item $\aleph_{\alpha + 1} =\aleph_{\alpha}^+ = \omega_{\alpha+1}$
  \item If $\beta$ is a limit ordinal, then $\aleph_{\beta} = \omega_{\beta} = \sup \{ \omega_{\alpha} \: | \: \alpha < \beta \}$.
\end{itemize}

A cardinal of the form $\aleph_{\alpha + 1}$ is a \emph{successor} cardinal, 
a cardinal $\aleph_{\beta}$ for limit $\beta$ is a \emph{limit cardinal}.

\subsection{The ordering of $\alpha \times \alpha$}

Define a well-ordering of the class $\operatorname{Ord} \times \operatorname{Ord}$ the following way:

\begin{center}
  $(\alpha, \beta) < (\gamma, \delta)$ iff either $\max(\alpha, \beta) < \max(\gamma, \delta)$ or \\ $\max(\alpha, \beta) = \max(\gamma, \delta)$ and $\alpha < \gamma$ or \\ $\max(\alpha, \beta) = \max(\gamma, \delta)$ and $\alpha = \gamma$ and $\beta < \delta$.
\end{center}

Then $<$ is a well-ordering and linear relation on $\operatorname{Ord}$. 
Moreover, $\alpha \times \alpha$ is the initial segment of $(\operatorname{Ord} \times \operatorname{Ord}, <)$ given by $(0, \alpha)$.

We let:

\begin{center}
  $\Gamma(\alpha, \beta) = \text{the order type of } \{ (\xi, \eta) \: | \: (\xi, \eta) < (\alpha, \beta) \}$
\end{center}

$\Gamma$ is also one-to-one:
\begin{center}
  $(\alpha, \beta) < (\gamma, \delta)$ iff $\Gamma(\alpha, \beta) < \Gamma(\gamma, \delta)$
\end{center}

$\Gamma$ is increasing and continuous and $\Gamma(\alpha \times \alpha) = \alpha$ for arbitrarily large $\alpha$.

\begin{theorem}
  $\aleph_{\alpha} \cdot \aleph_{\alpha} = \aleph_{\alpha}$
\end{theorem}

\begin{proof}
  Let us show that $\Gamma(\omega_{\alpha} \times \omega_{\alpha}) = \omega_{\alpha}$. 
  \begin{enumerate}
    \item If $\alpha = 0$, then $\Gamma(\omega \times \omega) = \omega$.
    \item Let $\alpha$ be the least ordinal such that $\Gamma(\omega_{\alpha} \times \omega_{\alpha}) \neq \omega_{\alpha}$.
    Let $\beta, \gamma$ be ordinals such that $\Gamma(\beta, \gamma) = \omega_{\alpha}$.
    Take $\delta < \omega_{\alpha}$ such that $\delta > \beta, \gamma$. 
    $\delta \times \delta$ is the initial segment of $\operatorname{Ord}^2$ and it contains $(\beta, \gamma)$. 
    So $\Gamma(\delta \times \delta) \supset \omega_{\alpha} = \Gamma(\beta, \gamma)$.
    Thus $|\delta \times \delta| \geq \aleph_{\alpha}$. 
    But $|\delta \times \delta| = |\delta| \cdot |\delta| = |\delta|$. But $|\delta| < \aleph_{\alpha}$ by the assumption 
    of minimality of $\alpha$. Contradiction.
  \end{enumerate}
\end{proof}

As a corollary:
\begin{center}
  $\aleph_{\alpha} + \aleph_{\beta} = \aleph_{\alpha} \cdot \aleph_{\beta} = \max(\aleph_{\alpha}, \aleph_{\beta})$
\end{center}

\subsection{Cofinality}

Let $\alpha, \beta > 0$ be limit ordinals. An increasing $\beta$-sequence 
$\langle \alpha_{\xi} : \xi < \beta \rangle$ is \emph{cofinal} in $\alpha$ if $\lim_{\xi \to \beta} \alpha_{\xi} = \alpha$.
A subset $X \subseteq \alpha$ is \emph{cofinal} in $\alpha$ whenever $\sup X = \alpha$.

Let $\alpha > 0$ be a limit ordinal, the \emph{cofinality} of $\alpha$ is:
\begin{center}
  $\operatorname{cf} \alpha = \text{the least ordinal $\beta$ such that $\exists$ 
  $\langle \alpha_{\xi} : \xi < \beta \rangle$ such that $\lim_{\xi \to \beta} \alpha_{\xi} = \alpha$}$
\end{center}

  Note that for each $\alpha$ $\operatorname{cf} \alpha$ is a limit ordinal and $\operatorname{cf} \alpha \leq \alpha$.

\begin{lemma}
  For each $\alpha$ $\operatorname{cf} (\operatorname{cf} \alpha) \leq \operatorname{cf} \alpha$.
\end{lemma}

\begin{proof}
  Let $\langle \alpha_{\xi} : \xi < \beta \rangle$ be cofinal in $\alpha$ and let $\langle \xi_{\nu} : \nu < \gamma \rangle$ be cofinal in $\beta$.

  Consider $\langle \alpha_{\xi_{\nu}} : \nu < \gamma \rangle$, then
  \begin{center}
    $\lim \limits_{\nu < \gamma} \alpha_{\xi_{\nu}} = \alpha$
  \end{center}
  since the limit of a subsequence equals the limit of a sequence as in usual real analysis or topology.
\end{proof}

\begin{lemma}~\label{confin1}
  Let $\alpha$ be a non-zero limit ordinal, then

  \begin{enumerate}
    \item~\label{confin1.1} If $A \subseteq \alpha$ and $\sup A = \alpha$, the order-type of $A$ is at least $\operatorname{cf} \alpha$.
    \item~\label{confin1.2} Let $\beta_0 \leq \beta_1 \leq \dots \leq \beta_{\xi} \leq \dots $ for $\xi < \gamma$ 
    be a non-decreasing sequence of ordinals such that $\lim_{\xi \to \gamma} = \alpha$, then $\operatorname{cf} \gamma = \alpha$.
  \end{enumerate}
\end{lemma}

\begin{proof}
$ $

  \begin{enumerate}
    \item The order-type of $A$ is the length of the increasing enumeration of $A$, 
    the limit of which (as an increasing sequence) is $\alpha$.
    \item If $\gamma = \lim_{\nu \to \operatorname{cf} \gamma} \xi_{\nu}$,
    then $\alpha = \lim_{\nu \to \operatorname{cf} \gamma} \beta_{\xi_{\nu}}$,
    and the non-decreasing sequence $\langle \beta_{\xi_{\nu}} : \nu < \operatorname{cf} \gamma\rangle$
    has an increasing sequence of the length at most $\operatorname{cf} \gamma$ and it has the same limit, 
    so $\operatorname{cf} \alpha \leq \operatorname{cf} \gamma$.

    To show $\operatorname{cf} \gamma \leq \operatorname{cf} \alpha$, 
    assume $\alpha = \lim_{\nu \to \operatorname{cf} \alpha} \alpha_{\nu}$.
    Take $\nu < \operatorname{cf} \alpha$, 
    let $\xi_{\nu}$ be the least $\xi$ greater than all $\xi_{\iota}$ for $\iota < \nu$
    such that $\beta_{\xi} > \alpha_{\nu}$.
    We have $\alpha = \lim_{\nu \to \operatorname{cf} \alpha} \beta_{\xi_{\nu}}$, so
    $\gamma = \lim_{\nu \to \operatorname{cf} \alpha} \xi_{\nu}$, so the inequation is proved.
  \end{enumerate}
\end{proof}

An infinite cardinal $\aleph_{\alpha}$ is \emph{regular} if $\operatorname{cf} \omega_{\alpha} = \omega_{\alpha}$.
$\aleph_{\alpha}$ is \emph{singular} if $\operatorname{cf} \omega_{\alpha} < \omega_{\alpha}$.

\begin{lemma}
  Let $\alpha$ be a limit ordinal, then $\operatorname{cf} \alpha$ is a regular cardinal.
\end{lemma}

\begin{proof}
  If $\alpha$ is not a cardinal, then there exists an ordinal $\beta < \alpha$ such that 
  $|\beta| = |\alpha|$, 
  then we construct a cofinal sequence in $\alpha$ of length $|\beta|$, then $\operatorname{cf} \alpha = |\beta|$ 
  and $\operatorname{cf} \alpha < \alpha$.
\end{proof}

Let $\kappa$ be a limit ordinal, a subset $X \subset \kappa$ is \emph{bounded} if $\sup X < \kappa$ and 
\emph{unbounded} if $\sup X = \kappa$.

\begin{lemma}~\label{boundedaleph} Let $\kappa$ be an aleph, then:

  \begin{enumerate}
    \item If $X \subset \kappa$ and $|X| < \operatorname{cf} \kappa$, then $X$ is bounded.
    \item If $\lambda < \operatorname{cf} \kappa$ and $f : \lambda \to \kappa$, 
    then $\operatorname{Im}f$ is bounded in $\kappa$.
  \end{enumerate}
\end{lemma}

\begin{proof}

  \begin{enumerate}
    \item Let $X$ be such subset of $\kappa$ and assume $X$ is unbounded, so $\sup X = \kappa$.
    By~\ref{confin1.1} of Lemma~\ref{confin1}, the order-type of $X$ is at least $\operatorname{cf} \kappa$, which contradicts to 
    $|X| < \operatorname{cf} \kappa$, so $X$ is bounded.
    \item  Follows from the first item.
  \end{enumerate}
\end{proof}

\begin{lemma}~\label{reg} {\bf (Hausdorff)}

  Let $\kappa$ be a cardinal, then the following are equivalent:

  \begin{enumerate}
    \item $\kappa$ is singular.
    \item There is a cardinal $\lambda < \kappa$ and a family $\{ S_{\xi} \: | \: \xi < \lambda \}$ such that
    each $S_{\xi} \subset \kappa$, $|S_{\xi}| < \kappa$ and $\kappa = \bigcup \limits_{\xi < \lambda} S_{\xi}$.
  \end{enumerate}

\end{lemma}

\begin{proof}
  $ $

  \begin{enumerate}
    \item (1) $\Rightarrow$ (2).

    If $\kappa$ is singular, then there is an increasing sequence 
    $\langle \alpha_{\xi} \: : \: \xi < \operatorname{cf} \kappa \rangle$, so a family of required subsets is
    actually a family of those $\alpha_{\xi}$'s and $\lambda = \operatorname{cf} \kappa$ which is strictly less than $\kappa$
    since $\kappa$ is singular.
    \item (2) $\Rightarrow$ (1).
    
    Let $\lambda$ be the least cardinal such that $\lambda < \kappa$ and there exists a family
    $\{ S_{\xi} \: | \: \xi < \lambda \}$ where each $S_{\xi} \subset \kappa$, $|S_{\xi}| < \kappa$ and
    \begin{center}
      $\kappa = \bigcup \limits_{\xi < \lambda} S_{\xi}$
    \end{center}

    For each $\xi < \lambda$, let $\beta_{\xi}$ be the order-type of 
    $\cup_{\nu < \xi} S_{\nu}$. The sequence $\langle \beta_{\xi} : \xi < \lambda \rangle$ is non-decreasing and 
    each $\beta_{\xi} < \kappa$ for all $\xi < \lambda$ since $\lambda$ is minimal.

    Let us show that $\lim \limits_{\xi \to \kappa} \beta_{\xi} = \kappa$ to show that 
    $\operatorname{cf} \kappa \leq \lambda$.

    Assume $\beta = \lim \limits_{\xi \to \kappa} \beta_{\xi}$. There is a one-to-one mapping 
    $f : \bigcup \limits_{\xi < \beta} S_{\xi} \to \lambda \times \beta$ such that:
    \begin{center}
      $f : \alpha \mapsto (\xi, \gamma)$
    \end{center}
    where $\xi$ is the least ordinal such that $\alpha \in S_{\xi}$ and $\gamma$ is the order-type of $S_{\xi} \cap \gamma$.

    We have $\lambda < \kappa$ and $|\lambda \times \beta| = \lambda \cdot |\beta|$, then $\kappa = \beta$.
  \end{enumerate}
\end{proof}

\begin{theorem}~\label{expcofin}
  Let $\kappa$ be an infinite cardinal, then $\kappa < \kappa^{\operatorname{cf} \kappa}$.
\end{theorem}

\begin{proof}

  Let $F$ be a collection of $\kappa$ functions from $\operatorname{cf} \kappa$ to $\kappa$:
  \begin{center}
    $F = \{ f_{\alpha} : \operatorname{cf} \kappa \to \kappa \: | \: \alpha < \kappa \}$
  \end{center}

  Let us construct $f$ that does not belong to $F$.

  We have $\kappa = \lim_{\xi < \operatorname{cf} \kappa} \alpha_{\xi}$, for $\xi < \operatorname{cf} \kappa$ we let:
  \begin{center}
    $f(\xi) = \text{least $\gamma$ such that $\gamma \neq \forall \alpha < \alpha_{\xi} \: f_{\alpha} \neq \gamma$}$
  \end{center}

  Such $\gamma$ does exist and $f$ is different from all the $f_{\alpha}$.
\end{proof}

An uncountable cardinal $\kappa$ is \emph{weakly inaccessible} if it is limit and regular, but we cannot prove the
existence of weakly inaccessible cardinals in ZFC.

\subsection{Some Exercises}

\begin{exercise}
  The set of all finite sequences in $\mathbb{N}$ is countable.
\end{exercise}

\begin{proof}
  The set of finite sequences is $\bigcup \limits_{n < \omega} \mathbb{N}^n$.
  Each of $\mathbb{N}^n$ is countable for each $n < \omega$, so is the whole union 
  $\bigcup \limits_{n < \omega} \mathbb{N}^n$.
\end{proof}

\begin{exercise}
  $\Gamma(\alpha \times \alpha) \leq \omega^{\alpha}$.
\end{exercise}

\begin{proof}

Induction on $\alpha$.

\begin{enumerate}
\item If $\alpha = 0$, then trivially
\begin{center}
  $\Gamma(0 \times 0) = \Gamma(0) = 0 < \omega^0 = 1$
\end{center}
\item Assume $\alpha = \beta + 1$ and $\Gamma(\beta \times \beta) \leq \omega^{\beta}$.
Take $\gamma(\beta) = \Gamma(\beta \times \beta)$.

Then
\begin{center}
$\gamma(\alpha) = \gamma(\beta + 1) = \gamma(\beta) + \beta + \beta + 1 = \gamma(\beta) + 2 \cdot \beta + 1$.
\end{center}

By the induction hypothesis, we have $\gamma(\beta) \leq \omega^{\beta}$, so
$\gamma(\beta) + 2 \cdot \beta + 1 \leq \omega^{\beta} + 2 \cdot \beta + 1 < \omega^{\beta + 1} = \omega^{\alpha}$.
\end{enumerate}
\item Assume $\alpha = \lim \limits_{\beta \to \alpha} \beta$ and $\Gamma(\beta \times \beta) \leq \omega^{\beta}$ for each $\beta$.

We have
\begin{center}
$\gamma(\alpha) = \gamma(\lim \limits_{\beta \to \alpha} \beta) = \lim \limits_{\beta \to \alpha} (\gamma(\beta)) \leq 
\lim \limits_{\beta \to \alpha} \omega^{\beta} = \omega^{\alpha}$.
\end{center}

\end{proof}

A set $B$ is a \emph{projection} of a set $A$ if there is a mapping of $A$ onto $B$.
$B$ is a projection of $A$ if and only if there is a partition $P$ of $A$ such that $|P| = |B|$.
If $|A| \geq |B| > 0$, then $B$ is a projection of $A$. 
Conversely, by the Axiom of Choice, one can show that $B$ is a projection of $A$, then $|A| \geq |B|$.
This cannot be proved if we assume no the Axiom of Choice.

\begin{exercise}
  Let $B$ a projection of $\omega_{\alpha}$, then $|B| \leq \aleph_{\alpha}$.
\end{exercise}

\begin{proof}
  If $B$ is a projection of $\omega_{\alpha}$, so 
  $\omega_{\alpha} \twoheadrightarrow B$, so $|B| \leq |\omega_{\alpha}| = \aleph_{\alpha}$.
\end{proof}

\begin{exercise}
  If $B$ is a projection of $A$, then $|2^B| \leq |2^A|$.
\end{exercise}

\begin{proof}
  Let $f : A \to B$ maps $A$ onto $B$. Define $g : 2^{B} \to 2^A$ as
  \begin{center}
    $g : X \mapsto f^{-1}(X)$
  \end{center}
  Then $g$ is one-to-one since
  \begin{center}
  $g(X) = g(Y) \leftrightarrow f^{-1}(X) = f^{-1}(Y) \leftrightarrow 
  \{ x \in A \: | \: f(x) \in X \} = \{ y \in A \: | \: f(y) \in Y \} \leftrightarrow X = Y$.
  \end{center}
\end{proof}

\begin{exercise}
  Let $\aleph_{\alpha}$ be an uncountable limit cardinal, then 
  $\operatorname{cf} \omega_{\alpha} = \operatorname{cf} \alpha$; 
  $\omega_{\alpha}$ is the limit of a cofinal sequence $\langle \omega_{\xi} \: : \: \xi < \operatorname{cf} \alpha \rangle$ of cardinals.
\end{exercise}

\begin{proof}
  $\aleph_{\alpha}$ is a limit cardinal, so 
  \begin{center}
  $\aleph_{\alpha} = \sup \{ \omega_{\beta} \: | \: \beta < \alpha \}$
  \end{center}
  But the sequence $\langle \omega_{\beta} \: : \: \beta < \alpha \rangle$ is non-decreasing, so by Lemma~\ref{confin1}(~\ref{confin1.2})
  we have $\operatorname{cf} \omega_{\alpha} = \operatorname{cf} \alpha$.

  Moreover:
  \begin{center}
    $\omega_{\alpha} = \lim \limits_{\xi \to \operatorname{cf} \alpha} \omega_{\xi}$.
  \end{center}
\end{proof}

\section{Real Numbers and The Baire Space}

The \emph{continuum} is the cardinality of $\mathbb{R}$ denoted as $\mathfrak{c}$.

\begin{theorem} {\bf (Cantor)}

  $\aleph_0 < \mathfrak{c}$.
\end{theorem}

\begin{proof}
  One can think of it as a consequence of Theorem~\ref{expcofin}.
\end{proof}

\begin{definition} The \emph{Continuum Hypothesis} (CH) is the following statement:

  \begin{center}
    $\aleph_1 = \mathfrak{c}$.
  \end{center}
\end{definition}

Let $(P, <)$ be an ordered set, a subset $D \subset P$ is a \emph{dense} 
subset of $P$ if $a < b$ in $P$ implies $a < d$ and $d < b$ for some $d \in D$.

\begin{theorem}
  $(\mathbb{R}, <)$ is the unique complete linear ordering that has a 
  countable dense subset isomorphic to $(\mathbb{Q}, <)$.
\end{theorem}

\begin{proof}

  Let $C$ and $C'$ be two complete dense linear orderings and let 
  $P$ and $P'$ be dense in $C$ and $C'$ respectively. 
  Let $f : P \cong P'$, so $f$ can be extended to $f^* : C \cong C'$ by letting:

  \begin{center}
    $f^* : x \mapsto \sup \{ f(p) \: | \: p \in P \: \& \: p \leq x \}$
  \end{center}

  That is, ${(.)}^*$ is functorial.
\end{proof}

The existence of $(\mathbb{R}, <)$ follows from the following general statement:
\begin{theorem}
  Let $(P, <)$ be a dense unbounded linear ordering, then there exists a complete dense unbounded linear ordering 
  $(C, \prec)$ such that:

  \begin{enumerate}
    \item $(P, <)$ embeds to $(C, \prec)$.
    \item $P$ is dense in $C$.
  \end{enumerate}
\end{theorem}

\begin{proof}
  Recall that a \emph{Dedekind cut} in $P$ is a pair $(A, B)$ of disjoint subsets of $P$ such that:
  \begin{enumerate}
    \item $A \cup B = P$.
    \item $\forall a \in A \: \forall b \in B \: a < b$.
    \item $A$ has no greatest element.
  \end{enumerate}

  Let $C$ be the set of all Dedekind cuts in $P$. We let $(A_1, B_1) \preceq (A_2, B_2)$
  if $A_1 \subset A_2$ and $B_2 \subset B_1$. $(C, \preceq)$ is complete.

  Let $\{ C_i \: | \: i \in I \} \neq \emptyset$ be a bounded subset of $C$, then 
  $(\bigcup \limits_i A_i, \bigcap \limits_i B_i)$ is its supremum.

  Let $p \in P$, let
  \begin{center}
    $A_p = \{ x \in P \: | \: x < p \}$

    $B_p = \{ x \in P \: | \: x \geq p \}$
  \end{center}

  Then $(\{ (A_p, B_p) \: | \: p \in P \}, \preceq) \cong (P, <)$ and is dense in $C$.
\end{proof}

$\mathbb{Q}$ is dense in $\mathbb{R}$, so every open interval $(a, b)$ contains some rational number.
Then if $S$ is a disjoint collection of open intervals, then $S$ is at most countable.

Let $P$ be a dense linearly ordered set, if every disjoint collection of open intervals is at most countable, then we say
that $P$ satisfies the \emph{countable chain condition}.

\emph{{\bf (Suslin's Problem)} Let $P$ be a dense linearly ordered set satisfying the countable chain condition. Is $P$ isomorphic to $(\mathbb{R}, <)$?}

Note that neither Suslin's Problem nor its negation can be decided in ZFC.

\subsection{Topology of $\mathbb{R}$}

The real line is equipped with the natural topology induced by the metric $d(a, b) = |b - a|$ coincides with
the order topology on $(\mathbb{R}, <)$. $\mathbb{R}$ is also a complete separable metric space.

Every open set in $\mathbb{R}$ is the union of intervals with rational endpoints, so there are continuum many open sets 
(and the same observation holds for open sets as well).

A subset $P$ is \emph{perfect} is it has no isolated points.

\begin{theorem}
  Every perfect set $P$ has cardinality $\mathfrak{c}$.
\end{theorem}

\begin{proof}
  We construct a one-to-one function $F$ from $\{ 0, 1\}^{\omega}$ to $P$. 
  Let $S$ be the set of all finite binary sequences and let $s \in S$.
  
  By induction on $len(s)$ one can find closed intervals $I_s$ 
  such that for each $n < \omega$ and for each $s \in S$ such that $len(s) = n$:
  \begin{enumerate}
    \item $I_s \cap P$ is perfect,
    \item the diameter of $I_s$ is $\leq 1/2$,
    \item $I_{0:s}, I_{1:s} \subset I_s$ and $I_{0:s} \cap I_{1:s} = \emptyset$
  \end{enumerate}

  Take $f \in \{ 0, 1\}^{\omega}$, the set $P \cap \bigcap \limits_{n < \omega} I_{f \upharpoonright n}$ has exactly one element, so let:
  \begin{center}
    $F : f \mapsto \bigcap \limits_{n < \omega} I_{f \upharpoonright n}$
  \end{center}
\end{proof}

\begin{theorem}~\label{cantor-bendixon} ({\bf Cantor-Bendixon})

 If $F$ is an uncountable closed set, then $F = P \cup S$, where $P$ is perfect and $S$ is at most countable.

\end{theorem}

\begin{proof}

\item Let $F \subset \mathbb{R}$, let
\begin{center}
  $F^{'} = \text{the set of all limit points of $F$}$
\end{center}
$F^{'}$ is also called the \emph{derived set} of $F$. $F'$ is closed and obviously a subset of $A$. 

We let:
\begin{enumerate}
  \item $F_0 = A$.
  \item $F_{\alpha+1} = F^{'}_{\alpha}$.
  \item $F_{\alpha} = \bigcap \limits_{\gamma < \alpha} F_{\gamma}$ if $\alpha > 0$ is a limit ordinal.
\end{enumerate}

Since $F_0 \supset F_1 \supset \dots \supset F_{\alpha} \supset$, so we have an ordinal $\theta$ such that 
$F_{\theta} = F_{\theta + 1}$ (otherwise we could map the proper class of ordinals onto some set).
We let $P = F_{\alpha}$. If $P$ is nonempty, then $P$ is also perfect. 

Let us show that $F - P$ is at most countable. 
Let $\langle J_k : k < \omega \rangle$ be an enumeration of rational intervals. We have
\begin{center}
$F - P = \bigcup \limits_{\alpha < \theta} (F_{\alpha} - F_{\alpha + 1})$
\end{center}
So if $a \in F - P$, then there exists $\alpha < \theta$ such that $a \in F_{\alpha} - F_{\alpha + 1}$, that is,
$a$ is an isolated point of $F_{\alpha}$. We let $k_a$ be the least $k$ such that $a$ 
is the only point of $F_{\alpha}$ in $J_k$. 

If $\alpha \leq \beta$ and $a \neq b$ and $b$ is isolated in $F_{\beta}$, then $b \notin J_{k_a}$, so $k_a \neq k_b$, so
the mapping $a \mapsto k_a$ is one-to-one.

\end{proof}

\begin{col}
  If $C \subseteq \mathbb{R}$ is closed, then either $|C| = 2^{\aleph_0}$ or $|C| \leq \aleph_0$.
\end{col}

A set $A \subset \mathbb{R}$ is \emph{nowhere dense} if $\operatorname{Int}\operatorname{Cl} A = \emptyset$.
The following theorem shows that $\mathbb{R}$ is not of the \emph{first category}, that is,
$\mathbb{R}$ is not the union of a countable family of nowhere dense sets.

\begin{theorem}~\label{baire:category} ({\bf The Baire Category Theorem})

  Let $\{ D_i \: | \: i < \omega \}$ be a countable family of dense open subsets of $\mathbb{R}$, then
  $D = \bigcap \limits_{i < \omega} D_i$ is dense in $\mathbb{R}$.
\end{theorem}

\begin{proof}
  We show that $D \cap I \neq \emptyset$ for each open interval $I$.

  Note that each finite intersection $D_0 \cap D_1 \cap \dots \cap D_n$ is dense and open for each $n < \omega$.
  Let $\langle J_k \: : \: k < \omega \rangle$ be an enumeration of rational intervals.
  
  Let $I_0 := I$ and for each $n$ $I_{n + 1} = J_k = (q_k, r_k)$ 
  where $k$ is the smallest index such that $[q_k, r_k] \subset I_n \cap D_n$.

  Take $a = \lim_{k \to \infty} q_k$, then $a \in I \cap D$.
\end{proof}

\subsection{Borel Sets}

\begin{definition}
  An \emph{algebra of sets} is a collection $\mathcal{S}$ of subsets of $S$ satisfying the following axioms:
  \begin{itemize}
    \item $S \in \mathcal{S}$,
    \item $X, Y \in \mathcal{S} \Rightarrow X \cup Y \in \mathcal{S}$,
    \item $X \in \mathcal{S} \Rightarrow -X \in \mathcal{S}$
  \end{itemize}

  An algebra of sets is {$\sigma$-algebra} if its closed under countable unions:
  \begin{itemize}
    \item Given $\{ X_n \in \mathcal{S} \: | \: n < \omega \}$, then $\bigcup \limits_{n < \omega} X_n 
    \in \mathcal{S}$.
  \end{itemize}
\end{definition}

Let $\mathcal{X}$ be a collection of sets, there is the smallest $\sigma$-algebra containing $\mathcal{X}$.

\begin{definition}
  A set of reals $B$ is \emph{Borel} if it belongs to the smallest $\sigma$-algebra that contains all open sets.
\end{definition}

\subsection{Lebesgue Measure}

The \emph{outer measure} $\mu(X)$ for $X \subset \mathbb{R}^n$ is the infimum of all possible sums of
the form
\begin{center}
  $\sum \limits_{k < \omega} v(I_k)$
\end{center}
where $\{ I_k \: | \: k < \omega \}$ is the collection of $n$-dimensional intervals such that
\begin{center}
  $X \subset \bigcup \limits_{k < \omega} I_k$
\end{center}
where $v(I)$ is the volume of $I$.

Clearly $\mu^*(X) \geq 0$ for each $X \subset \mathbb{R}^n$ and $\mu^*(X)$ could equal $\infty$. A set $X$ is \emph{null}
$\mu^*(X) = 0$.

A set $A$ is \emph{Lebesgue measurable} if for each $X \subset \mathbb{R}^n$ one has
\begin{center}
  $\mu^*(X) = \mu^*(X \cap A) + \mu^*(X - A)$.
\end{center}
The above formula is also knows

If $A$ is measurable, then we write $\mu(A)$ instead of $\mu^*(A)$. We shall call $\mu(A)$ the \emph{Lebesgue measure} of $A$.

Some standard facts about Lebesgue measure:
\begin{enumerate}
  \item Every interval $I$ is Lebesgue measurable and $\mu(I) = v(I)$.
  \item The Lebesgue measurable sets form a $\sigma$-algebra. In particular, every Borel set is Lebesgue measurable.
  \item $\mu$ is $\sigma$-additive. Let $\{ A_n \: | \: n < \omega \}$ be a family of pairwise disjoint and Lebesgue measurable sets, 
  then
  \begin{center}
    $\mu(\bigcup \limits_{n < \omega} A_n) = \sum \limits_{n < \omega} \mu(A_n)$.
  \end{center}
  \item $\mu$ is $\sigma$-finite. If $A$ is Lebesgue measurable, then there are sets $A_n$ for $n < \omega$
  such that $A = \cup_{n < \omega} A_n$ and $\mu(A_n) < \infty$ for each $n < \omega$.
  \item Every null set is measurable.
  \item If $A$ is measurable, then
  \begin{center}
    $\mu(A) = \sup \{ \mu(K) \: | \: K \subset A \: \& \: \text{$K$ is compact}\}$
  \end{center}
\end{enumerate}

\subsection{The Baire Space}

The \emph{Baire Space} is the space $\mathcal{N} = \omega^{\omega}$ of infinite sequences of natural numbers
with the topology defined the following way.
Let $s$ be a finite sequence $s = \langle a_k \: : \: k < n \rangle$, we let:
\begin{center}
  $O(s) = \{ f \in \mathcal{N} \: | \: s \subset f \} = \{ \langle c_k \: | \: k < \omega \rangle \: | \: \forall k < n \: c_k = a_k \}$
\end{center}
All those $O(s)$'s form the open basis for $\mathcal{N}$.

The Baire space is separable and metrisable. The metric is defined as $d(f, g) = 1/2^{n + 1}$ 
where $n$ is the smallest natural number such that $f(n) \neq g(n)$. 
We also have separability since the set of all eventually constant sequences is dense in $\mathcal{N}$.

Every infinite sequence $\langle a_k \: : \: k < \omega \rangle$ defines a continued fraction $1/(a_0 + 1/(a_1 + 1/(a_2 + \dots)))$,
so we have a continuous bijection between infinite sequences and irrational points of the open interval $(0, 1)$. 
Moreover, the Baire space is homeomorphic to the space of irrational numbers.

Now we describe the characterisation of perfect sets in the Baire space.

Let $\operatorname{Seq}$ be the set of all finite sequences in $\mathcal{N}$. 
A \emph{tree} is a set $T \subset \operatorname{Seq}$ satisfying:
\begin{center}
  If $t \in T$ and there exists $n < \omega$ such that $s = t \upharpoonright n$, then $s \in T$.
\end{center}

Let $T$ be a tree, let $[T]$ be the set of all infinite paths through $T$:
\begin{center}
  $[T] = \{ f \in \mathbb{N} \: | \: \forall n < \omega \: f \upharpoonright n \in T \}$
\end{center}

For each $T$, the set $[T]$ is closed in the Baire space. Let $f \in \mathcal{N}$
such that $f \notin [T]$. Then there exists $n < \omega$ such that $s = f \upharpoonright n \notin T$, 
so the open neighbourhood of $f$ $O(s) = \{ g \in \mathcal{N} \: | \: g \supset s \}$. Thus $[T]$ is closed.

Conversely, let $F$ be closed in $\mathcal{N}$, then the set
\begin{center}
  $T_F = \{ s \in \operatorname{Seq} \: | \: \exists f \in F \: s \subset f \}$
\end{center}
is a tree and one can verify that $[T_F] = F$. If $f \in \mathcal{N}$ such that $f \upharpoonright n \in T$
for each $n < \omega$, then for each $n$ there is some $g \in F$ such that 
$g \upharpoonright n = f \upharpoonright n$, so $f \in F$ since $F$ is closed.

If $f$ is an isolated point of a closed set $F$ in $\mathcal{N}$, then there is $n \in \mathbb{N}$ such that no
$g \in F$ such that $g \neq f$ and $g \upharpoonright n = f \upharpoonright n$, so we have no branching starting from the
$n$-th position.

So we have the notion of a perfect set $P$ in the Baire space. A tree $T$ is \emph{perfect} if $t \in T$, then there
exist incomparable $t_1, t_2 \supset t$ such that both of them are in $T$ and neither $t_1 \subset t_2$ nor $t_2 \subset t_1$.

\begin{theorem}
  A closed set $F \subset \mathcal{N}$ is perfect iff the tree $T_F$ is perfect.
\end{theorem}

Let us discuss the Cantor-Bendixon analysis of closed subsets of the Baire space. Let $T$ be a tree, define:
\begin{center}
  $T^{'} = \{ t \in T \: | \: \exists t_1, t_2 \supset t \: (t_1, t_2 \in T \: \& \: \neg (t_1 \subset t_2 \lor t_2 \subset t_1))\}$
\end{center}

Then a set $T$ is perfect iff $T = T^{'} \neq \emptyset$.

$[T] - [T^{'}]$ is at most countable: take $f \in [T]$ such that $f \notin [T^{'}]$.
Take $s_f = f \upharpoonright n$ where $n < \omega$ is the smallest index such that $f \upharpoonright n \notin T'$.
If $f, g \in [T] - [T']$, then $s_f \neq s_g$ by the definition of $T'$, so the mapping $f \mapsto s_f$ is one-to-one.

Now let:
\begin{center}
  $T_0 = T$
  
  $T_{\alpha + 1} = T^{'}_{\alpha}$

  $T_{\alpha} = \bigcap \limits_{\beta < \alpha} T_{\beta}$ if $\alpha > 0$ is limit.
\end{center}

We have $T_0 \supset T_1 \supset \dots \supset T_{\alpha} \supset \dots$. $T_0$ is at most countable, so there is
$\theta < \omega_1$ at which the sequence stabilises. If $T_{\theta} \neq \emptyset$, then $T_{\theta}$ is perfect.

One can verify that:
\begin{center}
  $[\bigcap \limits_{\beta < \alpha} T_{\beta}] = \bigcap \limits_{\beta < \alpha} [T_{\beta}]$
\end{center}
so we have
\begin{center}
  $[T] - [T_{\theta}] = \bigcup \limits_{\beta < \alpha} ([T_{\alpha} - T^{'}_{\alpha}])$
\end{center}
and the set $[T] - [T_{\theta}]$ is at most countable. So we have a version of Theorem~\ref{cantor-bendixon} for the Baire space.

\vspace{\baselineskip}

In descriptive set theory we often deal with the \emph{Lebesgue measure} on the Baire space $\mathcal{N}$.
Such a measure is defined as the extension of the product measure $m$ on Borel sets induced by
the probability measure on $\mathbb{N}$ that measures each singleton $\{ n \}$ as $1/2^{n + 1}$.

So if $s \in \operatorname{Seq}$ such that that $\operatorname{len}(s) \geq 1$ we have
\begin{center}
  $m(O(s)) = \prod \limits_{k = 0}^{n - 1} 1/2^{s(k) + 1}$
\end{center}

\subsection{Polish Space}

\begin{definition}
  A \emph{Polish space} is a topological space homeomorphic to a separable complete metric space.
\end{definition}

The examples of Polish spaces are $\mathbb{R}$, $\mathcal{N}$, the Cantor space, $[0, 1]$,
the Hilbert cube $[0, 1]^{\omega}$.

\subsection{Some Exercises}

\begin{exercise}
  The set of all continuous functions $f : \mathbb{R} \to \mathbb{R}$ has cardinality $\mathfrak{c}$.
\end{exercise}

\begin{proof}
  $\mathbb{Q}$ is dense in $\mathbb{R}$, so for every $a \in \mathbb{R}$ there is a sequence
  $\langle q_n \: | \: n \in \mathbb{N} \rangle$ of rationals such that
  \begin{center}
    $\lim \limits_{n \to \infty} q_n = a$
  \end{center}
  If $f$ is continuous, then
  \begin{center}
  $f(a) = f(\lim \limits_{n \to \infty} q_n) = \lim \limits_{n \to \infty} (f(q_n))$
  \end{center}
  Thus every continuous $f$ is determined by its values at its rational points $\{ f(q) \: | \: q \in \mathbb{Q}\}$.
\end{proof}

\begin{exercise}
  The set of all algebraic numbers is countable.
\end{exercise}

\begin{proof}
  If $a \in \mathbb{R}$ is algebraic, then there is a polynomial $p(x) = c_n x^n + c_{n - 1} x^{n - 1} + \ldots + c_1 x + c_0$ such
  that $p(a) = 0$ and each $c_i \in \mathbb{Z}$. There is a clear bijection between algebraic numbers and finite sequences of integers
  (or sequences of polynomial coefficients), thus the set of all algebraic numbers is countable.
\end{proof}

\begin{exercise}
  Let $S$ be a countable set of reals, then $|\mathbb{R} - S| = \mathfrak{c}$.
\end{exercise}

\begin{proof}
  Consider $|\mathbb{R} \times \mathbb{R} - S|$, then it is going to have the cardinality $\mathfrak{c}$.
\end{proof}

\begin{exercise}
$ $

\begin{enumerate}
  \item The set of irrational numbers has cardinality $\mathfrak{c}$.
  \item The set of transcendental numbers has cardinality $\mathfrak{c}$.
\end{enumerate}
\end{exercise}

\begin{proof}
Both parts are derived from the previous exercise by the obvious instantiation.
\end{proof}

\begin{exercise}
  The set of all open sets of $\mathbb{R}$ has cardinality $\mathfrak{c}$.
\end{exercise}

\begin{proof}
  Each open set is the union of some collection intervals, but such a collection is at most countable
  since the disjoint collection of open intervals is countable as far as $\mathbb{Q}$ is dense $\mathbb{R}$.
\end{proof}

\begin{exercise}
  The Cantor Set is perfect.
\end{exercise}

\begin{proof}
  We remove the open middle thirds from $[0, 1]$ infinitely many times, so we have
  The sequence of sets $\{ C_n \: | \: n < \omega \}$

  where
  \begin{enumerate}
    \item $C_0 = [0, 1]$,
    \item $C_{n + 1} = \frac{C_{n}}{3} \cup (\frac{2}{3} + \frac{C_{n}}{3})$
    \item $C = \bigcap \limits_{n < \omega} C_n$
  \end{enumerate}
  One can approximate any element $c \in C$ by the converging sequence of endpoints, thus $C$ is perfect.
\end{proof}

\begin{exercise}
  Let $P \subset \mathbb{R}$ be a perfect set and let $(a, b)$ be an open interval such that 
  $(a, b) \cap P \neq \emptyset$, then $|P \cap (a, b)| = \mathfrak{c}$.
\end{exercise}

\begin{proof}
  Take $x \in (a, b) \cap P$ and take $\epsilon > 0$ such that $(x - \epsilon, x + \epsilon) \subset (a, b)$,
  but $P$ is perfect, so there is $y \in P$ such that $y \in (x - \epsilon, x + \epsilon)$, 
  so $y \in P \cap (a, b)$.
\end{proof}

\begin{exercise}~\label{uncountable:diff}
  Let $P_1, P_2 \subset \mathbb{R}$ be perfect sets such that $P_1 \not\subset P_2$, then $|P_1 - P_2| = \mathfrak{c}$.
\end{exercise}
\begin{proof}
$- P_2$ is open, so $- P_2$ can be viewed as the union of some intervals. The intersection of $P_1$ with each of them is $\mathfrak{c}$ by
the previous exercise.
\end{proof}

Let $A \subset \mathbb{R}$ be a set of real, then $a \in \mathbb{R}$ is a \emph{condensation point} of $A$
if every neighbourhood of $a$ contains uncountably many elements of $A$. $A^*$ is the set of all condensation points of $A$.

\begin{exercise}~\label{perf:cons}
  If $P$ is perfect, then $P^* = P$.
\end{exercise}

\begin{proof}
   $P$, so every condensation point of $P$ belongs to $P$, so
   $P^* \subset P$. $P$ has no isolated points, every element of $P$
   has a sequence converging to it, so every element of $P$ is a condensation point.
\end{proof}

\begin{exercise}~\label{monotone:2}
  If $F$ is closed and $P \subset F$ is perfect, then $P \subset F^*$.
\end{exercise}

\begin{proof}
  The $^*$ is functorial, so $P^* \subset F^*$, so $P \subset F^*$ by Exercise~\ref{perf:cons}.
\end{proof}

\begin{exercise}
  Let $F$ be an uncountable closed set and let $P$ be a perfect set from Theorem~\ref{cantor-bendixon}, then
  $F^* \subset P$ and thus $F^* = P$.
\end{exercise}

\begin{proof}
  We have $P \subset F^*$ by Exercise~\ref{monotone:2}. Take $a \in F^*$. We need $a \in P$.
  $a$ is a condensation point, $U_a \cap F = \mathfrak{c}$ for any neighbourhood of $a$ and $F - P$ is at most
  countable by the Cantor-Bendixon theorem,
  so $U_a \cap P \neq \emptyset$. So $a$ is a limit point of $P$, but $P$ is closed so $a \in P$.
\end{proof}

\begin{exercise}
  Let $F$ be an uncountable closed set, then $F = F^* \cup (F - F^*)$ is the unique partition of
  $F$ into a perfect set and an at most countable set.
\end{exercise}

\begin{proof}~\label{monotone:2}
  By Cantor-Bendixon theorem, $F^*$ is countable and $(F - F^*)$ is at most countable.
  Let $P \subset F$ be a perfect set, then $P \subset F^*$. If $P$ and $F^*$ are different, then
  by Exercise~\ref{uncountable:diff}, $|P - F^*| = \mathfrak{c}$, which is contradiction, so $P = F^*$, and thus
  such a partition is unique.
\end{proof}

\begin{exercise}
  $\mathbb{Q}$ is not the intersection of a countable collection of open sets.
\end{exercise}

\begin{proof}
  Let 
  \begin{center}
  $\mathbb{Q} = \bigcap \limits_{n < \omega} O_n$
  \end{center}
  where each $O_n \subset \mathbb{R}$ is open. Each $O_n$ is also dense in $\mathbb{R}$ since $\mathbb{Q} \subset O_n$
  for each $n < \omega$.

  Assume we have some numeration $q : \omega \to \mathbb{Q}$ of rationals. Construct the following sequence of intervals:
  \begin{enumerate}
    \item $I_0$ is a closed subinterval of $O_0$ such that $q_0 \notin I_0$.
    \item $I_{n + 1}$ is a closed subinterval of $I_n \cap D_n$ such that $q_{n + 1} \notin I_{n + 1}$.
  \end{enumerate}

  Then $\cap_{n < \omega} I_n \subset \cap_{n < \omega} D_n$, but $\cap_{n < \omega} D_n = \emptyset$ and
  $\cap_{n < \omega} D_n \cap \mathbb{Q} = \emptyset$.
\end{proof}

\begin{exercise}
  Let $f$ be a continuous function and let $B$ be a Borel set, then $f^{-1}(B)$ is Borel.
\end{exercise}

\begin{proof}
  Follows from the definition of a continuous function.
\end{proof}


\section{The Axiom of Choice}

Recall that the axiom of choice (AC) says that if we have a family of sets $S$ such that $\emptyset \notin S$, then
we have a \emph{choice function} on $S$ such that $f(X) \in X$.

In some cases we can show the existence of a choice function without using the axiom of choice. 
For example, for families of a complete lattice, the choice function can return the supremum or infimum of each set
belonging to a family.

Using the axiom of choice one can also show that every infinite set has cardinality equal to $\aleph_{\alpha}$ for some $\alpha$.

\begin{theorem}{({\bf Zermelo})}

  Every set can be well-ordered.
\end{theorem}

\begin{proof}
  Let $A$ be a set. It is sufficient to construct a transfinite sequence 
  $\langle a_{\alpha} \: : \: \alpha < \theta \rangle$ that enumerates $A$. We do that by induction and by using
  the choice function $f$ on non-empty subsets of $A$. For $\alpha$ we let:
  \begin{center}
    $a_{\alpha} = f (A - \{ a_{\xi} \: | \: \xi < \alpha \})$
  \end{center}
  whenever $A - \{ a_{\xi} \: | \: \xi < \alpha \}$ is non-empty. 
  Let $\theta$ be the smallest ordinal such that $A = \{ a_{\alpha} \: | \: \alpha < \theta \}$.
  Thus $\langle a_{\alpha} \: : \: \alpha < \theta \rangle$ enumerates $A$.
\end{proof}

As it is well-known, Zermelo's theorem implies the axiom of choice.
Let $S$ be a family of sets such that $\emptyset \notin S$.
By Zermelo's theorem, we can well-order $\cup S$, so let $f(X)$ be the smallest element of $X$.

Note that Zermelo's theorem also implies that $\mathbb{R}$ can be well ordered and also that $2^{\aleph_0}$ is an aleph
and $2^{\aleph_0} \geq \aleph_1$.

Another important consequence of the axiom of choice:
\begin{theorem}
  The union of a countable family of countable sets is countable.
\end{theorem}

\begin{proof}~\label{countableunion}
  Let $A_n$ be a countable set for each $n < \omega$. For each $n$ let us choose an enumeration 
  $\langle a_{n, k} \: : \: k < \omega \rangle$ of $A_n$. So we have a projection of $\mathbb{N} \times \mathbb{N}$ onto
  $\bigcup \limits_{n < \omega} A_n$ by mapping $(n, k) \mapsto a_{n,k}$.
\end{proof}

In fact, the theorem above can be generalised the following way:
\begin{theorem}~\label{union}
  $|S| \leq S \cdot \sup \{ |X| \: | \: X \in S \}$.
\end{theorem}

\begin{proof}
  Let $\kappa = |S|$ and $\lambda = \sup \{ |X| \: | \: X \in S \}$.
  We have $S = \{ X_{\alpha} \: | \: \alpha < \kappa \}$ and for each $\alpha < \kappa$ we choose an 
  enumeration $X_{\alpha} = \{ a_{\alpha, \beta} \: | \: \beta < \lambda_{\alpha} \}$ where $\lambda_{\alpha} = |X_{\alpha}|$.
  Clearly that $\lambda_{\alpha} \leq \lambda$ for each $\alpha < \kappa$.
  So we have a projection of $\kappa \times \lambda$ onto $\cup S$ by mapping $(\alpha, \beta) \mapsto a_{\alpha, \beta}$.
\end{proof}

\begin{col}
  For every $\alpha$ $\aleph_{\alpha + 1}$ is a regular cardinal.
\end{col}

\begin{proof}
  If $\aleph_{\alpha + 1}$ were singular for some $\alpha$, then 
  $\omega_{\alpha + 1}$ would be the union of at most $\aleph_{\alpha}$ 
  sets of cardinality $\aleph_{\alpha}$ by Lemma~\ref{reg}, which
  would mean that $\aleph_{\alpha + 1} = \aleph_{\alpha}$ by Theorem~\ref{union}.
  Contradiction.
\end{proof}

Let $(P, <)$ be a poset, an element $a \in P$ is \emph{maximal} if no $b \in P$ such that $b > a$.
Let $X$ be a non-empty subset of $P$, then $c$ is the \emph{upper bound} of $X$ if $c \geq X$. 
$X$ is a \emph{chain} in $P$ if any two elements of $X$ are comparable.

\begin{theorem} {\bf (Zorn)}

  Let $(P, <)$ be a poset such that every chain $C$ has an upper bound, then $P$ has a maximal element.
\end{theorem}

\begin{proof}
  Let $f$ be a choice function on non-empty subsets of $P$. We construct a chain $C$ leading to a maximal element.

  Construct the following elements by induction:
  \begin{center}
    $a_{\alpha} = \text{an element of $P$ such that $a_{\alpha} > a_{\xi}$ for every $\xi > \alpha$ if it exists}$
  \end{center}

  If $\alpha > 0$ is a limit ordinal, then $C_{\alpha}$ is a chain in $P$ and $a_{\alpha}$ does exist.
  Eventually, there is $\theta$ such that no $a_{\theta+1} > a_{\theta}$. Thus $a_{\theta}$ is maximal.
\end{proof}

As it is known, Zorn's lemma implies the axiom of choice.
Let $S$ be a family of non-empty sets, then we check that the set
$\{ f \: | \: \text{$f$ is a choice function on some $S' \subset S$}\}$ ordered by inclusion
satisfies the condition of Zorn's lemma, so a maximal element of that poset is a choice function on $S$.


There is a weaker version of the axiom of choice for countable families of non-empty sets. 
The countable AC implies Theorem~\ref{countableunion} and regularity of $\aleph_1$, but the
countable AC is too weak to show that $\mathbb{R}$ can be well-ordered.

There is a stronger version of the countable AC.
\begin{definition} {\bf (The Principle of Dependent Choice (DC))}

  Let $R$ be a binary relation on $A$ such that for all $x \in A$ there exists $y \in A$ such that $y R x$, then there is
  a sequence $a_0, a_1, \dots, a_n, \dots$ for $n < \omega$ such that:
  \begin{center}
    $\forall n < \omega \: (a_{n + 1} R a_n)$
  \end{center}
\end{definition}

The Principle of Dependent Choices allows characterising well orderings and (as well as well-founded relations) the following way:

\begin{lemma}
  Let $(A, <)$ be a poset, then the following are equivalent:

  \begin{enumerate}
    \item $(A, <)$ is a well-ordering.
    \item No infinite sequences $a_0, a_1, \dots, a_n, \dots$ for $n < omega$ such that:
    \begin{center}
      $a_0 > a_1 > \dots > a_n > \dots$
    \end{center}
  \end{enumerate}
\end{lemma}

\subsection{Cardinal Arithmetic the Generalised Continuum Hypothesis}

Now let us discuss the cardinal exponentiation operator.

\begin{lemma}~\label{card1}
  Let $\lambda$ be infinite and $2 \leq \kappa \leq \lambda$, then $\kappa^{\lambda} = 2^{\lambda}$.
\end{lemma}

\begin{proof}
  $2^{\lambda} \leq \kappa^{\lambda} \leq (2^{\kappa})^{\lambda} = 2^{\kappa \cdot \lambda} = 2^{\lambda}$.
\end{proof}

The evaluation of $\kappa^{\lambda}$ is more complicated when $\lambda < \kappa$.
If $2^{\lambda} \geq \kappa$, then we have $\kappa^{\lambda} = 2^{\lambda}$ since $\kappa \leq (2^{\lambda})^{\lambda} = 2^{\lambda}$.
But if $2^{\lambda} < \kappa$, the only thing we can conclude:
\begin{center}
  $\kappa \leq \kappa^{\lambda} \leq 2^{\kappa}$
\end{center}
which is already known by Cantor's theorem.

Let $\lambda$ be a cardinal and let $A$ be a set such that $|A| \geq \lambda$, we let:
\begin{center}
  $[A]^{\lambda} = \{ X \in 2^A \: | \: |X| = \lambda \}$
\end{center}

\begin{lemma}~\label{cardfact1}
  If $|A| = \kappa \geq \lambda$, then the set $[A]^{\lambda}$ has cardinality $\kappa^{\lambda}$.
\end{lemma}

\begin{proof}
  On the one hand every function $f : \lambda \to A$ is a subset of $\lambda \times A$ and $|f| = \lambda$. Thus:
  \begin{center}
    $\kappa^{\lambda} \leq |[\lambda \times A]^{\lambda}| = |[A]^{\lambda}|$
  \end{center}
  On the other hand, there is a one-to-one function $F : [A]^{\lambda} \to A^{\lambda}$. 
  If $X \in [A]^{\lambda}$, let $F(X)$ be some function $f$ on $\lambda$ whose range is $X$.
\end{proof}

Let $\lambda$ be a limit cardinal, let:
\begin{center}
  $\kappa^{<\lambda} = \sup \{\kappa^{\mu} \: | \: \text{$\mu$ is a cardinal such that $\mu < \lambda$}\}$
\end{center}
We also define $\kappa^{<\lambda^{+}}$ for successors $\lambda^{+}$.

Let $\kappa$ be an infinite cardinal and $|A| \geq \kappa$, let:
\begin{center}
  $[A]^{<\kappa} = \{ X \in 2^{A} \: | \: |X| < \kappa \}$
\end{center}
Clearly, the cardinality of $[A]^{<\kappa}$ is $|A|^{<\kappa}$.

\subsection{Infinite Sums and Products}

Let $\{ \kappa_i \: | \: i \in I \}$ be an indexed family of cardinals, define:
\begin{center}
  $\sum \limits_{i \in I} \kappa_i = |\bigcup \limits_{i \in I} X_i|$
\end{center}
where each for $i \in I$ $|X_i| = \kappa_i$. Note that, by the Axiom of Choice, the definition of sum does not depend
on the choice of $\{ X_i \: | \: i \in I \}$.

Let $\lambda, \kappa$ be cardinals and let $\kappa_i = \kappa$, then:
\begin{center}
  $\sum \limits_{i < \lambda} \kappa_i = \lambda \cdot \kappa$
\end{center}

More generally, we have:
\begin{lemma}~\label{cardfact2}
  Let $\lambda$ be an infinite cardinal and $\kappa_i > 0$ for each $i < \lambda$, then:
  \begin{center}
    $\sum \limits_{i < \lambda} \kappa_i = \lambda \cdot \sup \limits_{i < \lambda} \kappa_i$
  \end{center}
\end{lemma}
\begin{proof}
  Let $\kappa = \sup \limits_{i < \lambda} \kappa_i$ and $\sigma = \sum \limits_{i < \lambda} \kappa_i$.
  On the one hand, we have $\forall i < \lambda \:\: \kappa_i \leq \kappa$, so
  \begin{center}
    $\sum \limits_{i < \lambda} \kappa_i \leq \lambda \cdot \kappa$
  \end{center}

  On the other hand, since $\kappa_1 \geq 1$ for each $i$, we have
  \begin{center}
  $\lambda = \sum \limits_{i < \lambda} 1 \leq \sigma$
  \end{center}
  $\sigma \geq \kappa_i$ for each $i$, so we have
  \begin{center}
    $\sigma \geq \sup \limits_{i < \lambda} \kappa_i = \kappa$
  \end{center}
  So $\sigma \geq \lambda \cdot \kappa$.
\end{proof}

Let $\{ X_i \: | \: i \in I \}$ be an indexed family of sets, we let:
\begin{center}
  $\prod \limits_{i \in I} X_i = \{ f \: | \: \text{$f$ is a function on $I$ such that $\forall i \in I \:\: f(i) \in X_i$} \}$
\end{center}
If each of $X_i$'s is non-empty, then the whole product is non-empty and this is equivalent to the axiom of choice.

Let $\{ \kappa_i \: | \: i \in I \}$ be a family of cardinals, define:
\begin{center}
  $\prod \limits_{i \in I} \kappa_i = |\prod \limits_{i \in I} X_i|$
\end{center}
where for each $i$ $X_i$ is a set of cardinality of $\kappa_i$. 
As in the case of sum, assuming the axiom of choice, one can show that the definition of product
does not depend on the choice of $X_i$'s.

If $\kappa_i = \kappa$ for each $i \in I$ and $I$ has cardinality $\lambda$, then:
\begin{center}
  $\prod \limits_{i \in I} \kappa_i = \lambda$
\end{center}

If $I$ is a disjoint union $I = \bigcup \limits_{j \in J} A_j$, then:
\begin{center}
  $\prod \limits_{i \in I} \kappa_i = \prod \limits_{j \in J} (\prod \limits_{i \in A_j} \kappa_i)$
\end{center}

If $\kappa_i \geq 2$ for each $i \in I$, then:
\begin{center}
  $\sum \limits_{i \in I} \kappa_i \leq \prod \limits_{i \in I} \kappa_i$
\end{center}
If $I$ is finite, then the inequality is self-evident. Assume $I$ is infinite. We have:
\begin{center}
  $\prod \limits_{i \in I} \kappa_i \geq \prod \limits_{i \in I} 2 = 2^{|I|} > |I|$
\end{center}
We show that $\sum_i \kappa_i \leq |I| \cdot \prod_i \kappa_i$.

Let $\{ X_i \: | \: i \in I \}$ be a disjoint family such that for each $i \in I$ $|X_i| = \kappa_i$.
Assign each $x \in \bigcup_i X_i$ to a pair $(i, f)$ such that $x \in X_i$ and $f \in \prod_i X_i$
such that $f(i) = x$.

\begin{lemma}~\label{prod:as:sup}
  Let $\lambda$ be an infinite cardinal and let $\langle \kappa_i \: : \: i < \lambda \rangle$ 
  be a non-descreasing sequence of ordinals, then
  \begin{center}
    $\prod \limits_{i \in I} \kappa_i = (\sup \limits_{i \in I} \kappa_i)^{\lambda}$
  \end{center}
\end{lemma}
\begin{proof}
  Let $\kappa = \sup_i \kappa_i$. Since $\kappa_i \leq \kappa$ for each $i < \lambda$, we have:
  \begin{center}
    $\prod \limits_{i \in I} \kappa_i \leq \prod \limits_{i \in I} \kappa = \kappa^{\lambda}$
  \end{center}
  Let us show $\kappa^{\lambda} \leq \prod \limits_{i \in I} \kappa_i$.

  Consider a partition of $\lambda$ into $\lambda$ disjoint sets $A_j$, each of which has cardinality $\lambda$:
  \begin{center}
    $\lambda = \bigcup \limits_{j < \lambda} A_j$
  \end{center}

  For each $j < \lambda$ we have:
  \begin{center}
    $\kappa = \sup \limits_{i \in A_j} \kappa_i \leq \prod \limits_{i \in A_j} \kappa_i$
  \end{center}
  And thus:
  \begin{center}
    $\prod \limits_{i \in I} \kappa_i = \prod \limits_{j < \lambda} (\prod \limits_{i \in A_j} \kappa_i) \geq \prod \limits_{j < \lambda} \kappa = \kappa^{\lambda}$
  \end{center}
\end{proof}

\begin{theorem} {\bf (K\"{o}nig)}

  Assume $\kappa_i < \lambda_i$ for each $i \in I$, then:
  \begin{center}
    $\sum \limits_{i \in I} \kappa_i < \prod \limits_{i \in I} \lambda_i$
  \end{center}
\end{theorem}

\begin{proof}
  Let us show $\Sigma_i \kappa_i \ngeq \Pi_i \lambda_i$. 
  Let $\{ T_i \: | \: i \in I\}$ be an indexed family such that $|T_i| = \lambda_i$.
  It sufficies to show that if we have a family $\{ Z_i \: | \: i \in I\}$ of subsets of $T = \Pi_i T_i$
  such that $|Z_i| < \kappa_i$ for each $i$, then $\cup_i Z_i \neq T$.

  For every $i \in I$, let $S_i$ be the projection of $Z_i$ into the $i$-th coordinate:
  \begin{center}
    $S_i = \{ f(i) \: | \: f \in Z_i \}$
  \end{center}

  As far as $|Z_i| < |T_i|$, we have $S_i \subset T_i$ and $S_i \neq T_i$ for each $i \in I$.
  Let $f \in T$ be a function such that $f(i) \notin S_i$. $f$ does not belong to any $Z_i$, so $\cup_i Z_i \neq T$.
\end{proof}

\begin{col}
  $\kappa < 2^{\kappa}$
\end{col}

\begin{proof}
  $\sum \limits_{i < \kappa} 1 < \prod \limits_{i < \kappa} 2$.
\end{proof}

\begin{col}~\label{cofinality:aleph}
  For each $\alpha$ $\aleph_{\alpha} < \operatorname{cf}(2^{\aleph_{\alpha}})$.
\end{col}

\begin{proof}
  Let us show that if for each $i < \omega_{\alpha}$ $\kappa_i < 2^{\aleph_{\alpha}}$,
  then $\Sigma_{i < \omega_{\alpha}} \kappa_i < 2^{\aleph_{\alpha}}$.
  Let $\lambda_i = 2^{\aleph_{\alpha}}$.
  \begin{center}
    $\sum \limits_{i < \omega_{\alpha}} \kappa_i < \prod \limits_{i < \omega_{\alpha}} \lambda_i = (2^{\aleph_{\alpha}})^{\aleph_{\alpha}} = 2^{\aleph_{\alpha}}$
  \end{center}
\end{proof}

\begin{col}
  For all $\alpha, \beta$ $\aleph_{\beta} < \operatorname{cf}(\aleph_{\alpha}^{\aleph_{\beta}})$.
\end{col}

\begin{proof}
  We show that if $\kappa_i < \aleph_{\alpha}^{\aleph_{\beta}}$ for each 
  $i < \omega_{\beta}$, then $\Sigma_{i < \omega_{\beta}} \kappa_i < \aleph_{\alpha}^{\aleph_{\beta}}$.
  Let $\lambda_i = \aleph_{\alpha}^{\aleph_{\beta}}$, then
  \begin{center}
    $\sum \limits_{i < \omega_{\beta}} \kappa_i < \prod \limits_{i < \omega_{\beta}} \kappa_i = \aleph_{\alpha}^{\aleph_{\beta}}$
  \end{center}
\end{proof}

\begin{col}
  Let $\kappa$ be an infinite cardinal, then $\kappa < \kappa^{\operatorname{cf} \kappa}$
\end{col}

\begin{proof}
  Let $i < \operatorname{cf} \kappa$ and $\kappa_i < \kappa$ be such that $\kappa = \Sigma_{i < \operatorname{cf} \kappa} \kappa_i$.
  \begin{center}
    $\kappa = \sum \limits_{i < \operatorname{cf} \kappa} \kappa_i < \prod \limits_{i < \operatorname{cf} \kappa} \kappa = \kappa^{\operatorname{cf} \kappa}$.
  \end{center}
\end{proof}

\subsection{The Continuum Function}

Cantor's theorem claims that $\aleph_{\alpha} < 2^{\aleph_{\alpha}}$, so $\aleph_{\alpha+1} \leq 2^{\aleph_{\alpha}}$
for each $\alpha$. The \emph{Generalised Continuum Hypothesis} (GCH) is the statement
\begin{center}
  $2^{\aleph_{\alpha}} = \aleph_{\alpha+1}$
\end{center}
for each $\alpha$. GCH is independent of ZFC, but ZFC + GCH proves the following properties of
cardinal exponentiation:
\begin{theorem} Assume GCH. Let $\kappa$ and $\lambda$ be infinite cardinals, then:

  \begin{enumerate}
    \item If $\kappa \leq \lambda$, then $\kappa^{\lambda} = \lambda^{+}$.
    \item If $\operatorname{cf} \kappa \leq \lambda < \kappa$, then $\kappa^{\lambda} = \kappa^{+}$.
    \item If $\lambda < \operatorname{cf} \kappa$, then $\kappa^{\lambda} = \kappa$.
  \end{enumerate}
\end{theorem}

\begin{proof}
$ $

\begin{enumerate}
  \item By Lemma~\ref{card1} we have $\kappa^{\lambda} = 2^{\lambda}$, but $2^{\lambda} = \lambda^{+}$.
  \item Combine Lemma~\ref{cardfact1} and Lemma~\ref{cardfact2}.
  \item By Lemma~\ref{boundedaleph} we have:
  \begin{center}
    $\kappa^{\lambda} = \{ \alpha^{\lambda} \: | \: \alpha < \kappa \}$
  \end{center}
  so:
  \begin{center}
    $|\alpha^{\lambda}| \leq 2^{|\alpha| \cdot \lambda} = (|\alpha| \cdot \lambda)^{+} \leq \kappa$
  \end{center}
\end{enumerate}
\end{proof}

The \emph{beth function} is defined by induction:
\begin{enumerate}
  \item $\beth_0 = \aleph_0$
  \item $\beta_{\alpha + 1} = 2^{\beta_{\alpha}}$
  \item $\beta_{\alpha} = \sup \{ \beta_{\beta} \: | \: \beta < \alpha \}$ if $\alpha$ is limit ordinal.
\end{enumerate}

So we can reword GCH as $\beta_{\alpha} = \aleph_{\alpha}$ for all $\alpha$.

Now we study the behaviour of the continuum function $\kappa \mapsto 2^{\kappa}$ assuming no GCH.

\begin{theorem}~\label{continuum:func} Let $\kappa, \lambda$ be cardinals, then

  \begin{enumerate}
    \item If $\kappa < \lambda$, then $2^{\kappa} \leq 2^{\lambda}$.
    \item $\kappa < \operatorname{cf} 2^{\kappa}$
    \item~\label{continuum:func:3} If $\kappa$ is a limit cardinal, then $2^{\kappa} = (2^{<\kappa})^{\operatorname{cf} \kappa}$
  \end{enumerate}
\end{theorem}

\begin{proof}
  
  $ $

  \begin{enumerate}
    \item Fairly obvious.
    \item Corollary~\ref{cofinality:aleph}.
    \item Let $\kappa = \Sigma_{i < \operatorname{cf} \kappa} \kappa_i$ where each $\kappa_i < \kappa$ for each $i$.
    We have
    \begin{center}
      $2^{\kappa} = 2^{\Sigma_{i < \operatorname{cf} \kappa} \kappa_i} = \prod \limits_{i < \operatorname{cf} \kappa} 2^{\kappa_i} \leq \prod \limits_{i < \operatorname{cf} \kappa} 2^{< \kappa} = (2^{< \kappa})^{\operatorname{cf} \kappa} \leq (2^{\kappa})^{\operatorname{cf} \kappa} \leq 2^{\kappa}$
    \end{center}
  \end{enumerate}
\end{proof}

\begin{col}
  Let $\kappa$ be a singular cardinal. 
  Assume the continuum function is eventually constant below $\kappa$, with value $\lambda$, 
  then $2^{\kappa} = \lambda$.
\end{col}

\begin{proof}
  If $\kappa$ is singular and it satisfies the assumption of the statement, then
  there is $\nu$ such that $\operatorname{cf} \kappa \leq \nu < \kappa$ and that
  $2^{< \kappa} = \lambda = 2^{\nu}$. Thus:
  \begin{center}
    $2^{\kappa} = (2^{<\kappa})^{\operatorname{cf} \kappa} = 2^{\nu}$.
  \end{center}
\end{proof}

The \emph{gimel function} is the function:
\begin{center}
  $\gimel(\kappa) = \kappa^{\operatorname{cf} \kappa}$
\end{center}

If $\kappa$ is a limit cardinal and the continuum function below $\kappa$ is not eventually constant,
then the cardinal $\lambda = 2^{<\kappa}$ is a limit of a non-decreasing sequence:
\begin{center}
  $\lambda = 2^{<\kappa} = \lim \limits_{\alpha \to \kappa} 2^{|\alpha|}$
\end{center}
Then, by Lemma~\ref{confin1}, $\operatorname{cf} \lambda = \operatorname{cf} \kappa$.
Thus, by Theorem~\ref{continuum:func}(\ref{continuum:func:3}), we have:
\begin{center}
$2^{\kappa} = (2^{<\kappa})^{\operatorname{cf} \kappa} = 2^{\operatorname{cf} \lambda}$
\end{center}

If $\kappa$ is regular, then $\kappa = \operatorname{cf} \kappa$ and, since 
$\kappa^{\kappa} = 2^{\kappa}$ we have:
\begin{center}
  $2^{\kappa} = \kappa^{\operatorname{cf} \kappa}$
\end{center}

So we can specify the behaviour of the continuum function in terms of the gimel function.
\begin{col}
  $ $

  \begin{enumerate}
    \item If $\kappa$ is a successor cardinal, then $2^{\kappa} = \gimel(\kappa)$.
    \item If $\kappa$ is a limit cardinal and $\lambda x. 2^x$ below $\kappa$ is eventually constant, then
    $2^{\kappa} = 2^{<\kappa} \cdot \gimel(\kappa)$.
    \item If $\kappa$ is a limit cardinal and $\lambda x. 2^x$ below $\kappa$ is not eventually constant, then
    $2^{\kappa} = \gimel( 2^{<\kappa})$.
  \end{enumerate}
\end{col}

\subsection{Cardinal Exponentiation}

Let $\kappa$ be a regular cardinal and let $\lambda < \kappa$, then every function $f : \lambda \to \kappa$ is bounded,
i.e., $\sup \{ f(\xi) \: | \: \xi < \lambda \} < \kappa$. Thus:
\begin{center}
  $\kappa^{\lambda} = \bigcup \limits_{\alpha < \kappa} \alpha^{\lambda}$
\end{center}
that is,
\begin{center}
  $\kappa^{\lambda} = \sum \limits_{\alpha < \kappa} |\alpha|^{\lambda}$
\end{center}
If $\kappa$ is a successor cardinal, then we obtain the \emph{Hausdorff formula}:
\begin{center}
  $\aleph_{\alpha + 1}^{\beta} = \aleph_{\alpha}^{\aleph_{\beta}} \cdot \aleph_{\alpha + 1}$
\end{center}

We can compute $\kappa^{\lambda}$ using the following fact.
If $\kappa$ is a limit cardinal, we use use the notation:
\begin{center}
  $\lim \limits_{\alpha \to \kappa} \alpha^{\lambda} := \sup \{ \mu^{\lambda} \: | \: \text{$\mu$ is a cardinal and $\mu < \kappa$}\}$
\end{center}

\begin{lemma}~\label{limitcard:lim}
  Let $\kappa$ be a limit cardinal and assume that $\operatorname{cf} \kappa \leq \lambda$, then
  \begin{center}
    $\kappa^{\lambda} = (\lim \limits_{\alpha \to \kappa} \alpha^{\lambda})^{\operatorname{cf} \kappa}$
  \end{center}
\end{lemma}
\begin{proof}
  Let $\kappa = \Sigma_{i < \operatorname{cf} \kappa} \kappa_i$, where $\kappa_i < \kappa$ for each $i$.
  We have:
  \begin{center}
    $\kappa^{\lambda} \leq (\prod \limits_{i < \operatorname{cf} \kappa} \kappa_i)^{\lambda} = 
    \prod \limits_{i < \operatorname{cf} \kappa} \kappa_i^{\lambda} \leq 
    \prod \limits_{i < \operatorname{cf} \kappa} (\lim \limits_{\alpha \to \kappa} \alpha^{\lambda})^{\operatorname{cf} \kappa} \leq (\kappa^{\lambda})^{\operatorname{cf} \kappa} = \kappa^{\lambda}$
  \end{center}
\end{proof}

\begin{theorem}~\label{cardexp}

  Let $\lambda$ be an infinite cardinal, then for all infinite cardinals $\kappa$, 
  the value of $\kappa^{\lambda}$ is computed as follows:

  \begin{enumerate}
    \item $\kappa \leq \lambda$ implies $\kappa^{\lambda} = 2^{\lambda}$.
    \item If there exists $\mu < \kappa$ such that $\kappa \leq \mu^{\lambda}$,
    then $\kappa^{\lambda} = \mu^{\lambda}$.
    \item Assume $\kappa > \lambda$ and if for all $\mu < \kappa$ $\mu^{\lambda} < \kappa$, then:
    \begin{enumerate}
      \item $\operatorname{cf} \kappa > \lambda$ implies $\kappa^{\lambda} = \kappa$.
      \item $\operatorname{cf} \kappa \leq \lambda$ implies $\kappa^{\lambda} = \kappa^{\operatorname{cf} \kappa}$.
    \end{enumerate}
  \end{enumerate}
\end{theorem}

\begin{proof}
  $ $

  \begin{enumerate}
    \item Follows from Lemma~\ref{card1}.
    \item $\mu^{\lambda} \leq \kappa^{\lambda} \leq (\mu^{\lambda})^{\lambda} = \mu^{\lambda}$.
    \item If $\kappa$ is a successor cardinal, then apply the Hausdorff formula.
    If $\kappa$ is a limit cardinal. We have $\kappa = \lim_{\alpha \to \kappa} \alpha^{\lambda}$.

    If $\operatorname{cf} \kappa > \lambda$, then every $f : \lambda \to \kappa$ is bounded and we have:
    \begin{center}
      $\kappa^{\lambda} = \lim \limits_{\alpha \to \kappa} \alpha^{\lambda} = \kappa$.
    \end{center}

    If $\operatorname{cf} \kappa \leq \lambda$, then, by Lemma~\ref{limitcard:lim}, we have:
    \begin{center}
      $\kappa^{\lambda} = (\lim \limits_{\alpha \to \kappa} \alpha^{\lambda})^{\operatorname{cf} \kappa} = \kappa^{\operatorname{cf} \kappa}$
    \end{center}
  \end{enumerate}
\end{proof}

Theorem~\ref{cardexp} allows defining all cardinal exponentiation in terms of the gimel function:
\begin{col}
  Let $\kappa$ and $\lambda$ be cardinals, then the value of $\kappa^{\lambda}$ 
  is either $2^{\lambda}$, or $\kappa$ or $\gimel(\mu)$ for some $\mu$ such that 
  $\operatorname{cf} \mu \leq \lambda < \mu$.
\end{col}

\begin{proof}
  Assume $\kappa^{\lambda} > 2^{\lambda} \cdot \kappa$. 
  Let $\mu$ be the least cardinal such that $\mu^{\lambda} = \kappa^{\lambda}$, so, by Theorem~\ref{cardexp},
  $\mu^{\lambda} = \mu^{\operatorname{cf} \mu}$.
\end{proof}

A cardinal $\kappa$ is a \emph{strong limit} cardinal if
\begin{center}
  $\forall \lambda < \kappa \:\: 2^{\lambda} < \kappa$
\end{center}

Every strong limit cardinal is a limit cardinal, and, assuming the generalised continuum hypothesis, the converse is also true.
If $\kappa$ is a strong limit cardinal, then
\begin{center}
  $\forall \lambda, \nu < \kappa \:\: \lambda^{\nu} < \kappa$
\end{center}
$\aleph_0$ is the smallest strong limit cardinal. Also, strong limit cardinals form a proper class: 
if $\alpha$ is an arbitrary cardinal, then the cardinal
\begin{center}
  $\kappa = \{ \alpha, 2^{\alpha}, 2^{2^{\alpha}}, \dots \}$
\end{center}
(of cofinality $\omega$) is a strong limit cardinal.

Also, if $\kappa$ is a strong limit cardinal, then $2^{\kappa} = \kappa^{\operatorname{cf} \kappa}$.
A cardinal $\kappa$ is \emph{strongly inaccessible} if $\kappa > \aleph_0$, if $\kappa$ is strong limit and regular.
Every strongly inaccessible cardinal is strongly inaccessible, and the converse is true assuming the generalised continuum hypothesis.
Generally, inaccessibility describes the impossibility of being obtained from smaller cardinals by usual set-theoretic operations:
\begin{center}
  $|X| < \kappa \Rightarrow 2^{|X|} < \kappa$.

  $|S| < \kappa$ and $|X| < \kappa$ for each $X \in S$, then $|\cup S| < \kappa$.
\end{center}

\subsection{The Singular Cardinal Hypothesis}

The \emph{Singular Cardinal Hypothesis} (SCH) states that
\begin{center}
  If $\kappa$ is singular, then $2^{\operatorname{cf} \kappa} < \kappa$ implies $2^{\operatorname{cf} \kappa} = \kappa^{+}$.
\end{center}
The singular cardinal hypothesis follows from the generalised continuum hypothesis. 
Indeed, if $\kappa \leq 2^{\operatorname{cf} \kappa}$, then $\kappa^{\kappa} = 2^{\operatorname{cf} \kappa}$.
If $2^{\operatorname{cf} \kappa} < \kappa$, then $\kappa^{+}$ is the least possible value of $\kappa^{\operatorname{cf} \kappa}$.

The singular cardinal hypothesis allows determining cardinal exponentiation by the values of
the continuum function on regular cardinals.

\begin{theorem}~\label{SCH:thm}
  Assume SCH holds, then:

  \begin{enumerate}
    \item If $\kappa$ is a singular cardinal, then:
    \begin{enumerate}
      \item If the continuum function is eventually constant below $\kappa$, then $2^{\kappa} = 2^{< \kappa}$.
      \item $2^{\kappa} = (2^{< \kappa})^+$ otherwise.
    \end{enumerate}
    \item If $\kappa$ and $\lambda$ are infinite cardinals, then:
    \begin{enumerate}
      \item If $\kappa \leq 2^{\lambda}$, then $\kappa^{\lambda} = 2^{\lambda}$.
      \item If $2^{\lambda} < \kappa$, then $\lambda < \operatorname{cf} \kappa$ implies $\kappa = \kappa^{\lambda}$.
      \item If $2^{\lambda} < \kappa$, then $\operatorname{cf} \kappa \leq \lambda$ implies $ \kappa^{\lambda} = \kappa^+$.
    \end{enumerate}
  \end{enumerate}
\end{theorem}

\subsection{Some Exercises}

\begin{exercise}
  There exists a subset $A \subset \mathbb{R}$ such that $A$ has cardinality $2^{\aleph_0}$ but it has no
  perfect subsets. 
\end{exercise}

\begin{proof}
  Let $\langle P_{\alpha} \: : \: \alpha < 2^{\aleph_0} \rangle$ be an enumeration of all perfect
  subsets of reals. Construct the following disjoint sets

  \begin{center}
    $A = \{ a_{\alpha} \: | \: \alpha < 2^{\aleph_0} \}$

    $B = \{ b_{\alpha} \: | \: \alpha < 2^{\aleph_0} \}$
  \end{center}
  the following way. For each $\alpha < 2^{\aleph_0}$ $a_{\alpha}$ and $b_{\alpha}$ are such that:
  \begin{center}
    $a_{\alpha} \notin \{ a_{\xi} \: | \: \xi < \alpha \} \cup \{ b_{\xi} \: | \: \xi < \alpha \}$

    $b_{\alpha} \in P_{\alpha} - \{ a_{\xi} \: | \: \xi < \alpha \}$
  \end{center}

  By the construction $A$ and $B$ are disjoint. Moreover, $A$ is the set. 
  Let $\rho$ be the order-type of $\mathbb{R}$ (which does exist by Zermelo's theorem), so there is a one-to-one
  mapping from $\rho$ to $A$ defined as $\alpha \mapsto a_{\alpha}$ and, by the construction, if 
  $\alpha, \beta < 2^{\aleph_0}$ are different, so are $a_{\alpha}$ and $b_{\beta}$.

  Let $P_{\alpha} \subseteq A$ be a perfect subset. In particular we have $P_{\alpha} \cap B= \emptyset$ by the construction.
  But $b_{\alpha} \in P_{\alpha} \cap B$, which is impossible, or $P_{\alpha} - \{ a_{\xi} \: | \: \xi < \alpha \} = \emptyset$, 
  which is also a contradiction.
\end{proof}

\begin{exercise}
  Let $(P, <)$ be a linear ordering and let $\kappa$ be a cardinal. 
  If every initial segment has cardinality $< \kappa$, then $|P| \leq \kappa$.
\end{exercise}

\begin{proof}
  Let $a \in P$, define $P_a = \{ b \in P \: | \: b \leq a \}$. The condition states that
  $|P_a| < \kappa$ for each $a \in P$.

  Assume $|P| > \kappa$. So we can construct a sequence 
  $\langle a_{\alpha} \: : \: \alpha \leq \kappa \rangle$ by letting
  every $a_{\alpha}$ to be greater that all previous $a_{\beta}$'s for each $\beta < \alpha$.
  Such $a_{\alpha}$ always exists since the union
  \begin{center}
    $\bigcup \limits_{\beta < \alpha} P_{a_{\beta}}$
  \end{center}
  is the union of at most $\kappa$ sets and the cardinality of each of which is at most $\kappa 
  \leq |P|$.

  But then $a_{\kappa}$ is smaller that all $a_{\alpha}$'s for each $\alpha < \kappa$, so $|P_{a_{\kappa}}| = \kappa$, which
  contradicts the assumption.
\end{proof}

\begin{exercise}
  If $A$ can be well-ordered, then $2^A$ can be linearly ordered.
\end{exercise}

\begin{proof}
  Let $X, Y \subset A$, define
  \begin{center}
    $X < Y$ iff the least element of $X \Delta Y$ belongs to $X$.
  \end{center}

  \begin{enumerate}
  \item The relation is clearly irreflexive since $X \Delta X = \emptyset$ for each $X$.

  \item Assume there are sets $X, Y$ such that neither $X < Y$ nor $Y < X$ nor $X = Y$.
  That is, the least element of $X \Delta Y$ belong neither to $X$ nor $Y$ and $X$ and $Y$ are different.
  Let $a$ be the least element of $X \Delta Y$, so $a \notin X$ and $b \notin Y$, so $A$ is not a well-ordering.

  \item This relation is transitive. Assume $X < Y < Z$, we need $X < Z$. 
  $X < Y$ means that the least element of $X \Delta Y$, say $a$, belongs to $X - Y$, whereas
  $Y < Z$ is the case if the least element of $Y \Delta Z$, say $b$ belongs to $Y - Z$.
  Note that the least element exists in every case since $A$ is well-ordered. 
  We also note that $a \neq b$ since $a \in X - Y$ and $b \in Y - Z$, so $a \notin Y$.

 
  We have the following alternatives:
  \begin{itemize}
    \item Assume $a < b$. Take any $x \in A$ such that $x < a$, then 
    $x \in X$ iff $x \in Y$ iff $x \in Z$, so $x \notin X \Delta Z$. But $a \notin Y$ and $a \notin C$, so $a$ is the least in $X \Delta Z$.
    \item Assume $b < a$. Take $x < b$, then $x \in X$ iff $x \in Y$ iff $x \in Z$, so $x \notin X \Delta Z$.
    Provided $b \in X - Y$, we conclude that $b$ is the least in $X \Delta Z$.
  \end{itemize}
  \end{enumerate}
\end{proof}

\begin{exercise}
  Assume the Countable Axiom of Choice, then every infinite set has a countable subset.
\end{exercise}

\begin{proof}
  Let $A$ be infinite, define $A_n = \{ f : n \to A \: | \: \text{$f$ is one-to-one} \}$. Consider
  a family $\{ A_n \: | \: n \in \omega \}$, each of those $A_n$ is non-empty, so we have a choice function
  $f$ on $\cup_{n < \omega} A_n$, so $f(n) : n \to A$, so take $\{ f(n)(0) \: | \: n < \omega \}$ and this set
  is clearly countable. 
\end{proof}

\begin{exercise}
  The Dependent Choice Principle implies The Countable Axiom of Choice.
\end{exercise}

\begin{proof}
  Let $S = \{ A_n \: | \: n < \omega \}$ be a countable family of non-empty sets. 
  Let $\mathcal{S} = \bigcup \limits_{n < \omega} A_n$.
  
  Construct a relation $R \subset \mathcal{S} \times \mathcal{S}$ such that
  \begin{center}
    $(a, b) \in R \leftrightarrow \exists n < \omega \:\: a \in A_{n + 1} \: \& \: b \in A_n$
  \end{center}
  so the relation satisfies the condition of DC, so we have the following sequence:
  
  \begin{center}
    $a_0 R^{-1} a_1 R^{-1} \dots a_n R^{-1} a_{n + 1} R^{-1} \dots $
  \end{center}

  So construct a function $f : \omega \to \mathcal{S}$ such that $f : n \mapsto a_n$.
\end{proof}

\begin{exercise}
  $\prod \limits_{0 < n < \omega} n = 2^{\aleph_0}$.
\end{exercise}

\begin{proof}
  The set of all countable sequences of natural numbers is $2^{\aleph_0}$, 
  but it is also obviously $\Pi_{0 < n < \omega} n$.
\end{proof}

\begin{exercise}
  $\prod \limits_{n < \omega} \aleph_n = \aleph_{\omega}^{\aleph_0}$
\end{exercise}

\begin{proof}
  By Lemma~\ref{prod:as:sup} we have:
  \begin{center}
  $\prod \limits_{n < \omega} \aleph_n = (\sup \limits_{n < \omega} \aleph_n)^{\aleph_0} = \aleph_{\omega}^{\aleph_0}$.
  \end{center}
\end{proof}

\begin{exercise}
  $\prod \limits_{\alpha < \omega + \omega} \aleph_{\alpha} = \aleph^{\aleph_0}_{\omega + \omega}$
\end{exercise}

\begin{proof}
  By Lemma~\ref{prod:as:sup} we have:
  \begin{center}
    $\prod \limits_{\alpha < \omega + \omega} \aleph_{\alpha} = (\sup \limits_{\alpha < \omega + \omega} \aleph_{\alpha})^{\aleph_0} = \aleph_{\omega}^{\aleph_0}$.
  \end{center}
  as far as $|\omega + \omega| = \aleph_0$.
\end{proof}

\begin{exercise}
  Let $\beta$ be such $2^{\aleph_{\alpha}} = \aleph_{\alpha + \beta}$ for each $\alpha$, then $\beta$ is finite.
\end{exercise}

\begin{proof}
  Assume $\beta \geq \alpha$, let $\alpha$ be the least such that $\alpha + \beta < \beta$.
  We have $0 < \alpha \leq \beta$ and $\alpha$ is limit. 
  
  Let $\kappa = \aleph_{\alpha + \alpha}$. $\kappa$ is singular since 
  \begin{center}
  $\operatorname{cf} \kappa = \operatorname{cf} \alpha \leq \alpha < \kappa$
  \end{center}

  For each $\xi < \alpha$ we have $\xi + \beta = \beta$ by our assumption, thus
  \begin{center}
    $2^{\alpha + \xi} = \aleph_{\alpha + \xi + \beta} = \aleph_{\alpha + \beta} = 2^{\kappa}$
  \end{center}
  which is a contradiction since $\aleph_{\alpha + \beta} < \aleph_{\alpha + \alpha+ \beta}$.
\end{proof}

\begin{exercise}
  $\prod \limits_{\alpha < \omega_1 + \omega} \aleph_{\alpha} = \aleph^{\aleph_1}_{\omega_1 + \omega}$.
\end{exercise}

\begin{proof}
  \begin{center}
    $\prod \limits_{\alpha < \omega_1 + \omega} \aleph_{\alpha} = (\sup \limits_{\alpha < \omega_1 + \omega} \aleph_{\alpha})^{\aleph_1} = \aleph^{\aleph_1}_{\omega_1 + \omega}$
  \end{center}
  since $|\omega_1 + \omega| = \aleph_1$.
\end{proof}

\begin{exercise}
  Let $\kappa$ be a limit cardinal and let $\lambda < \operatorname{cf} \kappa$, then
  \begin{center}
    $\kappa^{\lambda} = \sum \limits_{\alpha < \kappa} |\alpha|^{\lambda}$
  \end{center}
\end{exercise}

\begin{proof}
 
As far as $\lambda < \operatorname{cf} \kappa$, then the range of $f : \lambda \to \kappa$ is bounded, so
\begin{center}
$\kappa^{\lambda} = | \bigcup \limits_{\alpha < \kappa} \alpha^{\lambda}| \leq \sum \limits_{\alpha < \kappa} |\alpha|^{\lambda} \leq \kappa \cdot \kappa^{\lambda} = \kappa^{\lambda}$.
\end{center}
\end{proof}

\section{The Axiom of Regularity}

The \emph{Axiom of Regularity} states that the membership relation on any family of sets is well-founded:
\begin{center}
  $\forall S (S \neq \emptyset \to \exists s \in S \: S \cap x = \emptyset)$
\end{center}
that is, no infinite sequences are allowed:
\begin{center}
  $x_0 \ni x_1 \ni x_2 \ni \dots$
\end{center}
neither are cycles:
\begin{center}
  $x_0 \ni x_1 \ni x_2 \ni \dots \ni x_n \ni x_0$
\end{center}

Thus the Axiom of Regularity prevents some sets from existing. This is of interest for metamathematics of set theory, in
particular, we can classify all sets with respect to ranks and arrange them in a cumulative hierarchy.

Recall that a set $A$ is \emph{transitive} if $x \in A$ implies $x \subseteq A$.

\begin{lemma}
  Let $S$ be a set, then there exists a transitive set $T \supset S$.
\end{lemma}
\begin{proof}
  By induction:

  \begin{enumerate}
    \item $S_0 = S$
    \item $S_{n + 1} = \bigcup S_n$
    \item $T = \bigcup \limits_{n < \omega} S_n$
  \end{enumerate}
\end{proof}

$\operatorname{TC}(S)$ is the \emph{transitive closure} of $S$, that is, the minimal transitive set extending $S$.

\begin{lemma}
  Let $C$ be a non-empty class, then $C$ has an $\in$-minimal element.
\end{lemma}

\begin{proof}
  Let $S$ be a set from $C$. If $S \cap C = \emptyset$, then $S$ is minimal.
  Otherwise take $X = T \cap C$ where $T = \operatorname{TC}(S)$ and $X \neq \emptyset$.
  Then $X$ has a minimal $x$ such that $x \cap X = \emptyset$, then $x \cap C = \emptyset$.
\end{proof}

\subsection{The Cumulative Hierarchy of Sets}

We define by transfinite induction:

\begin{enumerate}
  \item $\mathcal{V}_0 = \emptyset$
  \item $\mathcal{V}_{\alpha + 1} = 2^{\mathcal{V}_\alpha}$
  \item $\mathcal{V}_{\alpha} = \bigcup \limits_{\beta < \alpha} \mathcal{V}_{\beta}$
\end{enumerate}

By induction, one can show the following:
\begin{enumerate}
  \item Each $\mathcal{V}_{\alpha}$ is transitive.
  \item $\alpha < \beta$ implies $\mathcal{V}_{\alpha} \subset \mathcal{V}_{\beta}$.
  \item $\alpha \subset \mathcal{V}_{\alpha}$.
\end{enumerate}

\begin{lemma}
  For every $x$ there exists $\alpha$ such that $x \in \mathcal{V}_{\alpha}$:
\begin{center}
  $\bigcup \limits_{\alpha} \mathcal{V}_{\alpha} = \mathcal{V}$
\end{center}
where $V = \{ x \: | \: x = x\}$.
\end{lemma}

\begin{proof}
Let $C$ be the class of all $x$ that no $\alpha$ exists such that $x \in \mathcal{V}_{\alpha}$.
If $C$ is non-empty, then $C$ has an $\in$-minimal element $x$.
That, $x \in C$ and $z \in \cup_{\alpha} \mathcal{V}_{\alpha}$ for some $\alpha$ for each $z \in x$.
Hence $x \subset \cup_{\alpha \in \operatorname{Ord}} \mathcal{V}_{\alpha}$.
By Replacement, there exists $\gamma$ such that $x \subset \cup_{\alpha < \gamma} \mathcal{V}_{\alpha}$, so 
$x \in \mathcal{V}_{\gamma + 1}$. So $C$ cannot be empty.
\end{proof}

Since every $x$ belongs to some $\mathcal{V}_{\alpha}$ for some $\alpha$, we can define \emph{the rank of $x$}:
\begin{center}
  $\operatorname{rank}(x) = \text{the smallest ordinal $\alpha$ such that $x \in \mathcal{V}_{\alpha + 1}$}$
\end{center}
Thus each $V_{\alpha}$ is a collection of sets having lower ranks and we have:
\begin{enumerate}
  \item $x \in y$ implies $\operatorname{rank}(x) < \operatorname{rank}(y)$.
  \item $\operatorname{rank}(\alpha)=\alpha$.
\end{enumerate}

The rank function is often needed when we deal with equivalence classes for equivalence relation on a proper class.
Let $C$ be a class, let
\begin{center}
  $\hat{C} = \{ x \in C \: | \: \forall z \in C \: \operatorname{rank}(x) \leq \operatorname{rank}(x) \}$
\end{center}
Note that $\hat{C}$ is always set and $\hat{C}$ is non-empty whenever $C$ is non-empty.

Let $\equiv$ be an equivalence relation on $C$. Apply the definition above to each equivalence class and define
\begin{center}
  $[x] = \{ y \in C \: | \: y \equiv x \land \forall z \in C \: (z \equiv x \to \operatorname{rank}(y) \leq \operatorname{rank}(z)) \}$
\end{center}
and
\begin{center} 
$C/_{\equiv} = \{ [x] \: | \: x \in C \}$
\end{center}

One can use the same to prove the \emph{Collection Principle}:
\begin{center}
  $\forall X \: \exists Y \:\: (\forall u \in X) [\exists v \varphi(u, v, p) \to (\exists v \in Y) \varphi(u, v, p)]$
\end{center}
where $p$ is a parameter.

We can formulate the collection principle the following way. 
Let $C_u$ be a collection of classes for $u \in X$, where $X$ is a set, then
there exists a set $Y$ such that for every $u \in X$
\begin{center}
  $C_u \neq \emptyset \Rightarrow C_u \cap Y = \emptyset$
\end{center}

To prove the collection principle, we let
\begin{center}
  $Y = \bigcup \limits_{u \in X} \hat{C}_u$
\end{center}
where $C_u = \{ v \: | \: \varphi(u, v, p) \}$, that is, 
\begin{center}
  $v \in Y \leftrightarrow \exists u \in X (\varphi(u, v, p) \: \& \: \forall z (\varphi(u, z, p) \to \operatorname{rank} v \leq \operatorname{rank} z))$
\end{center}

By Replacement, $Y$ is a set.

\subsection{$\in$-induction}

\begin{theorem}~\label{in:ind}
  Let $T$ be a transitive class and let $\Phi$ be a property such that:
  \begin{enumerate}
    \item $\Phi(\emptyset)$
    \item $x \in T \: \& \: \forall z \in x \: \Phi(z) \Rightarrow \Phi(x)$
  \end{enumerate}
  then every element of $T$ satisfies $\Phi$.
\end{theorem}

\begin{proof}
  Let $C$ be the class of all $x \in T$ such that $\Phi$ is not the case for $x$.
  If $C$ is non-empty, then either $\neg \Phi(\emptyset)$ or there
  exists $x \in T$ such that there exists $z \in x$ such that $\Phi(z)$ and $\neg \Phi(x)$.
\end{proof}

\begin{theorem}~\label{in:rec}
  Let $T$ be a non-empty transitive class and let $G$ be a function. Then there exists a unique function $F$ on $T$ such that
  \begin{center}
  $\forall x \in T \:\: F(x) = G(F \upharpoonright x)$
  \end{center}
\end{theorem}

\begin{proof}
  Let $x \in T$, we let $F(x) = y$ if and only if there exists a function $f$ such that 
  $\operatorname{dom}(f)$ is a transitive subset $T$ and 
  \begin{enumerate}
    \item $\forall z \in \operatorname{dom}(f) \:\: f(z) = G(f \upharpoonright z)$
    \item $f(x) = y$
  \end{enumerate}

  The uniqueness is proved by $\in$-induction.
\end{proof}

\begin{col}
  Let $A$ be a class, there is a unique class $B$ such that
  \begin{center}
    $B = \{ x \in A \:| \: x \subset B \}$
  \end{center}
\end{col}

\begin{proof}
  Let
\begin{center}
  $F(x) =
  \begin{cases}
    1, \:\: \text{if $x \in A$ and $F(z) = 1$ for all $z \in x$} \\
    0, \:\: \text{otherwise}
  \end{cases}$
\end{center}

Let $B = \{ x \: |\: F(x) = 1 \}$. The uniqueness is proved by $\in$-induction.
\end{proof}

In such case we say that each $x \in B$ is \emph{hereditarily} in $A$.

The Axiom of Regularity also implies that the universe does not admit non-trivial automorphisms.

\begin{theorem}~\label{trivialautomors}
  Let $T_1$ and $T_2$ be transitive classes and let $\pi$ be an $\in$-automorphism
  of $T_1$ onto $T_2$, i.e. $\pi$ is one-to-one and
  \begin{center}
    $u \in v \leftrightarrow \pi u \in \pi v$
  \end{center}

  Then $T_1 = T_2$ and $\pi u = u$ for every $u \in T_1$.
\end{theorem}
\begin{proof}
  One can show by $\in$-induction that $\pi x = x$ for each $x \in T_1$.
  Assume $\pi z = z$ for each $z \in x$ and let $y = \pi x$.

  We have $x \subset y$, then, as far as $z \in x$, we have $z = \pi z \in \pi x = y$.

  We also have $y \subset x$. Let $t \in y$. Provided $y \subset T_2$, there is
  $z \in T_1$ such that $\pi z = t$. Since $\pi z \in y$, we have $z \in x$ and so
  $t = \pi z = z$. Thus $t \in x$.
  Therefore $\pi x = x$ for each $x \in T_1$ and $T_1 = T_2$.
\end{proof}

\subsection{Well-Founded Relations}

Let $E$ be a binary relation on a class $P$. Let $x \in P$, we let the \emph{extension} of $x$:
\begin{center}
  $\operatorname{ext}_E(x) = \{ z \in P \: | \: z E x \}$
\end{center}

\begin{definition}
  A relation $E$ on $P$ is \emph{well-founded} if
  \begin{enumerate}
    \item Every non-empty set $x \subset P$ has an $E$-minimal element. 
    \item For all $x \in P$ $\operatorname{ext}_E(x)$ is a set.
  \end{enumerate}
\end{definition}

\begin{lemma}
  Let $E$ be a well-founded relation on a class $P$, then every class $C \subset P$ has an $E$-minimal element.
\end{lemma}

\begin{proof}
  We need some $x \in C$ such that $\operatorname{ext}_E(x) \cap x = \emptyset$.
  Let $S \in C$ be arbitrary assume $\operatorname{ext}_E(S) \cap C \neq \emptyset$.
  We let $X = T \cap C$ where
  \begin{enumerate}
    \item $S_0 = \operatorname{ext}_E(S)$ 

    \item $S_{n + 1} = \bigcup \limits_{n} \{ \operatorname{ext}_E(z) \: |\: z \in S_n \}$

    \item $T = \bigcup \limits_{n < \omega} S_n$.
  \end{enumerate}
\end{proof}

The following two theorems are proved similarly to Theorem~\ref{in:ind} and Theorem~\ref{in:rec} respecitvely.

\begin{theorem}
  Let $E$ be a well-founded relation on $P$ and let $\Phi$ be a property such that
  \begin{enumerate}
    \item Every $E$-minimal element of $P$ satisfies $\Phi$.
    \item IF $x \in P$ and if for each $z$ such that $z E x$ $\Phi(z)$ is the case, then $\Phi(x)$ holds.
  \end{enumerate}

  Then $\Phi$ holds for every element of $P$.
\end{theorem}

\begin{theorem}
  Let $E$ be a well-founded relation on $P$. Let $G$ be a function on $\mathcal{V} \times \mathcal{V}$, 
  then there exists a unique function $F$ on $P$ such that for each $x \in P$
  \begin{center}
    $F(x) = G(x, F \upharpoonright \operatorname{ext}_E(x))$.
  \end{center}
\end{theorem}

\begin{example} {\bf (The Rank Function) }

  Let us define, by induction, for all $x \in P$
  \begin{center}
    $\rho(x) = \sup \{ \rho(z) + 1 \: | \: z E x \}$.
  \end{center}
  The codomain of $\rho$ is either a particular ordinal or the class of all ordinals. One has
  \begin{center}
    $\forall x, y \in P \:\: (x E y \to \rho(x) < \rho(y))$.
  \end{center}
\end{example}

\begin{example} {\bf (The Transitive Collapse)}

  By induction, let
  \begin{center}
    $\forall x \in P \:\: \pi(x) = \{ \pi(z) \: | \: z E x \}$.
  \end{center}

  The range of $\pi$ is a transitive class such that
  \begin{center}
    $\forall x, y \in P \:\: (x E y \to \pi(x) \in \pi(y))$
  \end{center}
\end{example}

$\pi$ is one-to-one whenever $E$ is extensional.

\begin{definition}
  A well-founded relation $E$ on a class $P$ is \emph{extensional} if
  \begin{center}
    $\forall x, y \in P \:\: x \neq y \to \operatorname{ext}_E(x) \neq \operatorname{ext}_E(y)$
  \end{center}
\end{definition}
A class $M$ is \emph{extensional} if the membership relation on $M$ is extensional, that is,
\begin{center}
  $\forall x, y \in M \:\: x \neq y \to x \cap M \neq y \cap M$.
\end{center}

\begin{theorem} {\bf (Mostowski's Collapsing Theorem)}

  \begin{enumerate}
    \item Let $E$ be a well-founded relation and extensional relation on a class $P$, then
    there exists a unique transitive class $M$ and a unique isomorphism $\pi : (P, E) \cong (M, \in)$.
    \item Every extensional class $P$ is isomorphic to some transitive class $M$.
    \item~\label{most:iii} In the case of the previous item, if $T \subset P$ is transitive, 
    then $\pi x = x$ for every $x \in T$.
  \end{enumerate}
\end{theorem}

\begin{proof} Let us show the general existence of an isomorphism.

$E$ is well-founded, so we can define $\pi$ by well-founded induction.
That is, take any $x \in P$, then $\pi(x)$ can be defined by $\pi(z)$'s where $z E x$:
\begin{center}
  $\pi(x) = \{ \pi(z) \: | \: z E x \}$
\end{center}
In particular, if $E$ is the membership relation, then $\pi(x)$ becomes
\begin{center}
  $\pi(x) = \{ \pi(z) \: | \: z \in x \cap P \}$.
\end{center}
$\pi$ maps $P$ onto the class $M = \pi(P)$, which turns out to be transitive by the definition of $\pi$.
 
We use extensionality of $E$ in order to show that $\pi$ is one-to-one.
Let $z \in M$ be of least rank such that $z = \pi(x) = \pi(y)$ for some different $x, y \in P$.
$x \neq y$ implies $\operatorname{ext}_E(x) \neq \operatorname{ext}_E(y)$. In other words,
there is some $u \in \operatorname{ext}_E(x)$ such that $u \notin \operatorname{ext}_E(y)$.

Let $t = \pi(u)$. Since $t \in z = \pi(y)$, there is $v \in \operatorname{ext}_E(y)$ such that
$t = \pi(v)$. Thus we have $t = \pi(u) = \pi(v)$ for different $u, v$ of less rank that $z$ has since $t \in z$. Contradiction.

Now it follows that
\begin{center}
  $x E y \leftrightarrow \pi(x) E \pi(y)$
\end{center}
because

  $\begin{array}{lll}
    & xEy \leftrightarrow & \\
    & \:\:\:\: \text{By the definition of $\pi$}& \\
    & \pi(x) E \pi(y) \to & \\
    & \:\:\:\: \text{By the definition of $\pi$}& \\
    & \exists z \: z E y \: \& \: \pi(x) = \pi(z) \to & \\
    & \:\:\:\: \text{As far as $\pi$ is one-to-one}& \\
    & x = z \land x E y & \\
  \end{array}$

The uniqueness of the isomorphism as well as the transitive class $M = \pi(P)$
by Theorem~\ref{trivialautomors}. Let $\pi_1$ and $\pi_2$ be two isomorphisms of $P$ and $M_1$ and $M_2$,
then $\pi_2 \circ \pi_1^{-1} : M_1 \cong M_2$ and therefore $\pi_2 \circ \pi_1^{-1}$ is the identity mapping.
Thus $M_1 = M_2$.

To show (\ref{most:iii}), let $T \subset P$ be transitive. 
Observe $x \subset P$ for every $x \in T$ and $x \cap P = x$ and we have
\begin{center}
  $\pi(x) = \{ \pi(z) \: | \: z \in x \}$
\end{center}
for all $x \in T$. By $\in$-induction one can show that $\pi(x) = x$ for each $x \in T$.
\end{proof}

\subsection{The Bernays-G\"{o}del Axiomatic Set Theory}

In the Bernays-G\"{o}del set theory we consider two types of objects: \emph{sets} (denoted with lowercase letters) and 
\emph{classes} (denoted with uppercase letters).

\begin{enumerate}
  \item Extensionality: $\forall u (u \in X \leftrightarrow u \in Y) \to X = Y$.
  \item Every set is a class.
  \item If $X \in Y$, then $X$ is a set.
  \item Pairing: if $x, y$ are sets, so is $\{x, y\}$.
  \item Let $\varphi$ be a formula where only set variables are quantified, then
  \begin{center}
    $\forall X_1 \dots \forall X_n \exists Y \:\: Y = \{ x \: | \: \varphi(x, X_1, \dots, X_n )\}$
  \end{center}
  \item Infinite: there exists an infinite set.
  \item Union: for every set $x$ $\cup x$ exists.
  \item Powerset: for every set $x$ the powerset $P(x)$ exists.
  \item Replacement: if a class $F$ is a function and $x$ is a set, then $\{ F(z) \: | \: z \in x \}$ is a set.
  \item Regularity.
  \item Choice: any family of non-empty sets has a choice function.
\end{enumerate}

\subsection{Some Exercises}

\begin{exercise}
  $\operatorname{rank}(x) = \sup \{ \operatorname{rank}(z) + 1 \: | \: z \in x \}$
\end{exercise}

\begin{proof}
  If $x = \emptyset$, then 
  \begin{center}
    $\operatorname{rank}(\emptyset) = 1 = \sup \{ \operatorname{rank}(\emptyset) + 1 \: \}$
  \end{center}

  If there is $z \in x$, then assume that 
  \begin{center}
    $\operatorname{rank}(z) = \sup \{ \operatorname{rank}(z') + 1 \: | \: z' \in z \}$
  \end{center}

  On the other hand, we have
  \begin{center}
    $\operatorname{rank}(x) = \text{the least $\alpha$ such that $x \in V_{\alpha + 1}$}$
  \end{center}
  which is $\cup \operatorname{rank}(z) = \sup \{ \operatorname{rank}(z) + 1 \: | \: z \in x \}$.
\end{proof}

\begin{exercise}
  $|\mathcal{V}_{\omega + \alpha}| = \beth_{\alpha}$
\end{exercise}

\begin{proof}

  \begin{enumerate}
    \item $\alpha = 0$, then let us show $|\mathcal{V}_{\omega}| = \aleph_0 = \beth_0$.

    Recall that 
    \begin{center}
      $\mathcal{V}_{\omega} = \bigcup \limits_{n < \omega} \mathcal{V}_n$
    \end{center}
    In turn each of $\mathcal{V}_n$ is finite and we have
    \begin{center}
      $\mathcal{V}_0 \subset \mathcal{V}_1 \subset \dots \mathcal{V}_n \subset \dots$ for $n < \omega$.
    \end{center}
    so the whole union $\cup_{n < \omega} \mathcal{V}_n$ is countable.
    \item Assume $\alpha = \beta + 1$, we have to show that $|\mathcal{V}_{\omega + \beta + 1}| = \beth_{\beta + 1} = 2^{\beth_{\beta}}$

    Assume we have already showed that $|\mathcal{V}_{\omega + \beta}| = \beth_{\beta}$. But then
    
    \begin{center}
    $|\mathcal{V}_{\omega + (\beta + 1)}| = |\mathcal{V}_{(\omega + \beta) + 1}| = |P(\mathcal{V}_{\omega + \beta})| = 2^{\beth_{\beta}}$.
    \end{center}
    \item Let $\alpha = \sup \{\beta \: | \: \beta < \alpha \}$
    and assume $\mathcal{V}_{\omega + \beta} = \beth_{\beta}$ for each $\beta < \omega$.

    We have

    $\begin{array}{lll}
      & |\mathcal{V}_{\omega + \alpha}| = & \\
      & |\mathcal{V}_{\omega + \sup \{ \beta \: | \: \beta < \alpha\}}| = & \\
      & |\mathcal{V}_{\sup \{ \omega + \beta \: | \: \beta < \alpha \}} |= & \\
      & |\bigcup \limits_{\gamma} \mathcal{V}_{\gamma} | = & \\
      & \sup \{ \beth_{\beta} \: | \: \beta < \alpha\} = \beth_{\alpha}.&
    \end{array}$
  \end{enumerate}
\end{proof}

\section{Filters, Ultrafilters and Boolean Algebras}

\begin{definition}
Let $S$ be a non-empty set, a \emph{filter} is a collection $F$ of subsets of $S$ such that:
\begin{itemize}
  \item $S \in F$ and $\emptyset \notin F$,
  \item $A, B \in F$ implies $A \cap B \in F$,
  \item $A \subset B$ and $A \in F$, then $B \in F$. 
\end{itemize}

An \emph{ideal} is a collection $I$ of subsets of $S$ such that:
\begin{itemize}
  \item $\emptyset \in I$ and $S \notin I$,
  \item $A, B \in I$ implies $A \cup B \in I$,
  \item $A \subset B$ and $B \in I$, then $A \in I$.
\end{itemize}
If $I$ is an ideal, then $- I$ is a filter and if $F$ is a filter, then $- F$ is an ideal, so filters and ideals are
dual to each other.
\end{definition}

Here are some examples:
\begin{enumerate}
  \item A \emph{trivial} filter: $F = \{F \}$.
  \item A \emph{principial filter}: Let $X_0$ be a non-emptyset of $F$, the filter
  $\{ X \subset F \: | \: X_0 \subset X \}$ is called the principal filter generated by $X_0$.
  \item There are dual trivial and principal ideals.
  \item The \emph{Fr\'{e}chet filter}: let $S$ be an infinite set and let $I$ be the ideal of all finite subsets. The
  dual filter $F = \{ X \subset S \: | \: |S - X| < \aleph_0 \}$ is called the Fr\'{e}chet filter on $S$.
  Note that the Fr\'{e}chet filter cannot be principal.
  \item Let $A$ be an infinite set and let $S = [A]^{< \omega}$ be the set of all finite subsets of $A$.
  For each $P \in S$ let $\hat{P} = \{ Q \in S \: | \: P \subset Q \}$.

  Let $F$ be the set of all $X \subset S$ such that $X \supset \hat{P}$ for some $P \in S$. 
  Then $F$ is a non-principal filter on $S$.
  \item A set $A \subset \omega$ has \emph{density} $0$ if $\lim \limits_{n \to \infty} |A \cap n| / n = 0$.
  The set of all $A$'s of density $0$ is an ideal.
\end{enumerate}

A family $G$ of sets has \emph{the finite intersection property} if the following holds for each $n < \omega$:
\begin{center}
  $\forall G_0 \dots \forall G_n \: \bigcap \limits_{i < n + 1} G_i \neq \emptyset$.
\end{center}
Every filter satisfies the finite intersection property.

\begin{lemma}~\label{filter:exists}

$ $

  \begin{enumerate}
    \item Let $\mathcal{F}$ be a family of filters on $S$, then $\cap \mathcal{F}$ is a filter on $S$.
    \item Let $\mathcal{C}$ be a $\subset$-chain of filters on $S$, then $\cup \mathcal{C}$ is a filter on $S$.
    \item If $G \subset 2^S$ satisfies the finite intersection property, then there is a filter $F$ on $S$ such that
    $F \supset G$.
  \end{enumerate}
\end{lemma}

\begin{proof}
  $ $

  (i) and (ii) are simple, let $F$ be a set of such subsets $X$ that there are $G_0, \dots, G_n \in G$ such that 
  $\cap_i G_i \subset X$. Then $F$ is a filter.
\end{proof}

Every filter $F \supset G$ contains all finite intersections of sets from $G$, so the filter generated
generated as in the above lemma is the smallest filter containing $G$. In this case we say that a filter 
$F$ is \emph{generated} by $G$.

\begin{definition}
  A filter $U$ on a set $S$ is an \emph{ultrafilter} if
  for every $X \subset S$ either $X \in U$ or $- X \in U$.
\end{definition}
The dual notion is a \emph{prime ideal}, an ideal $I$ is prime if $X \in I$ or $- X \in I$
for every $X \subset I$.

A filter $F$ on $S$ is \emph{maximal} if there is no filter $F' \neq F$ on $S$ such that $F \subset F'$.

\begin{lemma}~\label{ultra:max}
  A filter $F$ is maximal iff $F$ is an ultrafilter.
\end{lemma}

\begin{proof}
  $ $

  \begin{enumerate}
    \item Let $U$ be an ultrafilter. Let $F \supset U$ and $X \in F - U$. Then $S - X \in U$, so $S - X \in F$, contradiction.
    \item Let $F$ be a filter such that $F$ is not an ultrafilter. Let us show that $F$ cannot be maximal.
    Let us $Y \subset S$ such that neither $Y \in F$ nor $S - Y \in F$. 
    Consider $G = F \cup \{ Y\}$, we claim $G$ has the finite intersection property.
    Take $X \in F$, then $X \cap Y \neq \emptyset$, for otherwise we would have $S - Y \supset X$, so $S - Y \in F$.
    Thus if $X_1, \dots, X_n \in F$, then $Y \cap X_1 \cap \dots \cap X_n \neq \emptyset$. 
    Thus $G$ has the finite intersection property, so there exists a filter $F' \supset G$ by Lemma~\ref{filter:exists}. 
    We have $Y \in F' - F$, so $F$ is not maximal.
  \end{enumerate}
\end{proof}

\begin{theorem}~\label{existence} {\bf (Tarski)}

  Every filter can be extended to an ultrafilter.
\end{theorem}

\begin{proof}
  Let $F_0 \subset S$ be an ultrafilter on $S$, let $P$ be the set of all filters extending $F_0$. 
  Consider the poset $(P, \subset)$. It satisfies Zorn's lemma by Lemma~\ref{filter:exists}, so it has
  the maximal element $U$, then $U$ is an ultrafilter by Lemma~\ref{ultra:max}.
\end{proof}
Note that the existence of ultrafilter cannot be shown assuming no the Axiom of Choice. 

Let $s \in S$, then the principal filter generated by the singleton $\{s\}$ is an ultrafilter.
If $S$ is finite, then every filter is an ultrafilter. But there are non-principal ultrafilters as well, for example,
one can apply Theorem~\ref{existence} to the Fr\'{e}chet filter.

If $S$ is infinite and $|S| = \kappa$, then there are at most $2^{2^{\kappa}}$ ultrafilters. 
An ultrafilter $D$ is uniform if $|X| = \kappa$ for each $X \in D$.

\begin{theorem}~\label{how:many:ultra}
  Let $S$ be an infinite set of cardinality $\kappa$, then there are $2^{2^{\kappa}}$ ultrafilters.
\end{theorem}

To show this theorem, we will show the following lemma. A family $\mathcal{A}$ 
of subsets $\kappa$ is \emph{independent} if for any distinct subset $X_1, \dots, X_n, Y_1, \dots, Y_m$ the intersection

\begin{center}
  $X_1 \cap \dots \cap X_n \cap (\kappa - Y_1) \cap \dots \cap (\kappa - Y_m)$
\end{center}
has cardinality $\kappa$.

\begin{lemma}
  There exists an independent family of subsets of $\kappa$ of cardinality $2^{\kappa}$.
\end{lemma}

\begin{proof}
  Let $P$ be the set of all pairs $(F, \mathcal{F})$ where $F \subset \kappa$ is finite and 
  $\mathcal{F}$ is a finite set of finite subsets of $\kappa$. Since $|P| = \kappa$, 
  it is enough to find an independent family $\mathcal{A}$ of subsets of $P$ such that $|\mathcal{A}| = 2^{\kappa}$.

  For each $u \subset \kappa$ let
  \begin{center}
    $X_u = \{ (F, \mathcal{F}) \in P \: | \: F \cap u \in \mathcal{F} \}$
  \end{center}
  and let $\mathcal{A} = \{ X_u \: | \: u \subset \kappa \}$. 
  If $u, v \subset \kappa$ are distinct, so are $X_u$ and $X_v$. Thus $|\mathcal{A}| = 2^{\kappa}$.

  Let us show that $\mathcal{A}$ is independent, let $u_1, \dots, u_n, v_1, \dots, v_m$
  be distinct subsets of $\kappa$. For each $i \leq n$ and $j \leq m$, 
  let $\alpha_{i, j}$ be some element of $\kappa$ such that either 
  $\alpha_{i,j} \in u_i - v_j$ or $\alpha_{i,j} \in v_j - u_i$. 
  Let $F$ be any finite subset of $\kappa$ such that $F \supset \{ \alpha_{i, j} \: | \: i \leq n, j \leq m \}$ and there
  are $\kappa$ many such finite sets.
  We have $F \cap u_i \neq F \cap v_j$ for any $i \leq n$ and $j \leq m$.
  Let $\mathcal{F} = \{ F \cap u_i \: | \: i \leq n \}$, we have $(F, \mathcal{F}) \in X_{u_i}$
  for all $i \leq n$ and $(F, \mathcal{F}) \notin X_{v_j}$ for each $j \leq m$. Consequentely
  \begin{center}
    $|X_{u_1} \cap \dots \cap X_{u_n} \cap X_{v_1} \cap \dots \cap X_{v_m}| = \kappa$.
  \end{center}
\end{proof}

\begin{proof}[Proof of Theorem~\ref{how:many:ultra}]
  Let $\mathcal{A}$ be an independent family of subsets of $\kappa$.
  For every function $f : \mathcal{A} \to \{ 0, 1 \}$, consider the following family
  \begin{center}
    $G_f = \{ X \:| \: |\kappa - X| < \kappa \} \cup \{ X \: | \: f(X) = 1\} \cup \{ \kappa - X \: | \: f(X) = 0 \}$
  \end{center}

  The family $G_f$ satisfies the finite intersection property, 
  so there is an ultrafilter $D_f \supset G_f$ such that $D_f$ is also uniform. If $f \neq g$, 
  then there is $X \in \mathcal{A}$ such that $f(X) \neq g(X)$,so either $X \in D_f$ and $\kappa - X \in D_g$ or vice versa.
  And there are $2^{2^{\kappa}}$ such ultrafilters.
\end{proof}

\subsection{Ultrafilters over $\omega$}

Let $D$ be a non-principal ultrafilter on $\omega$. $D$ is a \emph{$p$-point} if for every partition 
$\{ A_n \: | \: n < \omega \}$ of $\omega$ into $\aleph_0$ pieces such that no $n$ such that $A_n \in D$,
there exists $X \in D$ such that $X \cap A_n$ is finite for each $n < \omega$.

Note that there is a non-principal ultrafilter which is not a $p$-point. Let $\{ A_n \: | \: n < \omega \}$ 
be a partition of $\omega$ into $\aleph_0$ pieces. Let $F$ be the following filter on $\omega$:
\begin{center}
  $X \in F$ iff expect for finitely many $n$, $X \cap A_n$ contains all but finitely many elements of $A_n$.
\end{center}
Let $D$ be an ultrafilter extending $F$, $D$ is not a $p$-point since almost each intersection $X \cap A_n$ is countable.

The below theorem shows that the existence of $p$-points can be shown assuming the Continuum Hypothesis.
Note that there is a model of ZFC where no $p$-points exist.

A nonprincipal ultrafilter $D$ over $\omega$ is a \emph{Ramsey ultrafilter} if every partition 
$\{ A_n \: | \: n < \omega \}$ of $\omega$ into $\aleph_0$ pieces such that $A_n \notin D$ for each $n < \omega$,
there exists $X \in D$ each $X \cap A_n$ is a singleton. Clearly every Ramsey ultrafilter is a $p$-point.

\begin{theorem}
  Assume CH, then there exists a Ramsey ultrafilter.
\end{theorem}

\begin{proof}
  Let $\mathcal{A}_{\alpha}$, for $\alpha < \omega_1$, be an enumeration of all partitions of $\omega$ and let us
  construct an $\omega_1$-sequence of infinite subsets of $\omega$ the following way: 
  given $G_{\alpha}$, let $G_{\alpha + 1} = G_{\alpha}$ be such that either there is $A \in \mathcal{A}$ such that
  $X_{\alpha + 1} \subset A$ or $|X_{\alpha + 1} \cap A| \leq 1$ for each $A \in \mathcal{A}_{\alpha}$.

  If $\alpha$ is a limit ordinal, let $X_{\alpha}$ be such that $X_{\alpha} - X_{\beta}$ is finite for each $\beta < \alpha$
  and such $X_{\alpha}$ exists since $\alpha$ is countable.
  Then
  \begin{center}
    $D = \bigcup \limits_{\alpha < \omega_1} \{ X \: | \: X_{\alpha} \subset X \}$
  \end{center}
  is a Ramsey ultrafilter.
\end{proof}

\subsection{$\kappa$-complete Filters and Ideals}

A filter $F$ on $S$ is \emph{countably complete} ($\sigma$-complete) if whenever $\{ X_n \: | \: n < \omega \} \subset F$, then
\begin{center}
  $\bigcap \limits_{n < \omega} X_n \in F$.
\end{center}

A filter $F$ on $S$ is \emph{$\kappa$-complete} if whenever $\{ X_{\alpha} \: | \: \alpha < \gamma \} \subset F$ for any $\gamma < \kappa$, then
\begin{center}
  $\bigcap \limits_{\alpha < \gamma} X_{\kappa} \in F$.
\end{center}

Note that the notions of $\sigma$-complete and $\aleph_1$-filters are equivalent. $\sigma$-ideals (or \emph{countably complete ideals}) and $\kappa$-complete ideals are defined dually.
A simple example: let $S$ be of cardinality at least $\kappa$, then $I = \{ X \subset S \: | \: |X| < \kappa \}$ is a $\kappa$-complete ideal.

There are no nonprincipal $\sigma$-complete filters on a countable set. If $S$ is uncountable, then
\begin{center}
$\{ X \subset S \: | \: |X| \leq \aleph_0 \}$
\end{center}
is a $\sigma$-ideal on $S$.

Similarly, if $\kappa > \omega$ is regular and $|S| \geq \kappa$, then
\begin{center}
  $\{ X \subset S \: | \: |X| < \kappa \}$
\end{center}
is the smallest $\kappa$-complete ideal on $S$ containing all the singletons $\{ a\}$ for $a \in S$.

\subsection{Boolean Algebras}

\begin{definition}
  A \emph{Boolean algebra} is a set $B$ with at least two elements $0$ and $1$ and operations $+$, $\cdot$ and $-$ such that

  \vspace{\baselineskip}

  \begin{minipage}{0.45\textwidth}
      \begin{itemize}
        \item $u + v = v + u$,
        \item $u + (v + w) = (u + v) + w$,
        \item $u \cdot (v + w) = u \cdot v + u \cdot w$,
        \item $u \cdot (u + v) = u$,
        \item $u + (- u) = 1$.
      \end{itemize}
  \end{minipage}
  \hfill
  \begin{minipage}{0.45\textwidth}
      \begin{itemize}
        \item $u \cdot v = v \cdot u$,
        \item $u \cdot (v \cdot w) = (u \cdot v) \cdot w$,
        \item $u + (v \cdot w) = (u + v) \cdot (u + w)$,
        \item $u + (u \cdot v) = u$,
        \item $u \cdot (- u) = 0$.
      \end{itemize}
  \end{minipage}
\end{definition}

A subset $A$ of a Boolean algebra $B$ is a \emph{subalgebra} of $B$ such that $A$ contains $0$ and $1$ and
is closed under all the required operations. If $X \subset B$, there is a smallest subalgebra $A$ of $B$ containing $X$.
$A$ can be characterised in either of the following ways:
\begin{itemize}
  \item $A = \bigcap \{ C \: | \: X \subset C \subset B \: \& \: \text{$C$ is a subalgebra} \}$.
  \item $A$ is the set of all Boolean combinations of elements of $X$ in $B$.
\end{itemize}

Let $B$ be a Boolean algebra, let $B^+ = B \setminus \{ 0 \}$ denote the set of all non-zero elements of $B$.
Let $a \in B^+$, the set
\begin{center}
  $B \upharpoonright a = \{ u \in B \: | \: u \leq a \}$
\end{center}
is a Boolean algebra, where $+$ and $\cdot$ are the same and the complement of $u \in B \upharpoonright a$ is defined as 
$a - u$.

An element $a \in B$ is an \emph{atom} if $a \in B^{+}$ and if $b < a$ in $B$, then $b = 0$. A Boolean algebra
is \emph{atomic} is every $b \in B^{+}$ has an atom $a$ beneath $b$. 
A Boolean algebra is \emph{atomless} if it has no atoms at all.

Let $B_1$ and $B_2$ be Boolean algebras, a function $h : B_1 \to B_2$ is a homomorphism if it preserves all the operations
and constants.

The set $\operatorname{Im}(B_2) = \{ f(x) \: | \: x \in B_1 \}$ is a subalgebra of $B_2$ and $u \leq v$ impplies
$h(u) \leq h(v)$. A one-to-one homomorphism of $B_1$ onto $B_2$ is an \emph{isomorphism}. 
An \emph{embedding} of $B$ in $C$ is an isomorphism of $B$ onto some subalgebra of $C$.
A one-to-one mapping $h : B_1 \to B_2$ is an isomorphism if and only if
$f(u) \leq f(v)$ implies $u \leq v$.
An isomorphism of a Boolean algebra of $B$ onto itself is called an \emph{automorphism}.

\subsection{Ideals and Filters on Boolean Algebras}

An \emph{ideal} on a Boolean algebra $B$ is a subset $I \subset B$ such that
\begin{itemize}
  \item $0 \in I$,
  \item $a, b \in I \Rightarrow a + b \in I$,
  \item $a \leq b \: \& \: b \in I \Rightarrow a \in I$.
\end{itemize}

A \emph{filter} on a Boolean algebra $B$ is a subset $F \subset B$ such that
\begin{itemize}
  \item $1 \in F$,
  \item $a, b \in F \Rightarrow a \cdot b \in F$,
  \item $a \leq b \: \& \: a \in F \Rightarrow b \in F$.
\end{itemize}

The \emph{trivial} ideal is the singleton $\{ 0 \}$; an ideal is principal if $I = \{ u \in B \: | \: u \leq u_0 \}$ for
some $u_0 \neq 0$. Similarly for filters. A subset $G$ of $B^+$ has \emph{the finite intersection property} if no
$\{ u_1, \dots, u_n\} \subset G$ such that $u_1 \cdot \dots \cdot u_n = 0$. Similarly every such a subset
can be extended to some filter.

Let $h : B_1 \to B_2$ be a Boolean homomorphism, consider the set
\begin{center}
  $I = \{ u \in B \: | \: h(u) = 0 \}$.
\end{center}
is an ideal on $B$ (the \emph{kernel} of the composition). On the other hand, let $I$ be an ideal on $B$, consider the
following equivalence relation:
\begin{center}
  $u \sim v \leftrightarrow u \: \Delta \: v \in I$.
\end{center}
where
\begin{center}
  $u \: \Delta \: v = (u - v) + (v - u)$.
\end{center}

Let $C$ be the set of all equivalence classes, $C = B / \sim$, we equip $C$ with the following operations:
\begin{itemize}
  \item $[u] + [v] = [u + v]$,
  \item $[u] \cdot [v] = [u \cdot v]$,
  \item $-[u] = [- u]$,
  \item $0 = [0]$,
  \item $1 = [1]$.
\end{itemize}

Then $C$ is a Boolean algebra, the \emph{quotient} of $B$ modulo $I$, written $B/I$,
and $C$ is a homomorphic image of $B$ under the mapping $u \mapsto [u]$.

An ideal $I$ is \emph{prime} if for each $u \in B$ either $u \in I$ or $- u \in I$. A ideal is prime
if and only if $I$ is maximal if and only if $B/I = \{ 0, 1\}$.
Theorem~\label{existence} has the following generalisation:
\begin{theorem}~\label{prime:ideal} {\bf (The Prime Ideal Theorem)}

Every ideal on a Boolean algebra $B$ can extended to a prime ideal.
\end{theorem}

\begin{theorem} {\bf (Stone's Representation Theorem)}

Every Boolean algebra is isomorphic to an algebra of sets.
\end{theorem}

\begin{proof}
Let $B$ be a Boolean algebra. We let
\begin{center}
  $S = \{ p \: | \: \text{$p$ is an ultrafilter on $B$ }\}$
\end{center}
By Theorem~\ref{prime:ideal} $S$ is non-empty. Let $u \in B$, define
\begin{center}
$X_u = \{ p \in S \: | \: u \in p \}$
\end{center}
and also let
\begin{center}
  $\mathcal{S} = \{ X_u \: | \: u \in B \}$.
\end{center}

Consider the mapping $\pi(u) = X_u$ from $B$ onto $\mathcal{S}$. Clearly $\pi(1) = S$ and $\pi(0) = \emptyset$.
From the definition of an ultrafilter one has:
\begin{center}
  $\pi(u + v) = \pi(u) \cup \pi(v)$

  $\pi(u \cdot v) = \pi(u) \cap \pi(v)$

  $\pi(- u) = - \pi(u)$
\end{center}

$\pi$ is one-to-one since if $u \neq v$, then by Theorem~\ref{prime:ideal} there exists an ultrafilter $p$
on $B$ that contains $u$ and does not contain $v$. Thus $\pi(u) \neq \pi(v)$.
\end{proof}

\subsection{Complete Boolean Algebras}

Let $B$ be a Boolean algebra and let $X \subset B$, define
\begin{center}
  $\Pi \{ u \: | \: u \in X \} = \inf X$,

  $\Sigma \{ u \: | \: u \in X \} = \sup X$.
\end{center}
and they are well-defined whenever the corresponding supremum and infimum exist. We also define
\begin{center}
  $\Sigma \emptyset = 0$,

  $\Pi \emptyset = 1$.
\end{center}

A Boolean algebra is \emph{complete} if every supremum and infimum exist. 
Let $\kappa$ be an uncountable regular cardinal, then $B$ is \emph{$\kappa$-complete}
if every supremum and infimum of any subset of cardinality $< \kappa$ exist.
An $\aleph_1$-complete Boolean algebra is \emph{$\sigma$-complete} or \emph{countably complete}.

An algebra of sets $\mathcal{S}$ is \emph{$\kappa$-complete} if its closed under unions and
intersections of $< \kappa$ sets. A $\kappa$-complete algebra of sets is a 
$\kappa$-complete Boolean algebra and for every $X \subset \mathcal{S}$ such that $|X| < \kappa$ and $\Sigma X = \bigcup X$.

An ideal $I$ on a $\kappa$-complete Boolean algebra is \emph{$\kappa$-complete} if
\begin{center}
  $\sum \limits_{u \in X} u \in I$.
\end{center}
whenever $X \subset I$ and $|X| < \kappa$. A \emph{$\kappa$-complete filter} is a dual notion.

Let $I$ be a $\kappa$-complete ideal on a $\kappa$-complete Boolean algebra $B$, then $B/I$ is $\kappa$-complete and
\begin{center}
  $\sum \{ [u] \: | \: u \in X \} = [\sum  \{ u \: | \: u \in X \}]$
\end{center}
for every $X \subset B$ such that $|X| < \kappa$.

A \emph{$\sigma$-ideal} is an $\aleph_1$-complete ideal.

There are several important examples of $\sigma$-ideal on the Boolean algebra of all Borel sets of reals:
the $\sigma$-ideal of Borel sets of Lebesgue measure $0$ and the $\sigma$-ideal of meager Borel sets.

Let $A$ be a subalgerba of a Boolean algebra $B$, then $A$ is a \emph{dense} subalgebra of $B$
if for every $u \in B^{+}$ there is a $v \in A^{+}$ such that $v \leq u$.

A \emph{completion} of a Boolean algebra $B$ is a complete algebra $C$ such that $B$ is a dense subalgebra in $C$.

\begin{lemma}
  The completion of a Boolean algebra $B$ is unique up to isomorphism.
\end{lemma}

\begin{proof}
  Let $C$ and $D$ be completions of $B$. Define an isomorphism $\pi : C \cong D$ by
  \begin{center}
    $\pi(c) = \sum^D \{ u \in B \: | \: u \leq c \}$.
  \end{center}
\end{proof}

\begin{theorem}
  Every Boolean algebra has a completion.
\end{theorem}

\begin{proof}
  Let $A$ be a Boolean algebra, a subset $U \subset A^+$ is a \emph{cut} if $p \leq q$ and $q \in U$ implies $p \in U$.
  Associate the cut $U_p = \{ x \in A \: | \: x \leq p \}$ with each $p \in A^+$.
  A cut $U$ is \emph{regular} if $p \notin U$, then there is $q \leq p$ such that $U_q \cap U = \emptyset$.
  Note that every $U_p$ is regular and every cut contains some $U_p$.

  Let $B$ be the set of all regular cuts in $A^+$. $(B, \subset)$ is a complete Boolean algebra.
  The intersection of any family of regular cuts is a regular cut and each cut $U$ is included in at least
  regular cut $\overline{U}$. In fact
  \begin{center}
    $\overline{U} = \{ p \: | \: (\forall q \leq p) U \cap U_q \neq \emptyset \}$
  \end{center}

  Thus for $u, v \in B$ we have
  \begin{center}
    $u \cdot v = u \cap v$

    $u + v = \overline{u + v}$

    $-u = \{ p \: | \: U_p \cap U = \emptyset \}$
  \end{center}
  $\emptyset$ and $A^+$ are the zero and the unit of $B$. Furthermore, for $p, q \in A^{+}$ we have

  \begin{center}
    $U_p + U_q = U_{p + q}$,

    $U_p \cdot U_q = U_{p \cdot q}$,

    $-U_p = U_{-p}$.
  \end{center}
  Thus $A$ embeds in $B$.
\end{proof}

\subsection{Complete and Regular Subalgebras}

Let $B$ be a complete Boolean algebra. A subalgebra $A$ of $B$ is \emph{complete} if every $\Pi X$ and $\Sigma X$ exist
in $A$ for each $X \subset A$. A \emph{complete homomorphism} is a homomorphism $h$ of $B$ into $C$ such that
\begin{center}
  $h(\sum X) = \sum h[X]$

  $h(\prod X) = \prod h[X]$
\end{center}
A \emph{complete embedding} is an embedding that satisfies the condition above. Note that every isomorphism is complete.

As far as the intersection of any family of complete Boolean algebras is a complete Boolean algebra,
every $X \subset B$ is included in the smallest complete Boolean subalgebra containing $X$. 
Such an algebra is said to be \emph{completely generated by $X$}.

\begin{definition}
  A set $W \subset B^+$ is an \emph{antichain} in a Boolean algebra if $u \cdot v = 0$ for each distinct $u, v \in W$.
  If $W$ is an antichain and $\Sigma W = u$, then $W$ is a \emph{partition} of $u$. If $u = 1$, 
  then $W$ is a \emph{maximal antichain} or a \emph{partition}.
\end{definition}

Let $B$ be a Boolean algebra and $A$ is a subalgebra of $B$, then a maximal antichain in $A$ does not have
to be a maximal antichain in $B$, but if every maximal antichain in $A$ is maximal in $B$, then $A$ is a \emph{regular}
subalgebra of $B$.

\subsection{Saturation}

Let $\kappa$ be an infinite cardinal. A Boolean algebra $B$ is \emph{$\kappa$-saturated} if 
there is no partition $W$ of $B$ such that $|W| = \kappa$ and
\begin{center}
  $\operatorname{sat}(B) = \text{the least $\kappa$ such that $B$ is $\kappa$-saturated}$.
\end{center}
$B$ is also said to satisfy the \emph{$\kappa$-chain condition}; as far as $B$ is complete, 
$B$ is $\kappa$-saturated if and only if there is no descending $\kappa$-sequence 
$u_0 > u_1 > \dots > u_{\alpha} > \dots$ for $\alpha < \kappa$ where each $u_{\alpha} \in B$.
The $\aleph_1$-chain condition is called the \emph{countable chain condition}.

\begin{theorem}
  If $B$ is an infinite complete Boolean algebra, then $\operatorname{sat}(B)$ is a regular uncountable cardinal.
\end{theorem}

\begin{proof}
  Let $\kappa = \operatorname{sat}(B)$. Clearly $\kappa$ is uncountable, assume $\kappa$ is singular. We obtain
  a contradiction by providing a partition of size $\kappa$.

  Let $u \in B^+$, let $\operatorname{sat}(u)$ denote $\operatorname{sat}(B_u)$.
  $u \in B$ is \emph{stable} if $\operatorname{sat}(u) = \operatorname{sat}(v)$ for every non-zero $v \leq u$.
  The set $S$ of stable elements is dense in $B$, otherwise there would be a descending sequence 
  $u_0 < u_1 < u_2 < \dots$ with descending cardinals 
  $\operatorname{sat}(u_0) < \operatorname{sat}(u_1) < \operatorname{sat}(u_2) < \dots$.

  Let $T$ be a maximal set of pairwise disjoint elements of $S$. So $T$ is a partition of $B$ and $|T| < \kappa$.
  There are two subcases:
  \begin{enumerate}
    \item There is $u \in T$ such that $\operatorname{sat}(u) = \kappa$. We have
    $\operatorname{cf} \kappa < \kappa$, there is a partition $W$ of $u$ of size 
    $\operatorname{cf} \kappa$: $W = \{ u_{\alpha} \: | \: \alpha < \operatorname{cf} \kappa \}$.
    Let $\kappa_{\alpha}$ for $\alpha < \operatorname{cf} \kappa$ be an increasing sequence with limit $\kappa$.
    For each $\alpha$, $\operatorname{sat}(u_{\alpha}) = \operatorname{sat}(u) = \kappa$. 
    So let $W_{\alpha}$ be a partition of $u_{\alpha}$ of size $\kappa_{\alpha}$. Thus 
    \begin{center}
      $\bigcup \limits_{\alpha < \operatorname{cf} \kappa} W_{\alpha}$
    \end{center}
    is a partition of $u$ of size $\kappa$.
    \item For all $u \in T$ we have $\operatorname{sat}(u) < \kappa$ but $\sup \{ \operatorname{sat}(u) \: | \: u \in T\} = \kappa$.
    Let $\kappa_{\alpha} \to \kappa$ for $\alpha < \operatorname{cf} \kappa$.
    For each $\alpha < \operatorname{cf} \kappa$ we find $u_{\alpha} \in T$ distinct from all $u_{\beta}$,
    $\beta < \alpha$ which admits a partition $W_{\alpha}$ of size $\kappa_{\alpha}$.
    Then
    \begin{center}
      $\bigcup \limits_{\alpha < \operatorname{cf} \kappa} W_{\alpha}$
    \end{center}
    is an antichain $B$ of size $\kappa$.
  \end{enumerate}
\end{proof}

\subsection{Distributivity of Complete Boolean Algebras}

The following distributivity law holds in any complete Boolean algebra:
\begin{center}
  $\sum \limits_{i \in I} u_{0,i} \cdot \sum \limits_{j \in J} u_{1, j} = \sum \limits_{(i, j) \in I \times J} u_{0,i} \cdot u_{1, j}$
\end{center}

To formulate the general distributivity law, let $\kappa$ be a cardinal, then a complete Boolean algebra $B$
is \emph{$\kappa$}-distributive if
\begin{center}
  $\prod \limits_{\alpha < \kappa} \sum \limits_{i \in I_{\alpha}} u_{\alpha, i} = \sum \limits_{f \in \prod \limits_{\alpha < \kappa} I_{\alpha}} \prod \limits_{\alpha < \kappa} u_{\alpha, f(\alpha)}$
\end{center}

Every complete algebra of sets satisfies the general distributivity law. For now, we give two equivalent
formulations of \emph{$\kappa$}-distributivity.

Let $W$ and $Z$ be partitions of $B$, then $W$ is a \emph{refinement} of $Z$ if for every $w \in W$
there is $z \in Z$ such that $w \leq z$. A set $D \subset B$ is \emph{open dense} if $D$ is dense in $B$
and $0 \neq u \leq v \in D$ implies $u \in D$.

\begin{lemma} Let $B$ be a complete Boolean algebra and let $\kappa$ be a cardinal, then the following are equivalent:
  \begin{enumerate}
    \item $B$ is $\kappa$-distributive,
    \item The intersection of $\kappa$ sets open dense subsets of $B$ is open dense in $B$,
    \item Every collection of $\kappa$ refinements of $B$ has a common refinement.
  \end{enumerate}
\end{lemma}

\begin{proof}
  $ $

  \begin{enumerate}
    \item $(i) \to (ii)$.

    Let $D_{\alpha}$ for $\alpha < \kappa$ be open dense and 
    \begin{center}
    $D = \bigcap \limits_{\alpha < \kappa} D_{\alpha}$
    \end{center}
    $D$ is open, so let $u \in B^+$. Let
    \begin{center}
      $\{ u_{\alpha, i} \: | \: i \in I_{\alpha} \} = \{ u \cdot v \: | \: v \in D_{\alpha} \}$
    \end{center}
    then for each $\alpha$ we have
    \begin{center}
      $\sum \limits_{i} u_{\alpha, i} = u = \prod \limits_{\alpha < \kappa} \sum \limits_{i \in I_{\alpha}} u_{\alpha, i}$
    \end{center}
    Take $f \in \Pi_{\alpha} I_{\alpha}$ and let $u_f = \Pi_{\alpha} u_{\alpha, f(\alpha)}$. 
    Each non-zero $u_f$ is in $D$. But $\Sigma_f u_f = u$, so some $u_f$ is non-zero.
    \item $(ii) \to (iii)$.

    Let $W_{\alpha}$ for $\alpha < \kappa$ be partitions of $B$. For each $\alpha$ we let
    \begin{center}
      $D_{\alpha} = \{ u \: | \: \exists v \in W_{\alpha} \:\: u \leq v \}$.
    \end{center}
    Each $D_{\alpha}$ is open dense. Let
    \begin{center}
      $D = \bigcap \limits_{\alpha < \kappa} D_{\alpha}$
    \end{center}
    Let $W$ be a maximal set of pairwise disjoint elements of $D$. 
    $D$ is dense, $W$ is a partition of $B$ and $W$ refines each $W_{\alpha}$.
    \item $(iii) \to (i)$.

    Let $\{ u_{\alpha, i} \: | \: \alpha < \kappa, i \in I_{\alpha} \}$ be a collection of elements
    of $B$. First of all, observe that
    \begin{center}
      $\sum \limits_{f \in \prod \limits_{\alpha < \kappa} I_{\alpha}} \prod \limits_{\alpha < \kappa} u_{\alpha, f(\alpha)} \leq \prod \limits_{\alpha < \kappa} \sum \limits_{i \in I_{\alpha}} u_{\alpha, i}$.
    \end{center}
    Indeed, take $f \in \Pi_{\alpha < \kappa} I_{\alpha}$, let $u_f = \Pi_{\alpha < \kappa} u_{\alpha, f(\alpha)}$.
    We have $u_f \leq u_{\alpha, f(\alpha)}$, so $u_f \leq \Sigma_{i \in I_{\alpha}} u_{\alpha, i}$ 
    for each $\alpha < \kappa$. Thus, for each $\alpha$,
    \begin{center}
      $\sum \limits_{f} u_f \leq \sum \limits_{i} u_{\alpha, i}$
    \end{center}
    and thus
    \begin{center}
      $\sum \limits_{f} \prod \limits_{\alpha} u_{\alpha, f(\alpha)} = \sum \limits_{f} u_f \leq \prod \limits_{\alpha} \sum \limits_{i} u_{\alpha, i}$.
    \end{center}
    So assume (iii) holds. Let $u = \Pi_{\alpha} \Sigma_{i} u_{\alpha, i}$. We need to show that
    $\Sigma_f \Pi_{\alpha} u_{\alpha, f(\alpha)} = u$. Assume $u = 1$. For each $\alpha$ we can replace 
    $\{ u_{\alpha, i} \: | \: i \in I_{\alpha}\}$ by pairwise disjoint 
    $W_{\alpha} = \{ v_{\alpha, i} \: | \: i \in I_{\alpha}\}$ such that $v_{\alpha, i} \leq u_{\alpha, i}$ and
    $\Sigma_i v_{\alpha, i} = \Sigma_i u_{\alpha, i}$.
    Clearly we have $\Sigma_f \Pi_{\alpha} v_{\alpha, f(\alpha)} \leq \Sigma_f \Pi_{\alpha} u_{\alpha, f(\alpha)}$.
    Each $W_{\alpha}$ is a partition of $B$ and there is a partition $W$ refining each $W_{\alpha}$.
    So for each $w \in W$ there is $f$ such that $w \leq \Pi_{\alpha} v_{\alpha, f(\alpha)}$, so $\Sigma_f \Pi_{\alpha} v_{\alpha, f(\alpha)} = 1$.
  \end{enumerate}
\end{proof}

\subsection{Some Exercises}

\begin{exercise}
  If $F$ is a filter and $X \in F$, then $P(X) \cap F$ is a filter on $X$.
\end{exercise}

\begin{proof}
  Let $A \subset B \subset X$ and $A \in 2^X \cap F$, so $A \in F$. We need $B \subset X$ since $B$ obviosly belongs to $F$.

  Let $A, B \in 2^X \cap F$, so $A, B \subset X$ and $A, B \in F$, so $A \cap B \in F$ as well as
  $A \cap B \subset X$.

  Obviously, $X \in 2^X \cap F$. Assume $\emptyset \in 2^X \cap F$, so $\emptyset \in F$, contradiction.

\end{proof}

\begin{exercise}
  If $U$ is an ultrafilter, then $X \cup Y \in U$ implies either $X \in U$ or $Y \in U$.
\end{exercise}

\begin{proof}
  Assume $X \notin U$ and $Y \notin U$, so $- X, -Y \in U$, then $-X \cap -Y \in U$, so contradiction.
\end{proof}

\begin{exercise}
  Let $U$ be an ultrafilter on $S$, then the set $U'$ of all $X \subset S \times S$
  such that $\{ a \in S \: | \: \{ b \in S \:| \: (a, b) \in X \} \in U\} \in U$ is an ultrafilter on $S \times S$.
\end{exercise}

\begin{proof}
  Reword the definition from the condition as follows:
  \begin{center}
    $U' = \{ X \subset S \times S \: | \: \{ a \in S \: | \: \{ b \in S \:| \: (a, b) \in X \} \in U\} \in U \}$
  \end{center}
  \begin{enumerate}
    \item Obviously, $\emptyset \notin U'$ and $S \times S \in U'$.
    \item Let $A, B \in U'$, then we have
    \begin{center}
      $\{ a \in S \: | \: \{ b \in S \:| \: (a, b) \in A \} \in U\} \in U $ 

      $\{ a \in S \: | \: \{ b \in S \:| \: (a, b) \in B \} \in U\} \in U $ 
    \end{center}

    But $U$ is filter, so
    \begin{center}
      $\{ a \in S \: | \: \{ b \in S \:| \: (a, b) \in A \} \in U\} \cap \{ a \in S \: | \: \{ b \in S \:| \: (a, b) \in B \} \in U\} \in U$
    \end{center}
    But the latter is
    \begin{center}
     $\begin{array}{lll}
     & \{ a \in S \: | \: \{ b \in S \:| \: (a, b) \in A \} \in U \land \{ b \in S \:| \: (a, b) \in B \} \in U \} = & \\
     & \{ a \in S \: | \: \{ b \in S \:| \: (a, b) \in A \} \cap \{ b \in S \:| \: (a, b) \in B \} \in U \} = & \\
     & \{ a \in S \: | \: \{ b \in S \: | \: (a, b) \in A \land (a, b) \in B \} \in U\} = & \\
     & \{ a \in S \: | \: \{ b \in S \: | \: (a, b) \in A \cap B \} \in U\}& \\
     \end{array}$
    \end{center}
    which completes the proof.
    \item Let $A \subset B$ and $A \in U'$. We need $B \in U'$, that is, we need
    \begin{center}
      $\{ a \in S \: | \: \{ b \in S \: | \: (a, b) \in B \} \in U \} \in U$.
    \end{center}
    Assume the contrary, then
    \begin{center}
      $\{ a \in S \: | \: \{ b \in S \: | \: (a, b) \notin B \} \in U \} \in U$
    \end{center}
    But $A \in U'$, so
    \begin{center}
      $\{ a \in S \: | \: \{ b \in S \:| \: (a, b) \in A \} \in U\} \in U$
    \end{center}
    $U$ is a filter, so we have
    
    $\begin{array}{lll}
      &\{ a \in S \: | \: \{ b \in S \:| \: (a, b) \in A \} \in U\} \cap \{ a \in S \: | \: \{ b \in S \: | \: (a, b) \notin B \} \in U \} \in U \leftrightarrow& \\
      &\{ a \in S \: | \: \{ b \in S \: | \: (a, b) \in A\} \in U \land \{ b \in S \: | \: (a, b) \notin B \} \in U \} \in U \leftrightarrow & \\
      &\{ a \in S \: | \: \{ b \in S \: | \: (a, b) \in A \land (a, b) \notin B \} \in U \} \in U &
    \end{array}$

    But $A \subset B$, so no such $a, b \in S$ such that $(a, b) \in A$ and $(a, b) \notin B$, so the set above is empty,
    which implies $\empty \in U$, contradiction.

    \item Assume there is $A, B \subset X \times X$ such that
    $A \cup B \in U'$, which means that
    \begin{center}
    $\begin{array}{lll}
    & \{ a \in S \: | \: \{ b \in S \:| \: (a, b) \in A \cup B \} \in U\} \in U = & \\
    & \{ a \in S \: | \: \{ b \in S \:| \: (a, b) \in A \cup B \} \in U\} \in U = & \\
    & \{ a \in S \: | \: \{ b \in S \:| \: (a, b) \in A \} \cup \{ b \in S \: | \: (a, b) \in B \} \in U\} \in U = & \\
    \end{array}$
    \end{center}
    which implies that either $A \in U'$ or $B \in U'$, so $U'$ is prime and thus $U'$ is an ultrafilter.
  \end{enumerate}
\end{proof}

\begin{exercise}
  Let $U$ be an ultrafilter on $S$ and $f : S \to T$, then the set $f_*(U) = \{ X \subset T \: | \: f^{-1}(X) \in U \}$ 
  is an ultrafilter on $T$.
\end{exercise}

\begin{proof}
  \begin{enumerate}
    \item $f^{-1}(\emptyset) = \emptyset \notin U$, so $\emptyset \notin f_*(U)$.
    \item Let $A, B \in f_*(U)$, so $f^{-1}(A), f^{-1}(B) \in U$, so
    $f^{-1}(A) \cap f^{-1}(B) \in U$, but $f^{-1}(A \cap B) \in U$, so $A \cap B \in f_{*}(U)$.
    \item Let $A \in f_*(U)$ and $A \subset B \subset T$.
    So $f^{-1}(A) \in U$, but $U$ is upward closed, so $B \in U$.
    \item Assume $A \cup B \in f_*(U)$, so $f^{-1}(A \cup B) = f^{-1}(A) \cup f^{-1}(B) \in U$, so
    either $f^{-1}(A) \in U$ or $f^{-1}(B) \in U$.
    \item Let $A \subset B$ and $A \in f_*(U)$.
    We have $A = A \cap B$, so $f^{-1}(A \cap B) = f^{-1}(A) \cap f^{-1}(B) \in U$, so $f^{-1}(B) \in U$ as well since
    $f^{-1}(A) \cap f^{-1}(B) \subset f^{-1}(B)$ and $U$ is upward closed.
  \end{enumerate}
\end{proof}

\begin{exercise}
  Let $U$ be an ultrafilter on $\omega$ and let $\langle x_n \: : \: n < \omega \rangle$ be a bounded sequence of real numbers,
  then there exists a unique \emph{$U$-limit} $a = \lim \limits_U a_n$
  such that for every $\epsilon > 0$ $\{ n \: | \: |a_n - a| < \epsilon \} \in U$.
\end{exercise}

\begin{proof}
  Take $a_0$ and $b_0$ such that $a_0 \leq x_n \leq b_0$ for each $n$.
  Let $c_0 = \frac{a_0 + b_0}{2}$, then either
  $\{ n < \omega \: | \: x_n \in [a_0, c_0] \} \in U$ or  $\{ n < \omega \: | \: x_n \in [c_0, b_0] \} \in U$.
  We choose subinterval $[a_1, b_1]$ as the subinterval of $[a_0, c_0]$ and $[c_0, b_0]$
  such that $\{ n < \omega \: | \: x_n \in [a_1, b_1]\}$.

  Now we let $c_1 = \frac{a_1 + b_1}{2}$, let $A = \{ n < \omega \: |\: x_n \in [a_1, c_1] \}$ and $B = 
  \{ n < \omega \: | \: x_n \in [c_1, b_1]\}$. Either $A$ or $B$ belongs to $U$.
  Similarly we define $c_2$ and choose $[a_2, b_2]$.

  Thus we have monotonous sequences $\langle a_n \: : \: n < \omega \rangle$ and 
  $\langle b_n \: : \: n < \omega \rangle$ with the same limit
  \begin{center}
    $a = \lim \limits_{n \to \infty} a_n = \lim \limits_{n \to \infty} b_n$
  \end{center}
  and moreover $\{ n < \omega \: | \: x_n \in [a_1, b_1]\} \in U$.

  We put
  \begin{center}
    $a = \lim \limits_U x_n$
  \end{center}
  Take any $\epsilon > 0$ there is $n < \omega$ such that $[a_n, b_n] \subset (a - \epsilon, a + \epsilon)$, so
  $\{ n < \omega \: | \: x_n \in [a_n, b_n] \} \in U$ as far as $\{ n < \omega \: | \: x_n \in [a_1, b_1]\} \in U$.
  So  $\{ n \: | \: |a_n - a| < \epsilon \} \in U$.
\end{proof}

\begin{exercise}
  Let $D$ be a nonprincipal ultrafilter on $\omega$, then $D$ is a $p$-point if and only if the following holds:
  if $A_0 \supset A_1 \supset \ldots \supset A_n \supset \dots$ is a decreasing sequence of elements of $D$, then
  there exists $X \in D$ such that for each $n$ $X - A_n$ is finite.
\end{exercise}

\begin{proof}
$ $ 

\begin{enumerate}
  \item The "if" part.
  
  Assume there is a partition $\{ P_n \: | \: n < \omega \}$ disjoint from $D$.
  Construct a family of sets $\{ X_n \: | \: n < \omega \}$ where for each $n < \omega$:
  \begin{center}
  $X_n = \bigcup_{m \geq n} P_m$
  \end{center}
  So we have the increasing sequence $X_0 \subset X_1 \subset \ldots \subset X_n \subset \dots$
  where no such $n < \omega$ that $X_n \in D$. Thus we have the decreasing sequence
  $- X_0 \supset - X_1 \supset \ldots \supset - X_n \supset \cdots$, where each $- X_0$ belongs to $D$. 
  Then there is $X \in D$ such that for each $n < \omega$ $X - (- X_n) = X \cap X_n$ is finite.

  So one can show by induction that each intersection $X \cap P_n$ is finite.
  \item The "only if part".

  Let $A_0 \supset A_1 \supset \ldots \supset A_n \supset \dots$ be a decreasing sequence of elements in $D$,
  construct a family of sets $\{ P_n \: | \: n < \omega \}$ the following way by induction:
  \begin{itemize}
    \item $P_0 = - A_0$,
    \item $P_{n + 1} = A_n - \bigcap \limits_{j < n} A_j$
  \end{itemize}
  Then each $P_i \notin D$ and $\{ P_n \: | \: n < \omega \}$ is the partition of $\omega$.
  Thus there is $X \in D$ such that $X \cap P_n$ is finite for each $n$.
\end{enumerate}
\end{proof}

\begin{exercise}
  An ultrafilter on $\omega$ is Ramsey iff every function $f : \omega \to \omega$ is either 
  one-to-one on a set in $D$ or $f$ is constant on a set in $D$.
\end{exercise}

\begin{proof}
  TODO.
\end{proof}

Let $D, E$ be ultrafilters on $\omega$, then the \emph{Rudin-Keisler ordering} is defined as follows:
\begin{center}
  $D \leq E$ iff $\exists f : \omega \to \omega \:\: D = f_*(E)$.
\end{center}
$D \equiv E$ means there is a one-to-one function $f : \omega \to \omega$ such that $E = f_*(D)$.

\begin{exercise}
  If $D = f_*(D)$, then $\{ n \: | \: f(n) = n \} \in D$.
\end{exercise}

\begin{proof}
  Let $X = \{ n \: | \: f(n) < n \}$ and $Y = \{ n \: | \: f(n) > n\}$. Take $n \in X$, let
  $l(n)$ be the length of the maximal sequence $n > f(n) > f(f(n)) > \dots$.
  Let $X_0 = \{ n \: | \: l(n) \equiv 0 \mod 2\}$ and $X_1 = \{ n \: | \: l(n) \equiv 1 \mod 2\}$.
  Neither $X_0$ nor $X_1$ belong to $D$ since $f^{-1}(X_0) \cap X_0 = \emptyset$. 
  We handle $Y$ similarly, but we have got to show that the set $Z$ of all $n$ such that
  the sequence $n < f(n) < f(f(n)) < f(f(f(n))) < \dots$ is infinite does not belong to $D$.
  Take $x, y \in Z$, we say that $x \equiv y$ if $\exists k, m < \omega \:\: f^{k}(x) = f^m(y)$.
  Take $x \in Z$ and let $a_x$ be a fixed representative of the class ${y \: | \: y \equiv x }$.
  Let $l(x)$ be the least $k + m$ such that $f^k(x) = f^m(a_x)$.
  Take $Z_0 = \{ x \in Z \: | \: l(x) \equiv 0 \mod 2 \}$ and $Z_1 = \{ x \in Z \: | \: l(x) \equiv 1 \mod 2 \}$.
  Thus $Z_0 \cap f^{-1}(Z_0) = \emptyset$.
\end{proof}

\begin{exercise}
  $D \leq E$ and $E \leq D$ implies $D \equiv E$
\end{exercise}

\begin{proof} Let $D \leq E$ and $E \leq D$, then
  there are $f, g : \omega \to \omega$ such that $D = f_*(E)$ and $E = g_*(D)$. We have
  $D = (f \circ g)_*(D)$ since
  \begin{center}
    $(f \circ g)_*(D) = \{ X \subset \omega \: | \: (f \circ g)^{-1}(X) \in D \} = \{ X \subset \omega \: | \: X \in D \}$.
  \end{center}
  Thus by the above lemma, the set $X_0 = \{ n \: | \: f(g(n)) = n \}$ is in $D$ and thus 
  $f \circ g\upharpoonright_{X_0}$ is the identity function. Thus $g\upharpoonright_{X_0}$ and 
  $f\upharpoonright_{g[X_0]}$ are one-to-one, so $f$ and $g$ are both bijections on $X_0$ and $g[X_0]$.
  Let us find $X_0' \subset X_0 \in D$ such that $g\upharpoonright_{X_0}'$ 
  can be extended to a bijection $\overline{g} : \omega \to \omega$.

  If $|\omega - X_0| = |\omega - g[X_0]|$, so take any bijection $h : \omega - X_0 \to \omega - g[X_0]$. In this case
  we let $X_0' = X_0$ and $\overline{g}$ as $g \cup f$.
  Otherwise $X_0$ must be infinite and one split $X_0$ into two disjoint parts $X'_0$ and $X^{''}_0$ where 
  $X'_0 \in D$. And then we take a bijection $h : \omega - X'_0 \to \omega - g[X'_0]$ and $\overline{g} = g \cup h$.

  In either case $\overline{g}$ is a bijection, we have $X'_0 \in D$, $E = f_*(D)$ and 
  $g\upharpoonright_{X'_0} = \overline{g}\upharpoonright_{X'_0}$, 
  so $\overline{g}[X'_0] \in E$. The rest is to show that $\overline{g}_*(E) = D$. Since
  $g_*(E) = D$, we get
  \begin{center}
    $\{ g[X] \: | \: X \in D \} \subset E$,

    $\{ g^{-1}[Y] \: | \: Y \in E \} \subset D$.
  \end{center}

  Take any $Y \in E$ and let $Y' = Y \cap \overline{g}[X'_0]$ and let $X' = \overline{g}^{-1}[Y']$, 
  then $Y' \in E$, $X' \in D$ and $g'[X'] \subset Y$ and thus $\overline{g}_*(E) = D$.
\end{proof}


\begin{exercise}
  The set of all $X \subset \mathbb{R}$ having Lebesgue measure $0$ is a $\sigma$-ideal.
\end{exercise}

\begin{proof}
  Let $I = \{ X \subset \mathbb{R} \: | \: \mu(I) = 0 \}$.
  Then $\emptyset \in I$ since $\mu(\emptyset) = 0$.
  Let $A \subset B$ and $\mu(B) = 0$, but $\mu(A) \leq \mu(B)$ and $\mu(A) \geq 0$.
  The Lebesgue measure is $\sigma$-additive, so $I$ is closed under countable unions.
\end{proof}

Recall that a set $A \subset \mathbb{R}$ is \emph{meagre} if $A$ is the union of countable of nowhere dense sets.

\begin{exercise}
  The set of all $A \subset \mathbb{R}$ meagre sets is a $\sigma$-ideal.
\end{exercise}

\begin{proof}
  We let 
  \begin{center}
  $I = \{ A \subset \mathbb{R} \: | \: \text{$A$ is meagre} \}$
  \end{center}

  By Theorem~\ref{baire:category}, $\mathbb{R} \notin I$. We have 
  $\emptyset = \operatorname{Int}(\operatorname{Cl}\emptyset)$, so $\emptyset$ is nowhere dense.

  Now let $A \subset B$ and assume $B \in I$, thus there is a family $\{ B_n \: | \: n < \omega \}$ such that
  each $B_n \subset \mathbb{R}$ is nowhere dense in $\mathbb{R}$ and $B = \cup_{n < \omega} B_n$.
  Thus
  \begin{center} 
  $A \subset \bigcup \limits_{n < \omega} B_n \leftrightarrow A \cap \bigcup \limits_{n < \omega} B_n = \bigcup \limits_{n < \omega} A \cap B_n$
  \end{center}

  Note that each $A \cap B_n$ is nowhere dense for each $n < \omega$. Thus $A \in I$.

  Let $\{ A_n \: | \: n < \omega \}$ be a collection of elements of $I$. For each $k < \omega$
  $A_k$ is the countable union of some nowhere dense sets, that is
  \begin{center}
    $A_k = \bigcup \limits_{i < \omega} {A_i}_{k}$
  \end{center}
  where each ${A_i}_k$ is countable.
  Consider the family $\{ {A_i}_k \: | \: i, k < \omega \}$. Such a family is obviously countable, so
  $\cup_{i, k} {A_i}_k$ is meagre and thus belongs to $I$.
\end{proof}

\begin{exercise}~\label{sikorski:lemma}
  Let $A$ be a subalgebra of a Boolean algebra $B$. Let $u \in B$ and let $A(u)$ be the algebra generated by 
  $A \cup \{ u \}$. Let $C$ be a complete Boolean algebra. If $h : A \to C$ is a homomorphism, then
  $h$ extends to a homomorphism from $A(u)$ into $C$.
\end{exercise}

\begin{proof}
  We let
  \begin{center}
    $r = \sum \{ g(x) \: | \: x \in A \land x \leq u \}$

    $t = \prod \{ g(y) \: | \: y \in A \land u \leq y \}$
  \end{center}
  Both $r$ and $t$ do exist since $C$ is complete. Clearly $r \leq t$. Take $r \leq v \leq t$.

  Every element $c$ of $A(u)$ has the form $a \cdot u + b \cdot (- u)$. Define
  $h' : A(u) \to C$ as
  \begin{center}
    $h'(c) = h(a \cdot u + b \cdot (- u)) = h(a) \cdot t + h(b) \cdot - t$.
  \end{center}
  Then $h'$ is a Boolean homomorphism.
\end{proof}

\begin{exercise}[Sikorski's Extension Theorem]
  Let $A$ be a subalgebra of a Boolean algebra $B$ and let $h$ be a homomorphism into a complete Boolean algebra
  $C$. Then $h$ can be extended to a homomorphism from $B \to C$.
\end{exercise}

\begin{proof}
  Let $M$ be the set of all pairs $(h_D, D)$ such that $D$ is a subalgebra of $B$ containing $A$
  and $h_D : D \to C$ is a homomorphism extending $h$. $M$ is non-empty since $(h, A) \in M$.
  Define the $\leq$ relation on $M$ the following way:
  \begin{center}
    $(h_D, D) \leq (h_{D'}, D') \leftrightarrow D \subset D \land h_D \subset h_{D'}$.
  \end{center} 
  Clearly $(M, \leq)$ is a poset. Let us check $(M, \leq)$ satisfies Zorn's lemma.
  Let $C$ be a chain in $M$, so the following pair
  \begin{center}
    $(\bigcup \{ \pi_1 m \: | \: m \in C \}, \bigcup \{ \pi_2 m \: | \: m \in C \})$.
  \end{center}
  is the upper bound of $C$. So by Zorn's lemma, we have the maximal pair $(g, E)$.
  Let us show that $B = E$.
  Assume $B \neq E$, take $a \in B - E$ and apply Exercise~\ref{sikorski:lemma} to $A$, $B$ and $a$.
  Then we have a pair $(h', A(u)) > (g, E)$, but this is a contradiction.
\end{proof}

\begin{exercise}
  Let $B$ be a Boolean algebra and let $A$ be a regular subalgebra of $B$, then the inclusion mapping
  extends to a (unique) complete embedding of the completion of $A$ into the completion of $B$.
\end{exercise}

\begin{proof}
  TODO
\end{proof}

\section{Stationary Sets}

\subsection{Closed Unbounded Sets}

Let $X$ be a set of ordinals and $\alpha$ be a limit ordinal, then $\alpha$ is a \emph{limit point} of $X$ if
$a = \sup (X \cap \alpha)$.

\begin{definition}
  Let $\kappa$ be a regular uncountable cardinal. 
  A set $C \subset \kappa$ is a \emph{closed unbouned subset} (or a \emph{club}) of $\kappa$ if $C$ is unbounded in $\kappa$
  and it contains all limit points less than $\kappa$.

  A set $S \subset \kappa$ is \emph{stationary} if $S \cap C \neq \emptyset$ for every closed unbounded $C \subset \kappa$.
\end{definition}

An unbounded set $C \subset \kappa$ is closed if and only if for every sequence 
$\alpha_0 < \alpha_1 < \ldots < \alpha_{\gamma} < \ldots$ for where each $\alpha_{\gamma} \in C$ $\gamma < \kappa$
we have $\lim_{\xi \to \gamma} \alpha_{\xi} \in C$.

\begin{lemma}~\label{club:int}
  If $C, D \subset \kappa$ are closed unbounded, so is $C \cap D$.
\end{lemma}

\begin{proof}
  $C \cap D$ is self-evidently closed, let us show it is unbounded.
  Take any $\alpha < \kappa$. $C$ is unbounded, so one can take $\alpha_1 > \alpha$ such that $\alpha \in C$.
  Similarly one can take $\alpha_2 > \alpha_1$ from $D$ and et cetera.

  So we construct a sequence
  \begin{center}
    $\alpha < \alpha_1 < \alpha_2 < \ldots < \alpha_n < \ldots$
  \end{center}
  where $\{ \alpha_1, \alpha_3, \dots \} \subset C$ and $\{ \alpha_2, \alpha_4, \dots \} \subset D$. Let $\beta$
  be the limit of the above sequence. Then $\beta < \kappa$, so $\beta \in C$ and $\beta \in D$.
\end{proof}

The collection of all closed unbounded subsets of $\kappa$ has the finite intersection property.
The filter generated by this collection consists of all $X \subset \kappa$ that contain some closed unbounded set.
Such a filter is called the closed unbounded filter on $\kappa$.

The set of all limit ordinals $\alpha < \kappa$ is closed unbounded in $\kappa$. 
Let $A \subset \kappa$ be an unbounded subset of $\kappa$, then the set of limit points of $A$ is closed in $\kappa$.

Recall that a function $f : \kappa \to \kappa$ is \emph{normal} if $f$ is continuous and increasing.
The range of a normal function is a closed unbounded set. Conversely, if $C$ is closed unbounded
then there exists a unique normal function that enumerates $C$.

\begin{theorem}
The intersection of fewer than $\kappa$ closed unbounded subsets of $\kappa$ is closed unbounded.
\end{theorem}

\begin{proof}
  Take $\gamma < \kappa$, let us show that the intersection of a sequence 
  $\langle C_{\alpha} \: : \: \alpha < \gamma \rangle$ of clubs is a club.
  The induction step for successor ordinals was actually proved in Lemma~\ref{club:int}.

  Let $\gamma$ be limit and assume the statement holds for each $\alpha < \gamma$.
  Then we can replace each $C_{\alpha}$ with $\cap_{\xi < \alpha} C_{\xi}$ and obtain a decreasing sequence with the same
  intersections. Assume that for $\alpha < \gamma$
  \begin{center}
    $C_0 \supset C_1 \supset \ldots \supset C_{\alpha} \supset \dots $
  \end{center}
  are closed unbounded and let $C = \cap_{\alpha < \gamma} C_{\alpha}$.
  $\alpha$ is closed, to show unboundedness of $C$, take $\alpha < \kappa$. Let us construct a $\gamma$-sequence
  \begin{center}
    $\beta_0 < \beta_1 < \ldots < \beta_{\xi} < \ldots$
  \end{center}
  the following way. Let $\beta_0 \in C_0$ such that $\beta_0 > \alpha$ and for each 
  $\xi < \gamma$ we let $\beta_{\xi} \in C_{\xi}$ such that $\beta_{\xi} > \sup \{ \beta_{\nu} \: | \: \nu < \xi \}$.

  $\kappa$ is regular and $\gamma < \kappa$, the above sequence does exist and 
  $\lim_{\xi \to \gamma} \beta_{\xi} = \beta < \kappa$. For each $\eta < \kappa$, $\beta$ is the limit 
  of a sequence of a sequence $\langle \beta_{\xi} \: : \: \eta \leq \xi < \gamma \rangle \in C_{\eta}$, so $\beta \in C_{\eta}$.
  Thus $\beta \in C$.
\end{proof} 

Let $\langle X_{\alpha} \: | \: \alpha < \kappa \rangle$ be a sequence of subsets of $\kappa$.
The \emph{diagonal intersection} of $X_{\alpha}$ for $\alpha < \kappa$ is defined as follows
\begin{center}
  $\Delta_{\alpha < \kappa} X_{\alpha} = \{ \xi < \kappa \: | \: \xi \in \bigcap \limits_{\alpha < \xi} X_{\alpha} \}$.
\end{center}

Note that $\Delta_{\alpha < \kappa} X_{\alpha} = \Delta_{\alpha < \kappa} Y_{\alpha}$
where $Y_{\alpha} = \{ \xi \in X_{\alpha} \: | \: \xi > \alpha \}$. Also note that
$\Delta_{\alpha} X_{\alpha} = \cap_{\alpha} (X_{\alpha} \cup \{ \xi \: |\: \xi \leq \alpha \})$.

\begin{lemma}~\label{diag:lemma}
  The diagonal intersection of a $\kappa$-sequence of clubs is a club.
\end{lemma}

\begin{proof}
  Let $\langle C_{\alpha} \: : \: \alpha < \kappa \rangle$ be a sequence of clubs.
  We can replace each $C_{\alpha}$ with $\cap_{\xi < \alpha} C_{\xi}$, the diagonal intersection
  remains the same. By the previous theorem we can assume that, for $\alpha < \kappa$
  \begin{center}
    $C_0 \supset C_1 \supset C_2 \supset \ldots \supset C_{\alpha} \supset \dots$.
  \end{center}

  Let $C = \Delta_{\alpha < \kappa} C_{\alpha}$. Let us show $C$ is closed.
  Let $\alpha$ be a limit point of $C$. We need $\alpha \in C$ or $\alpha \in C_{\xi}$ for each $\xi < \alpha$.
  Let $X = \{ \nu \: | \: \xi < \nu < \alpha \}$. Every $\nu \in X$ is in $C_{\xi}$ by the previous theorem.
  Thus $X \subset C_{\xi}$ and $\alpha = \sup X \in C_{\xi}$, so $\alpha \in C$.

  Let us show that $C$ is unbounded. Let $\alpha < \kappa$. Construct a sequence 
  $\langle \beta_n \: : \: n < \omega \rangle$ the following way. Let $\beta_0 > \alpha$
  such that $\beta_0 \in C_0$ and for each $n$, let $\beta_{n + 1} > \beta_n$
  such that $\beta_{n + 1} \in C_{\beta_n}$. Let us that that $\beta = \lim_{n} \beta_n$ is in $C$.
  If $\xi < \beta$, let us show that $\beta \in C_{\xi}$. Since $\xi < \beta$, there is $n < \omega$
  such that $\xi < \beta_n$. Each $\beta_k$ for $k > n$ belongs to $C_{\beta_n}$, so $\beta \in C_{\beta_n}$.
  Thus $\beta \in C_{\xi}$ and thus $\beta \in C$.
\end{proof}

\begin{col}
  The closed unbounded filter on $\kappa$ is closed under diagonal intersections.
\end{col}

The dual notion of a closed unbounded filter is the ideal of nonstationary sets, the \emph{nonstationary ideal} $I_{NS}$.
$I_{NS}$ is $\kappa$-complete and is closed under diagonal unions:
\begin{center}
  $\sum \limits_{\alpha < \kappa} X_{\alpha} = \{ \xi < \kappa \: : \: \xi \in \bigcup \limits_{\alpha < \xi} X_{\alpha} \}$
\end{center}

The quotient algebra $B = P(\kappa)/_{I_{NS}}$ is a $\kappa$-complete where infinitary operations are induced
by infinitary intersection and union (up to $\kappa$, of course). By Lemma~\ref{diag:lemma}, $B$ is also $\kappa^+$-complete.
Indeed, let $\{ X_{\alpha} \: | \: \alpha < \kappa \}$ be a collection of subsets of $\kappa$, then the equivalence
classes of $\Delta_{\alpha} X_{\alpha}$ and $\Sigma_{\alpha} X_{\alpha}$ are the greatest lower bound and the 
least upper bound of the equivalence classes $[X_{\alpha}]$ in $B$. Also note that if $\langle X_{\alpha} \: : 
\: \alpha < \kappa \rangle$ and $\langle Y_{\alpha} \: : \: \alpha < \kappa \rangle$ are two enumerations of the same collection,
then $\Delta_{\alpha} X_{\alpha}$ and $\Delta_{\alpha} Y_{\alpha}$ differ only by a nonstationary set.

\begin{definition}
  An ordinal function $f$ on a set $S$ is \emph{regressive} if $f(\alpha) < \alpha$ for all non-zero $\alpha \in S$.
\end{definition}

\begin{theorem}[\bf Fodor]~\label{fodor}

  Let $f$ be a regressive function on a stationary set $S \subset \kappa$, then there is a stationary set
  $T \subset S$ and some $\gamma < \kappa$ such that $f(\alpha) = \gamma$ for all $\alpha \in T$.
\end{theorem}

\begin{proof}
  Assume that for each $\gamma < \kappa$ the set $\{ \alpha \in S \: | \: f(\alpha) = \gamma \}$
  is nonstationary and choose a closed unbounded set $C_{\gamma}$ such that $f(\alpha \neq \gamma)$
  for each $\alpha \in S \cap C_{\gamma}$. Let $C = \Delta_{\gamma < \kappa} C_{\gamma}$.
  The set $S \cap C$ is stationary and if $\alpha \in S \cap C$, so we have $f(\alpha \neq \gamma)$
  for $\gamma < \alpha$, or $f(\alpha) \geq \alpha$, but $f$ is regressive. Contradiction.
\end{proof}

Let $\kappa$ be a regular uncountable cardinal and let $\lambda < \kappa$ be regular, let
\begin{center}
  $E^{\kappa}_{\lambda} = \{ \alpha < \kappa \: | \: \operatorname{cf} \alpha = \lambda \}$
\end{center}
One can check that $E^{\kappa}_{\lambda}$ is a stationary subset of $\kappa$.

Note that the closed unbounded filter on $\kappa$ is not an ultrafilter. 
This is because there is a stationary subset of $\kappa$ with a stationary complement. If $\kappa > \omega_1$,
this is clear since $E_{\omega}^{\kappa}$ and $E_{\omega_1}^{\kappa}$ are disjoint. If $\kappa = \omega_1$, then
the decomposition of $\omega_1$ into disjoint stationary sets requires the Axiom of Choice.

\begin{lemma}~\label{stationary:lemma}
  Every stationary subset of $E_{\omega}^{\kappa}$ is the union of $\kappa$ disjoint stationary sets.
\end{lemma}

\begin{proof}
  Let $W = \{ \alpha < \kappa \: | \: \operatorname{cf} \alpha = \omega \}$ be stationary. Take $\alpha \in W$ and
  choose an increasing sequence $\langle a^{\alpha}_n \: : \: n < \omega \rangle$ such that 
  $\lim_n a^{\alpha}_n = \alpha$. Let us first show that there is $n < \omega$ such that for all $\eta < \kappa$ the set
  \begin{center}
    $\{ \alpha \in W \: | \: a^{\alpha}_n \geq \eta \}$
  \end{center}
  is stationary. Otherwise there is $\eta_n$ and a club $C_n$ such that $a^{\alpha}_n < \eta_n$
  for each $\alpha \in C_n \cap W$ for every $n < \omega$. 
  If we let $\eta$ be the supremum of the $\eta_n$ and $C$ the intersection of $C_n$'s,
  we have $a_n^{\alpha} < \eta$ for all $n$ and all $\alpha \in C \cap W$. Contradiction.
  Now let $n$ be such that $\{ \alpha \in W \: | \: a^{\alpha}_n \geq \eta \}$ is stationary 
  for every $\eta < \kappa$. Define the following function $f$ on $W$
  \begin{center}
    $f(\alpha) = a^{\alpha}_n$
  \end{center}
  $f$ is regressive, so by Theorem~\ref{fodor}, there is a stationary set $S_{\eta} \subset \{ \alpha \in W \: | \: a^{\alpha}_n \geq \eta \}$
  and $\gamma_{\eta} \geq \eta$ such that $f(\alpha) = \gamma_{\eta}$ on $S_{\eta}$.
  If $\gamma_{\eta} \neg \gamma_{\eta'}$, then $S_{\eta} \cap S_{\eta'} = \emptyset$. But $\kappa$ is regular, thus
  \begin{center}
    $\kappa = |\{ S_{\eta} \: | \: \eta < \kappa\}| = |\{ \gamma_{\eta} \: | \: \eta < \kappa \}|$.
  \end{center}
\end{proof}

\begin{lemma}
  Let $S$ be a stationary set of a cardinal $\kappa$ such that evert $\alpha \in S$ is a regular uncountable
  cardinal. Then the set
  \begin{center}
  $T = \{ \alpha \in S \: | \: \text{$S \cap \alpha$ is not stationary subset of $\alpha$} \}$
  \end{center}
  is stationary.
\end{lemma}

\begin{proof}
  We need to show that $T$ intersects every closed unbounded subset of $\kappa$. 
  Take any unbounded closed $C \subset \kappa$. The set $C'$ of all limit points of $C$ is also unbounded, 
  so $S \cap C'$ is non-empty. Let $\alpha$ be the least element of $S \cap C'$.
  $\alpha$ is regular and a limit point of $C$, so $C \cap \alpha$ is closed unbounded subset of $\alpha$ and so
  is $C' \cap \alpha$. $\alpha$ is the least element of $S \cap C'$, so $C' \cap \alpha$ is disjoint
  from $S \cap \alpha$ and thus $S \cap \alpha$ is a non-stationary subset of $\alpha$. Thus $\alpha \in T \cap C$.
\end{proof}

\begin{theorem}[Solovay]

  Let $\kappa$ be a regular uncountable cardinal, then every stationary subset of $\kappa$ is the disjoint union
  of $\kappa$ stationary subsets.
\end{theorem}

\begin{proof}
  The proof is fairly similar to the proof of Lemma~\ref{stationary:lemma}.

  Let $A$ be a stationary subset of $\kappa$. We can assume that the set
  \begin{center}
    $W = \{ \alpha \in A \: | \: \text{$\alpha$ is a regular cardinal and $A \cap \alpha$ is nonstationary}\}$
  \end{center}
  is stationary. For each $\alpha \in W$ there exists a continuous increasing sequence 
  $\langle a^{\alpha}_{\xi} \: : \: \xi < \alpha \rangle$ such that $a^{\alpha}_{\xi} \notin W$
  for all $\alpha$ and $\xi$ and $\alpha = \lim_{\xi \to \alpha} a^{\alpha}_{\xi}$.

  First of all, let us show that there is $\xi$ such that for all $\eta < \kappa$, the set
  $\{ \alpha \in W \: | \: a^{\alpha}_{\xi} \geq \eta \}$ is stationary. Otherwise, there is
  for each $\xi$ some $\eta(\xi)$ and a club $C_{\xi}$ such that $a^{\alpha}_{\xi} < \eta(\xi)$
  for all $\alpha \in C_{\xi} \cap W$ whenever $a^{\alpha}_{\xi}$ is defined.

  Let $C$ be the diagonal intersection of the $C_{\xi}$. Then if $\alpha \in C \cap W$,
  then $a^{\alpha}_{\xi} < \eta(\xi)$ for all $\xi < \alpha$. Now let $D$ be the closed unbounded
  set of all $\gamma \in C$ such that $\eta(\xi) < \gamma$ for all $\xi < \gamma$. $W$ is stationary as well as
  $W \cap D$. Let $\gamma < \alpha$ be two ordinals in $W \cap D$. If $\xi < \gamma$, then 
  $a^{\alpha}_{\xi} < \eta(\xi) < \gamma$, so $a^{\alpha}_{\gamma}$. This is a contradiction since
  $\gamma \in W$, but $a^{\alpha}_{\gamma} \notin W$.

  We have found $\xi$ such that the set $\{ \alpha \in W \: | \: a^{\alpha}_{\xi} \geq \eta \}$ is stationary
  for all $\eta < \kappa$. We proceed as in Lemma~\ref{stationary:lemma}. Let $f$ be the function on $W$ defined by
  $f(\alpha) = a^{\alpha}_{\xi}$. $f$ is regressive, so for every $\eta < \kappa$ we find a stationary subset $S_{\eta}$
  of $\{ \alpha \in W \: | \: a^{\alpha}_{\xi} \geq \eta \}$ by Fodor's theorem and $\gamma_{\eta} \geq \eta$
  such that $f(\alpha) = \gamma_{\eta}$ on $S_{\eta}$. Assume $\gamma_{\eta} \neq \gamma_{\eta'}$,
  so $S_{\eta} \cap S_{\eta'} = \emptyset$. $\kappa$ is regular, so 
  \begin{center}
  $|\{ S_{\eta} \: | \: \eta < \kappa\}| = \{ \gamma_{\eta} \: | \: \eta < \kappa \} = \kappa$.
  \end{center}
\end{proof}

\subsection{Mahlo Cardinals}

Let $\kappa$ be an inaccessible cardinals. The set of all cardinals below $\kappa$ is a closed unbounded subset of
$\kappa$, so is the set of its limit points. The set of all strong limit cardinals below $\kappa$ is closed unbounded
as well.

If $\kappa$ is the least unaccessuble cardinal, then all strong limit cardinals below $\kappa$ are singular and
the set of all strong limit cardinals below $\kappa$ is a club. If $\kappa$ is the $\alpha$-th inaccessible 
for $\alpha < \kappa$, then the set of all regular cardinals below $\kappa$ is nonstationary.

An inaccessible cardinal $\kappa$ is a \emph{Mahlo cardinal} if the set of all regular cardinals below $\kappa$
is stationary. A cardinal is \emph{weakly Mahlo} if it is weakly inaccessible and the set of all regular cardinals below $\kappa$
is stationary.

\subsection{Normal Filters}

Let $F$ be a filter on a cardinal $\kappa$. $F$ is \emph{normal} if it is closed under diagonal intersections:
\begin{center}
  $\forall \alpha < \kappa \:\: X_{\alpha} \in F \Rightarrow \Delta_{\alpha < \kappa} X_{\alpha} \in F$.
\end{center}

An ideal $I$ on $\kappa$ is \emph{normal} if the dual filter is normal.
The closed unbounded filter is $\kappa$-complete and normal and contains all complements of bounded sets. 
It is the smallest such filter on $\kappa$.

\begin{lemma}
  Let $\kappa$ be a regular and uncountable cardinal and let $F$ be a normal filter
  on $\kappa$ that contains all final segments $\{ \alpha \: | \: \alpha_0 < \alpha < \kappa \}$, then $F$ contains 
  all closed unbounded sets.
\end{lemma}

\begin{proof}
  First of all the set $C_0$ of all limit ordinals is in $F$: $C_0$ is the diagonal intersection of the sets
  $X_{\alpha} = \{ \xi \: | \: \alpha < \xi < \kappa \}$. Let $C$ be a closed unbounded set and let 
  $C = \{ a_{\alpha} \: | \: \alpha < \kappa \}$ be an increasing enumeration of $C$. We let 
  $X_{\alpha} = \{ \xi \: | \: a_{\alpha} < \xi < \kappa \}$, so $C \supset C_0 \cap \Delta_{\alpha < \kappa} X_{\alpha}$.
\end{proof}

\subsection{Silver's Theorem}

We are going to show the following theorems

\begin{theorem}[\bf Silver]~\label{silver:1}
  Let $\kappa$ be a singular cardinal such that $\operatorname{cf} \kappa > \omega$.
  If for each $\alpha < \kappa$ we have $2^{\alpha} = \alpha^+$, then $2^{\kappa} = \kappa^+$.
\end{theorem}

\begin{theorem}[\bf Silver]~\label{silver:2}
  If the Singular Cardinals Hypothesis holds for all singular cardinals of cofinality $\omega$, then it holds for all singular cardinals.
\end{theorem}

First of all, we need the following lemma:
\begin{lemma}~\label{silver:lemma} Let $\kappa$ be a singular cardinal such that $\operatorname{cf} \kappa > \omega$ and
  for each $\lambda < \kappa$ we have $\lambda^{\operatorname{cf} \kappa} < \kappa$.
  Let $\langle \kappa_{\alpha} \: : \: \alpha < \operatorname{cf} \kappa \rangle$ be a normal sequence
  of cardinals such that $\lim \kappa_{\alpha} = \kappa$. If the set 
  $\{ \alpha < \operatorname{cf} \kappa \: | \: \kappa^{\operatorname{cf} \kappa_{\alpha}}_{\alpha} = \kappa^+_{\alpha}\}$
  is stationary in $\operatorname{cf} \kappa$, then $\kappa^{\operatorname{cf} \kappa} = \kappa^+$.
\end{lemma}

Note that if GCH holds below $\kappa$, then the assumptions of Lemma~\ref{silver:lemma} are satisfied
and $2^{\kappa} = \kappa^{\operatorname{cf} \kappa}$. So Theorem~\ref{silver:1} from Lemma~\ref{silver:lemma}.

\begin{proof}[Proof of Theorem~\ref{silver:2}]
  Let us show by induction on the cofinality of $\kappa$ that $2^{\operatorname{cf} \kappa} < \kappa$ implies
  $\kappa^{\operatorname{cf} \kappa} = \kappa^+$. By the assumption of the theorem, this implication holds
  for each $\kappa$ of countable cofinality.
  Let $\kappa$ be a cardinal such that $\operatorname{cf} \kappa \geq \omega_1$. Then we apply
  the induction hypothesis and Theorem~\ref{SCH:thm} to verify by induction on $\lambda$
  that $\lambda^{\operatorname{cf} \kappa} < \kappa$ for all $\lambda < \kappa$.

  Let $\langle \kappa_{\alpha} \: | \: \alpha < \operatorname{cf} \kappa \rangle$ be a normal sequence of cardinals
  such that $\kappa = \lim \kappa_{\alpha}$. Then the set
  \begin{center}
    $S = \{ \alpha < \operatorname{cf} \kappa \: | \: \operatorname{cf} \kappa_{\alpha} = \omega \land \kappa_{\alpha} > 2^{\aleph_0}\}$
  \end{center}
  is stationary in $\kappa$. Moreover, for every $\alpha \in S$ 
  $\kappa^{\operatorname{cf} \kappa_{\alpha}}_{\alpha} = \kappa^+_{\alpha}$ by the assumption.
  Thus $\kappa^+ = \kappa^{\operatorname{cf} \kappa}$.
\end{proof}

Now let us show Lemma~\ref{silver:lemma}. We consider the instance of the statement for $\kappa = \aleph_{\omega_1}$.
Let $f$ and $g$ be functions on $\omega_1$. $f$ and $g$ are \emph{almost disjoint} if there is
$\alpha_0 < \omega_1$ such that $f(\alpha) \neq g(\alpha)$ for each $\alpha \geq \alpha_0$.
A family $F$ of functions on $\omega_1$ is an \emph{almost disjoint family} if all functions from $F$ are pairwise
almost disjoint.

Lemma~\ref{silver:lemma} follows from
\begin{lemma}~\label{silver:lemma:2}
  Assume that $\aleph^{\aleph_1}_{\alpha} < \aleph_{\omega_1}$ for all $\alpha < \omega_1$. Let $F$
  be an almost disjoint family of functions
  \begin{center}
    $F \subset \prod \limits_{\alpha < \omega_1} A_{\alpha}$
  \end{center}
  such that the set $\{ \alpha < \omega_1 \: | \: |A_{\alpha}| \leq \aleph_{\alpha + 1}\}$ is stationary.
  Then $|F| < \aleph_{\omega_1 + 1}$.
\end{lemma}
\begin{proof}[Proof of Lemma~\ref{silver:lemma:2} from Lemma~\ref{silver:lemma}.]

  Assume that $\aleph^{\aleph_1}_{\alpha} < \aleph_{\omega_1}$ and 
  $\aleph^{\operatorname{cf} \aleph_{\alpha}}_{\alpha} = \aleph_{\alpha + 1}$ for a stationary set of $\alpha$'s.
  We need $\aleph^{\aleph_1}_{\omega_1} = \aleph_{\omega_1 + 1}$.
  Take $h : \omega_1 \to \aleph_{\omega_1}$, we let $f_h = \langle h_{\alpha} \: : \: \alpha < \omega_1 \rangle$
  where $\operatorname{dom} h_{\alpha} = \omega_1$ and
  \begin{center}
    $h_{\alpha}(\xi) = \begin{cases}
      h(\xi), \:\: \text{if $h(\xi) < \aleph_{\alpha}$}
      0, \:\: \text{otherwise}
    \end{cases}$
  \end{center}
  and let $F = \{ f_h \: | \: h \in \aleph_{\omega_1}^{\omega_1}\}$. If $h \neq g$, then $f_h$ and $f_g$ are almost disjoint.
  Moreover, 
  \begin{center}
  \end{center}
\end{proof}

\bibliographystyle{alpha}
\bibliography{Text}


\end{document}