\documentclass[8pt]{article}
\usepackage{graphicx} % Required for inserting images
\usepackage{amsthm}
\usepackage{amsmath}
\usepackage{amsfonts}
\usepackage{amsopn}
\usepackage{comment}
\usepackage{amssymb}
\usepackage{hyperref}
\usepackage{bussproofs}
\usepackage[all, 2cell]{xy}
\usepackage[all]{xy}
\usepackage{rotating}
\usepackage{lscape}
\usepackage{minted}


\theoremstyle{definition}
\newtheorem{definition}{Definition}[section]

\theoremstyle{definition}
\newtheorem{theorem}{Theorem}[section]

\theoremstyle{definition}
\newtheorem{claim}{Claim}[section]

\theoremstyle{definition}
\newtheorem{ex}{Example}[section] 

\theoremstyle{definition}
\newtheorem{cons}{Construction}[section] 

\theoremstyle{definition}
\newtheorem{rem}{Remark}[section] 


\theoremstyle{definition}
\newtheorem{prop}{Proposition}[section]

\theoremstyle{definition}
\newtheorem{lemma}{Lemma}[section]

\theoremstyle{definition}
\newtheorem{fact}{Fact}[section]

\theoremstyle{definition}
\newtheorem{remark}{Remark}[section]

\theoremstyle{definition}
\newtheorem{notation}{Notation}[section]

\theoremstyle{definition}
\newtheorem{example}{Example}[section]

\theoremstyle{definition}
\newtheorem{exercise}{Exercise}[section]

\theoremstyle{definition}
\newtheorem{col}{Corollary}[section]

\theoremstyle{question}
\newtheorem{question}{Question}

\let\strokeL\L
\renewcommand\L{\mathbf{L}}

\title{Some Notes on Set Theory, Pt 1}
\author{Daniel Rogozin}
\date{ }

\begin{document}

\maketitle

\tableofcontents

\newpage

\section{Cardinals}

An ordinal number $\alpha$ is a \emph{cardinal number} if no $\beta < \alpha$ such that $|\alpha| = |\beta|$.
Further, we shall use $\kappa, \lambda, \mu$ to denote cardinal numbers.

Let $W$ be a well-ordered set, then there exists an ordinal $\alpha$ such that $|W| = |\alpha|$, so we let:
\begin{center}
  $|W| = \text{the least ordinal $\alpha$ such that $|W| = |\alpha|$}$
\end{center}

An \emph{aleph} is an infinite cardinal number. 

Let $\alpha$ be an ordinal, then $\alpha^+$ is the least cardinal bigger than $\alpha$.

\begin{lemma}
  $ $

  \begin{enumerate}
    \item For every $\alpha$ there is a cardinal number $\kappa$ such that $\kappa > \alpha$.
    \item Let $X$ be a set of cardinal, then $\sup X$ is a cardinal.
  \end{enumerate}
\end{lemma}

\begin{proof}
  $ $

  \begin{enumerate}
    \item Let $X$ be a set, let 
    \begin{center}
      $h(X) = \text{the least $\alpha$ such that no injection from $\alpha$ into $X$}$
    \end{center}

    Consider $X \times X$, so $2^{X \times X}$ is the set of relations on $X$ and there 
    are well-orderings of subsets of $X$ amongst all relations in $2^{X \times X}$, so consider the set
    \begin{center}
      $Y = \{ R \subseteq Y \times Y \: | \: Y \subseteq X \: \& \: \text{$Y$ is a well-ordering }\}$
    \end{center}

    So there is a set of ordinals:
    \begin{center}
      $Ord(Y) = \{ \alpha \in \operatorname{Ord} \: | \: \exists R \in Y \: \text{$\alpha$ is the order type of $Y$}  \}$
    \end{center}
    Note that $Ord(Y)$ is a set and take the least element ordinal $\beta$ does not belong to $Ord(Y)$. So $h(X) = \beta$.
    To be more precise, we have:
    \begin{center}
      $\beta = \sup Ord(Y)$
    \end{center}

    Then $|\alpha| < h(\alpha)$ for each ordinal $\alpha$.

    \item Let $\alpha = \sup X$. Let $f$ be a one-to-one function from $\alpha$ onto some $\beta < \alpha$.
    Let $\kappa$ be a cardinal such that $\beta < \kappa \leq \alpha$, then $|\kappa| = |\{ f(\xi) \: | \: \xi < \kappa \}| \leq \beta$, 
    so contradiction and $\alpha$ is a cardinal.
  \end{enumerate}
\end{proof}

The enumeration of all alephs is defined by transfinite induction:
\begin{itemize}
  \item $\aleph_0 = \omega$
  \item $\aleph_{\alpha + 1} =\aleph_{\alpha}^+ = \omega_{\alpha+1}$
  \item If $\beta$ is a limit ordinal, then $\aleph_{\beta} = \omega_{\beta} = \sup \{ \omega_{\alpha} \: | \: \alpha < \beta \}$.
\end{itemize}

A cardinal of the form $\aleph_{\alpha + 1}$ is a \emph{successor} cardinal, 
a cardinal $\aleph_{\beta}$ for limit $\beta$ is a \emph{limit cardinal}.

\subsection{The ordering of $\alpha \times \alpha$}

Define a well-ordering of the class $\operatorname{Ord} \times \operatorname{Ord}$ the following way:

\begin{center}
  $(\alpha, \beta) < (\gamma, \delta)$ iff either $\max(\alpha, \beta) < \max(\gamma, \delta)$ or \\ $\max(\alpha, \beta) = \max(\gamma, \delta)$ and $\alpha < \gamma$ or \\ $\max(\alpha, \beta) = \max(\gamma, \delta)$ and $\alpha = \gamma$ and $\beta < \delta$.
\end{center}

Then $<$ is a well-ordering and linear relation on $\operatorname{Ord}$. 
Moreover, $\alpha \times \alpha$ is the initial segment of $(\operatorname{Ord} \times \operatorname{Ord}, <)$ given by $(0, \alpha)$.

We let:

\begin{center}
  $\Gamma(\alpha, \beta) = \text{the order type of } \{ (\xi, \eta) \: | \: (\xi, \eta) < (\alpha, \beta) \}$
\end{center}

$\Gamma$ is also one-to-one:
\begin{center}
  $(\alpha, \beta) < (\gamma, \delta)$ iff $\Gamma(\alpha, \beta) < \Gamma(\gamma, \delta)$
\end{center}

$\Gamma$ is increasing and continuous and $\Gamma(\alpha \times \alpha) = \alpha$ for arbitrarily large $\alpha$.

\begin{theorem}
  $\aleph_{\alpha} \cdot \aleph_{\alpha} = \aleph_{\alpha}$
\end{theorem}

\begin{proof}
  Let us show that $\Gamma(\omega_{\alpha} \times \omega_{\alpha}) = \omega_{\alpha}$. 
  \begin{enumerate}
    \item If $\alpha = 0$, then $\Gamma(\omega \times \omega) = \omega$.
    \item Let $\alpha$ be the least ordinal such that $\Gamma(\omega_{\alpha} \times \omega_{\alpha}) \neq \omega_{\alpha}$.
    Let $\beta, \gamma$ be ordinals such that $\Gamma(\beta, \gamma) = \omega_{\alpha}$.
    Take $\delta < \omega_{\alpha}$ such that $\delta > \beta, \gamma$. 
    $\delta \times \delta$ is the initial segment of $\operatorname{Ord}^2$ and it contains $(\beta, \gamma)$. 
    So $\Gamma(\delta \times \delta) \supset \omega_{\alpha} = \Gamma(\beta, \gamma)$.
    Thus $|\delta \times \delta| \geq \aleph_{\alpha}$. 
    But $|\delta \times \delta| = |\delta| \cdot |\delta| = |\delta|$. But $|\delta| < \aleph_{\alpha}$ by the assumption 
    of minimality of $\alpha$. Contradiction.
  \end{enumerate}
\end{proof}

As a corollary:
\begin{center}
  $\aleph_{\alpha} + \aleph_{\beta} = \aleph_{\alpha} \cdot \aleph_{\beta} = \max(\aleph_{\alpha}, \aleph_{\beta})$
\end{center}

\subsection{Cofinality}

Let $\alpha, \beta > 0$ be limit ordinals. An increasing $\beta$-sequence 
$\langle \alpha_{\xi} : \xi < \beta \rangle$ is \emph{cofinal} in $\alpha$ if $\lim_{\xi \to \beta} \alpha_{\xi} = \alpha$.
A subset $X \subseteq \alpha$ is \emph{cofinal} in $\alpha$ whenever $\sup X = \alpha$.

Let $\alpha > 0$ be a limit ordinal, the \emph{cofinality} of $\alpha$ is:
\begin{center}
  $\operatorname{cf} \alpha = \text{the least ordinal $\beta$ such that $\exists$ 
  $\langle \alpha_{\xi} : \xi < \beta \rangle$ such that $\lim_{\xi \to \beta} \alpha_{\xi} = \alpha$}$
\end{center}

  Note that for each $\alpha$ $\operatorname{cf} \alpha$ is a limit ordinal and $\operatorname{cf} \alpha \leq \alpha$.

\begin{lemma}
  For each $\alpha$ $\operatorname{cf} (\operatorname{cf} \alpha) \leq \operatorname{cf} \alpha$.
\end{lemma}

\begin{proof}
  Let $\langle \alpha_{\xi} : \xi < \beta \rangle$ be cofinal in $\alpha$ and let $\langle \xi_{\nu} : \nu < \gamma \rangle$ be cofinal in $\beta$.

  Consider $\langle \alpha_{\xi_{\nu}} : \nu < \gamma \rangle$, then
  \begin{center}
    $\lim \limits_{\nu < \gamma} \alpha_{\xi_{\nu}} = \alpha$
  \end{center}
  since the limit of a subsequence equals the limit of a sequence as in usual real analysis or topology.
\end{proof}

\begin{lemma}~\label{confin1}
  Let $\alpha$ be a non-zero limit ordinal, then

  \begin{enumerate}
    \item~\label{confin1.1} If $A \subseteq \alpha$ and $\sup A = \alpha$, the order-type of $A$ is at least $\operatorname{cf} \alpha$.
    \item Let $\beta_0 \leq \beta_1 \leq \dots \leq \beta_{\xi} \leq \dots $ for $\xi < \gamma$ 
    be a non-decreasing sequence of ordinals such that $\lim_{\xi \to \gamma} = \alpha$, then $\operatorname{cf} \gamma = \alpha$.
  \end{enumerate}
\end{lemma}

\begin{proof}
$ $

  \begin{enumerate}
    \item The order-type of $A$ is the length of the increasing enumeration of $A$, 
    the limit of which (as an increasing sequence) is $\alpha$.
    \item If $\gamma = \lim_{\nu \to \operatorname{cf} \gamma} \xi_{\nu}$,
    then $\alpha = \lim_{\nu \to \operatorname{cf} \gamma} \beta_{\xi_{\nu}}$,
    and the non-decreasing sequence $\langle \beta_{\xi_{\nu}} : \nu < \operatorname{cf} \gamma\rangle$
    has an increasing sequence of the length at most $\operatorname{cf} \gamma$ and it has the same limit, 
    so $\operatorname{cf} \alpha \leq \operatorname{cf} \gamma$.

    To show $\operatorname{cf} \gamma \leq \operatorname{cf} \alpha$, 
    assume $\alpha = \lim_{\nu \to \operatorname{cf} \alpha} \alpha_{\nu}$.
    Take $\nu < \operatorname{cf} \alpha$, 
    let $\xi_{\nu}$ be the least $\xi$ greater than all $\xi_{\iota}$ for $\iota < \nu$
    such that $\beta_{\xi} > \alpha_{\nu}$.
    We have $\alpha = \lim_{\nu \to \operatorname{cf} \alpha} \beta_{\xi_{\nu}}$, so
    $\gamma = \lim_{\nu \to \operatorname{cf} \alpha} \xi_{\nu}$, so the inequation is proved.
  \end{enumerate}
\end{proof}

An infinite cardinal $\aleph_{\alpha}$ is \emph{regular} if $\operatorname{cf} \omega_{\alpha} = \omega_{\alpha}$.
$\aleph_{\alpha}$ is \emph{singular} if $\operatorname{cf} \omega_{\alpha} < \omega_{\alpha}$.

\begin{lemma}
  Let $\alpha$ be a limit ordinal, then $\operatorname{cf} \alpha$ is a regular cardinal.
\end{lemma}

\begin{proof}
  If $\alpha$ is not a cardinal, then there exists an ordinal $\beta < \alpha$ such that 
  $|\beta| = |\alpha|$, 
  then we construct a cofinal sequence in $\alpha$ of length $|\beta|$, then $\operatorname{cf} \alpha = |\beta|$ 
  and $\operatorname{cf} \alpha < \alpha$.
\end{proof}

Let $\kappa$ be a limit ordinal, a subset $X \subset \kappa$ is \emph{bounded} if $\sup X < \kappa$ and 
\emph{unbounded} if $\sup X = \kappa$.

\begin{lemma}~\label{boundedaleph} Let $\kappa$ be an aleph, then:

  \begin{enumerate}
    \item If $X \subset \kappa$ and $|X| < \operatorname{cf} \kappa$, then $X$ is bounded.
    \item If $\lambda$ < $\operatorname{cf} \kappa$ and $f : \lambda \to \kappa$, 
    then $\operatorname{Im}f$ is bounded in $\kappa$.
  \end{enumerate}
\end{lemma}

\begin{proof}

  \begin{enumerate}
    \item Let $X$ be such subset of $\kappa$ and assume $X$ is unbounded, so $\sup X = \kappa$.
    By~\ref{confin1.1} of Lemma~\ref{confin1}, the order-type of $X$ is at least $\operatorname{cf} \kappa$, which contradicts to 
    $|X| < \operatorname{cf} \kappa$, so $X$ is bounded.
    \item  Follows from the first item.
  \end{enumerate}
\end{proof}

\begin{lemma}~\label{reg} {\bf (Hausdorff)}

  Let $\kappa$ be a cardinal, then the following are equivalent:

  \begin{enumerate}
    \item $\kappa$ is singular.
    \item There is a cardinal $\lambda < \kappa$ and a family $\{ S_{\xi} \: | \: \xi < \lambda \}$ such that
    each $S_{\xi} \subset \kappa$, $|S_{\xi}| < \kappa$ and $\kappa = \bigcup \limits_{\xi < \lambda} S_{\xi}$.
  \end{enumerate}

\end{lemma}

\begin{proof}
  $ $

  \begin{enumerate}
    \item (1) $\Rightarrow$ (2).

    If $\kappa$ is singular, then there is an increasing sequence 
    $\langle \alpha_{\xi} \: : \: \xi < \operatorname{cf} \kappa \rangle$, so a family of required subsets is
    actually a family of those $\alpha_{\xi}$'s and $\lambda = \operatorname{cf} \kappa$ which is strictly less than $\kappa$
    since $\kappa$ is singular.
    \item (2) $\Rightarrow$ (1).
    
    Let $\lambda$ be the least cardinal such that $\lambda < \kappa$ and there exists a family
    $\{ S_{\xi} \: | \: \xi < \lambda \}$ where each $S_{\xi} \subset \kappa$, $|S_{\xi}| < \kappa$ and
    \begin{center}
      $\kappa = \bigcup \limits_{\xi < \lambda} S_{\xi}$
    \end{center}

    For each $\xi < \lambda$, let $\beta_{\xi}$ be the order-type of 
    $\cup_{\nu < \xi} S_{\nu}$. The sequence $\langle \beta_{\xi} : \xi < \lambda \rangle$ is non-decreasing and 
    each $\beta_{\xi} < \kappa$ for all $\xi < \lambda$ since $\lambda$ is minimal.

    Let us show that $\lim \limits_{\xi \to \kappa} \beta_{\xi} = \kappa$ to show that 
    $\operatorname{cf} \kappa \leq \lambda$.

    Assume $\beta = \lim \limits_{\xi \to \kappa} \beta_{\xi}$. There is a one-to-one mapping 
    $f : \bigcup \limits_{\xi < \beta} S_{\xi} \to \lambda \times \beta$ such that:
    \begin{center}
      $f : \alpha \mapsto (\xi, \gamma)$
    \end{center}
    where $\xi$ is the least ordinal such that $\alpha \in S_{\xi}$ and $\gamma$ is the order-type of $S_{\xi} \cap \gamma$.

    We have $\lambda < \kappa$ and $|\lambda \times \beta| = \lambda \cdot |\beta|$, then $\kappa = \beta$.
  \end{enumerate}
\end{proof}

\begin{theorem}~\label{expcofin}
  Let $\kappa$ be an infinite cardinal, then $\kappa < \kappa^{\operatorname{cf} \kappa}$.
\end{theorem}

\begin{proof}

  Let $F$ be a collection of $\kappa$ functions from $\operatorname{cf} \kappa$ to $\kappa$:
  \begin{center}
    $F = \{ f_{\alpha} : \operatorname{cf} \kappa \to \kappa \: | \: \alpha < \kappa \}$
  \end{center}

  Let us construct $f$ that does not belong to $F$.

  We have $\kappa = \lim_{\xi < \operatorname{cf} \kappa} \alpha_{\xi}$, for $\xi < \operatorname{cf} \kappa$ we let:
  \begin{center}
    $f(\xi) = \text{least $\gamma$ such that $\gamma \neq \forall \alpha < \alpha_{\xi} \: f_{\alpha} \neq \gamma$}$
  \end{center}

  Such $\gamma$ does exist and $f$ is different from all the $f_{\alpha}$.
\end{proof}

An uncountable cardinal $\kappa$ is \emph{weakly inaccessible} if it is limit and regular, but we cannot prove the
existence of weakly inaccessible cardinals in ZFC.

\section{Real Numbers and The Baire Space}

The \emph{continuum} is the cardinality of $\mathbb{R}$ denoted as $\mathfrak{c}$.

\begin{theorem} {\bf (Cantor)}

  $\aleph_0 < \mathfrak{c}$.
\end{theorem}

\begin{proof}
  One can think of it as a consequence of Theorem~\ref{expcofin}.
\end{proof}

\begin{definition} The \emph{Continuum Hypothesis} (CH) is the following statement:

  \begin{center}
    $\aleph_1 = \mathfrak{c}$.
  \end{center}
\end{definition}

Let $(P, <)$ be an ordered set, a subset $D \subset P$ is a \emph{dense} 
subset of $P$ if $a < b$ in $P$ implies $a < d$ and $d < b$ for some $d \in D$.

\begin{theorem}
  $(\mathbb{R}, <)$ is the unique complete linear ordering that has a 
  countable dense subset isomorphic to $(\mathbb{Q}, <)$.
\end{theorem}

\begin{proof}

  Let $C$ and $C'$ be two complete dense linear orderings and let 
  $P$ and $P'$ be dense in $C$ and $C'$ respectively. 
  Let $f : P \cong P'$, so $f$ can be extended to $f^* : C \cong C'$ by letting:

  \begin{center}
    $f^* : x \mapsto \sup \{ f(p) \: | \: p \in P \: \& \: p \leq x \}$
  \end{center}

  That is, ${(.)}^*$ is functorial.
\end{proof}

The existence of $(\mathbb{R}, <)$ follows from the following general statement:
\begin{theorem}
  Let $(P, <)$ be a dense unbounded linear ordering, then there exists a complete dense unbounded linear ordering 
  $(C, \prec)$ such that:

  \begin{enumerate}
    \item $(P, <)$ embeds to $(C, \prec)$.
    \item $P$ is dense in $C$.
  \end{enumerate}
\end{theorem}

\begin{proof}
  Recall that a \emph{Dedekind cut} in $P$ is a pair $(A, B)$ of disjoint subsets of $P$ such that:
  \begin{enumerate}
    \item $A \cup B = P$.
    \item $\forall a \in A \: \forall b \in B \: a < b$.
    \item $A$ has no greatest element.
  \end{enumerate}

  Let $C$ be the set of all Dedekind cuts in $P$. We let $(A_1, B_1) \preceq (A_2, B_2)$
  if $A_1 \subset A_2$ and $B_2 \subset B_1$. $(C, \preceq)$ is complete.

  Let $\{ C_i \: | \: i \in I \} \neq \emptyset$ be a bounded subset of $C$, then 
  $(\bigcup \limits_i A_i, \bigcap \limits_i B_i)$ is its supremum.

  Let $p \in P$, let
  \begin{center}
    $A_p = \{ x \in P \: | \: x < p \}$

    $B_p = \{ x \in P \: | \: x \geq p \}$
  \end{center}

  Then $(\{ (A_p, B_p) \: | \: p \in P \}, \preceq) \cong (P, <)$ and is dense in $C$.
\end{proof}

$\mathbb{Q}$ is dense in $\mathbb{R}$, so every open interval $(a, b)$ contains some rational number.
Then if $S$ is a disjoint collection of open intervals, then $S$ is at most countable.

Let $P$ be a dense linearly ordered set, if every disjoint collection of open intervals is at most countable, then we say
that $P$ satisfies the \emph{countable chain condition}.

\emph{{\bf (Suslin's Problem)} Let $P$ be a dense linearly ordered set satisfying the countable chain condition. Is $P$ isomorphic to $(\mathbb{R}, <)$?}

Note that neither Suslin's Problem nor its negation can be decided in ZFC.

\subsection{Topology of $\mathbb{R}$}

The real line is equipped with the natural topology induced by the metric $d(a, b) = |b - a|$ coincides with
the order topology on $(\mathbb{R}, <)$. $\mathbb{R}$ is also a complete separable metric space.

Every open set in $\mathbb{R}$ is the union of intervals with rational endpoints, so there are continuum many open sets 
(and the same observation holds for open sets as well).

A subset $P$ is \emph{perfect} is it has no isolated points.

\begin{theorem}
  Every perfect set $P$ has cardinality $\mathfrak{c}$.
\end{theorem}

\begin{proof}
  We construct a one-to-one function $F$ from $\{ 0, 1\}^{\omega}$ to $P$. 
  Let $S$ be the set of all finite binary sequences and let $s \in S$.
  
  By induction on $len(s)$ one can find closed intervals $I_s$ 
  such that for each $n < \omega$ and for each $s \in S$ such that $len(s) = n$:
  \begin{enumerate}
    \item $I_s \cap P$ is perfect,
    \item the diameter of $I_s$ is $\leq 1/2$,
    \item $I_{0:s}, I_{1:s} \subset I_s$ and $I_{0:s} \cap I_{1:s} = \emptyset$
  \end{enumerate}

  Take $f \in \{ 0, 1\}^{\omega}$, the set $P \cap \bigcap \limits_{n < \omega} I_{f \upharpoonright n}$ has exactly one element, so let:
  \begin{center}
    $F : f \mapsto \bigcap \limits_{n < \omega} I_{f \upharpoonright n}$
  \end{center}
\end{proof}

\begin{theorem}~\label{cantor-bendixon} ({\bf Cantor-Bendixon})

 If $F$ is an uncountable closed set, then $F = P \cup S$, where $P$ is perfect and $S$ is at most countable.

\end{theorem}

\begin{proof}

\item Let $F \subset \mathbb{R}$, let
\begin{center}
  $F^{'} = \text{the set of all limit points of $F$}$
\end{center}
$F^{'}$ is also called the \emph{derived set} of $F$. $F'$ is closed and obviously a subset of $A$. 

We let:
\begin{enumerate}
  \item $F_0 = A$.
  \item $F_{\alpha+1} = F^{'}_{\alpha}$.
  \item $F_{\alpha} = \bigcap \limits_{\gamma < \alpha} F_{\gamma}$ if $\alpha > 0$ is a limit ordinal.
\end{enumerate}

Since $F_0 \supset F_1 \supset \dots \supset F_{\alpha} \supset$, so we have an ordinal $\theta$ such that 
$F_{\theta} = F_{\theta + 1}$ (otherwise we could map the proper class of ordinals onto some set).
We let $P = F_{\alpha}$. If $P$ is nonempty, then $P$ is also perfect. 

Let us show that $F - P$ is at most countable. 
Let $\langle J_k : k < \omega \rangle$ be an enumeration of rational intervals. We have
\begin{center}
$F - P = \bigcup \limits_{\alpha < \theta} (F_{\alpha} - F_{\alpha + 1})$
\end{center}
So if $a \in F - P$, then there exists $\alpha < \theta$ such that $a \in F_{\alpha} - F_{\alpha + 1}$, that is,
$a$ is an isolated point of $F_{\alpha}$. We let $k_a$ be the least $k$ such that $a$ 
is the only point of $F_{\alpha}$ in $J_k$. 

If $\alpha \leq \beta$ and $a \neq b$ and $b$ is isolated in $F_{\beta}$, then $b \notin J_{k_a}$, so $k_a \neq k_b$, so
the mapping $a \mapsto k_a$ is one-to-one.

\end{proof}

\begin{col}
  If $C \subseteq \mathbb{R}$ is closed, then either $|C| = 2^{\aleph_0}$ or $|C| \leq \aleph_0$.
\end{col}

A set $A \subset \mathbb{R}$ is \emph{nowhere dense} if $\operatorname{Int}\operatorname{Cl} A = \emptyset$.
The following theorem shows that $\mathbb{R}$ is not of the \emph{first category}, that is,
$\mathbb{R}$ is not the union of a countable family of nowhere dense sets.

\begin{theorem} ({\bf The Baire Category Theorem})

  Let $\{ D_i \: | \: i < \omega \}$ be a countable family of dense open subsets of $\mathbb{R}$, then
  $D = \bigcap \limits_{i < \omega} D_i$ is dense in $\mathbb{R}$.
\end{theorem}

\begin{proof}
  We show that $D \cap I \neq \emptyset$ for each open interval $I$.

  Note that each finite intersection $D_0 \cap D_1 \cap \dots \cap D_n$ is dense and open for each $n < \omega$.
  Let $\langle J_k \: : \: k < \omega \rangle$ be an enumeration of rational intervals.
  
  Let $I_0 := I$ and for each $n$ $I_{n + 1} = J_k = (q_k, r_k)$ 
  where $k$ is the smallest index such that $[q_k, r_k] \subset I_n \cap D_n$.

  Take $a = \lim_{k \to \infty} q_k$, then $a \in I \cap D$.
\end{proof}

\subsection{The Baire Space}

The \emph{Baire Space} is the space $\mathcal{N} = \omega^{\omega}$ of infinite sequences of natural numbers
with the topology defined the following way.
Let $s$ be a finite sequence $s = \langle a_k \: : \: k < n \rangle$, we let:
\begin{center}
  $O(s) = \{ f \in \mathcal{N} \: | \: s \subset f \} = \{ \langle c_k \: | \: k < \omega \rangle \: | \: \forall k < n \: c_k = a_k \}$
\end{center}
All those $O(s)$'s form the open basis for $\mathcal{N}$.

The Baire space is separable and metrisable. The metric is defined as $d(f, g) = 1/2^{n + 1}$ 
where $n$ is the smallest natural number such that $f(n) \neq g(n)$. 
We also have separability since the set of all eventually constant sequences is dense in $\mathcal{N}$.

Every infinite sequence $\langle a_k \: : \: k < \omega \rangle$ defines a continued fraction $1/(a_0 + 1/(a_1 + 1/(a_2 + \dots)))$,
so we have a continuous bijection between infinite sequences and irrational points of the open interval $(0, 1)$. 
Moreover, the Baire space is homeomorphic to the space of irrational numbers.

Now we describe the characterisation of perfect sets in the Baire space.

Let $\operatorname{Seq}$ be the set of all finite sequences in $\mathcal{N}$. 
A \emph{tree} is a set $T \subset \operatorname{Seq}$ satisfying:
\begin{center}
  If $t \in T$ and there exists $n < \omega$ such that $s = t \upharpoonright n$, then $s \in T$.
\end{center}

Let $T$ be a tree, let $[T]$ be the set of all infinite paths through $T$:
\begin{center}
  $[T] = \{ f \in \mathbb{N} \: | \: \forall n < \omega \: f \upharpoonright n \in T \}$
\end{center}

For each $T$, the set $[T]$ is closed in the Baire space. Let $f \in \mathcal{N}$
such that $f \notin [T]$. Then there exists $n < \omega$ such that $s = f \upharpoonright n \notin T$, 
so the open neighbourhood of $f$ $O(s) = \{ g \in \mathcal{N} \: | \: g \supset s \}$. Thus $[T]$ is closed.

Conversely, let $F$ be closed in $\mathcal{N}$, then the set
\begin{center}
  $T_F = \{ s \in \operatorname{Seq} \: | \: \exists f \in F \: s \subset f \}$
\end{center}
is a tree and one can verify that $[T_F] = F$. If $f \in \mathcal{N}$ such that $f \upharpoonright n \in T$
for each $n < \omega$, then for each $n$ there is some $g \in F$ such that 
$g \upharpoonright n = f \upharpoonright n$, so $f \in F$ since $F$ is closed.

If $f$ is an isolated point of a closed set $F$ in $\mathcal{N}$, then there is $n \in \mathbb{N}$ such that no
$g \in F$ such that $g \neq f$ and $g \upharpoonright n = f \upharpoonright n$, so we have no branching starting from the
$n$-th position.

So we have the notion of a perfect set $P$ in the Baire space. A tree $T$ is \emph{perfect} if $t \in T$, then there
exist incomparable $t_1, t_2 \supset t$ such that both of them are in $T$ and neither $t_1 \subset t_2$ nor $t_2 \subset t_1$.

\begin{theorem}
  A closed set $F \subset \mathcal{N}$ is perfect iff the tree $T_F$ is perfect.
\end{theorem}

Let us discuss the Cantor-Bendixon analysis of closed subsets of the Baire space. Let $T$ be a tree, define:
\begin{center}
  $T^{'} = \{ t \in T \: | \: \exists t_1, t_2 \supset t \: (t_1, t_2 \in T \: \& \: \neg (t_1 \subset t_2 \lor t_2 \subset t_1))\}$
\end{center}

Then a set $T$ is perfect iff $T = T^{'} \neq \emptyset$.

$[T] - [T^{'}]$ is at most countable: take $f \in [T]$ such that $f \notin [T^{'}]$.
Take $s_f = f \upharpoonright n$ where $n < \omega$ is the smallest index such that $f \upharpoonright n \notin T'$.
If $f, g \in [T] - [T']$, then $s_f \neq s_g$ by the definition of $T'$, so the mapping $f \mapsto s_f$ is one-to-one.

Now let:
\begin{center}
  $T_0 = T$
  
  $T_{\alpha + 1} = T^{'}_{\alpha}$

  $T_{\alpha} = \bigcap \limits_{\beta < \alpha} T_{\beta}$ if $\alpha > 0$ is limit.
\end{center}

We have $T_0 \supset T_1 \supset \dots \supset T_{\alpha} \supset \dots$. $T_0$ is at most countable, so there is
$\theta < \omega_1$ at which the sequence stabilises. If $T_{\theta} \neq \emptyset$, then $T_{\theta}$ is perfect.

One can verify that:
\begin{center}
  $[\bigcap \limits_{\beta < \alpha} T_{\beta}] = \bigcap \limits_{\beta < \alpha} [T_{\beta}]$
\end{center}
so we have
\begin{center}
  $[T] - [T_{\theta}] = \bigcup \limits_{\beta < \alpha} ([T_{\alpha} - T^{'}_{\alpha}])$
\end{center}
and the set $[T] - [T_{\theta}]$ is at most countable. So we have a version of Theorem~\ref{cantor-bendixon} for the Baire space.

\section{The Axiom of Choice}

Recall that the axiom of choice (AC) says that if we have a family of sets $S$ such that $\emptyset \notin S$, then
we have a \emph{choice function} on $S$ such that $f(X) \in X$.

In some cases we can show the existence of a choice function without using the axiom of choice. 
For example, for families of a complete lattice, the choice function can return the supremum or infimum of each set
belonging to a family.

Using the axiom of choice one can also show that every infinite set has cardinality equal to $\aleph_{\alpha}$ for some $\alpha$.

\begin{theorem}{({\bf Zermelo})}

  Every set can be well-ordered.
\end{theorem}

\begin{proof}
  Let $A$ be a set. It is sufficient to construct a transfinite sequence 
  $\langle a_{\alpha} \: : \: \alpha < \theta \rangle$ that enumerates $A$. We do that by induction and by using
  the choice function $f$ on non-empty subsets of $A$. For $\alpha$ we let:
  \begin{center}
    $a_{\alpha} = f (A - \{ a_{\xi} \: | \: \xi < \alpha \})$
  \end{center}
  whenever $A - \{ a_{\xi} \: | \: \xi < \alpha \}$ is non-empty. 
  Let $\theta$ be the smallest ordinal such that $A = \{ a_{\alpha} \: | \: \alpha < \theta \}$.
  Thus $\langle a_{\alpha} \: : \: \alpha < \theta \rangle$ enumerates $A$.
\end{proof}

As it is well-known, Zermelo's theorem implies the axiom of choice.
Let $S$ be a family of sets such that $\emptyset \notin S$.
By Zermelo's theorem, we can well-order $\cup S$, so let $f(X)$ be the smallest element of $X$.

Note that Zermelo's theorem also implies that $\mathbb{R}$ can be well ordered and also that $2^{\aleph_0}$ is an aleph
and $2^{\aleph_0} \geq \aleph_1$.

Another important consequence of the axiom of choice:
\begin{theorem}
  The union of a countable family of countable sets is countable.
\end{theorem}

\begin{proof}~\label{countableunion}
  Let $A_n$ be a countable set for each $n < \omega$. For each $n$ let us choose an enumeration 
  $\langle a_{n, k} \: : \: k < \omega \rangle$ of $A_n$. So we have a projection of $\mathbb{N} \times \mathbb{N}$ onto
  $\bigcup \limits_{n < \omega} A_n$ by mapping $(n, k) \mapsto a_{n,k}$.
\end{proof}

In fact, the theorem above can be generalised the following way:
\begin{theorem}~\label{union}
  $|S| \leq S \cdot \sup \{ |X| \: | \: X \in S \}$.
\end{theorem}

\begin{proof}
  Let $\kappa = |S|$ and $\lambda = \sup \{ |X| \: | \: X \in S \}$.
  We have $S = \{ X_{\alpha} \: | \: \alpha < \kappa \}$ and for each $\alpha < \kappa$ we choose an 
  enumeration $X_{\alpha} = \{ a_{\alpha, \beta} \: | \: \beta < \lambda_{\alpha} \}$ where $\lambda_{\alpha} = |X_{\alpha}|$.
  Clearly that $\lambda_{\alpha} \leq \lambda$ for each $\alpha < \kappa$.
  So we have a projection of $\kappa \times \lambda$ onto $\cup S$ by mapping $(\alpha, \beta) \mapsto a_{\alpha, \beta}$.
\end{proof}

\begin{col}
  For every $\alpha$ $\aleph_{\alpha + 1}$ is a regular cardinal.
\end{col}

\begin{proof}
  If $\aleph_{\alpha + 1}$ were singular for some $\alpha$, then 
  $\omega_{\alpha + 1}$ would be the union of at most $\aleph_{\alpha}$ 
  sets of cardinality $\aleph_{\alpha}$ by Lemma~\ref{reg}, which
  would mean that $\aleph_{\alpha + 1} = \aleph_{\alpha}$ by Theorem~\ref{union}.
  Contradiction.
\end{proof}

Let $(P, <)$ be a poset, an element $a \in P$ is \emph{maximal} if no $b \in P$ such that $b > a$.
Let $X$ be a non-empty subset of $P$, then $c$ is the \emph{upper bound} of $X$ if $c \geq X$. 
$X$ is a \emph{chain} in $P$ if any two elements of $X$ are comparable.

\begin{theorem} {\bf (Zorn)}

  Let $(P, <)$ be a poset such that every chain $C$ has an upper bound, then $P$ has a maximal element.
\end{theorem}

\begin{proof}
  Let $f$ be a choice function on non-empty subsets of $P$. We construct a chain $C$ leading to a maximal element.

  Construct the following elements by induction:
  \begin{center}
    $a_{\alpha} = \text{an element of $P$ such that $a_{\alpha} > a_{\xi}$ for every $\xi > \alpha$ if it exists}$
  \end{center}

  If $\alpha > 0$ is a limit ordinal, then $C_{\alpha}$ is a chain in $P$ and $a_{\alpha}$ does exist.
  Eventually, there is $\theta$ such that no $a_{\theta+1} > a_{\theta}$. Thus $a_{\theta}$ is maximal.
\end{proof}

As it is known, Zorn's lemma implies the axiom of choice.
Let $S$ be a family of non-empty sets, then we check that the set
$\{ f \: | \: \text{$f$ is a choice function on some $S' \subset S$}\}$ ordered by inclusion
satisfies the condition of Zorn's lemma, so a maximal element of that poset is a choice function on $S$.


There is a weaker version of the axiom of choice for countable families of non-empty sets. 
The countable AC implies Theorem~\ref{countableunion} and regularity of $\aleph_1$, but the
countable AC is too weak to show that $\mathbb{R}$ can be well-ordered.

There is a stronger version of the countable AC.
\begin{definition} {\bf (The Principle of Dependent Choice (DC))}

  Let $R$ be a binary relation on $A$ such that for all $x \in A$ there exists $y \in A$ such that $y R x$, then there is
  a sequence $a_0, a_1, \dots, a_n, \dots$ for $n < \omega$ such that:
  \begin{center}
    $\forall n < \omega \: (a_{n + 1} R a_n)$
  \end{center}
\end{definition}

The Principle of Dependent Choices allows characterising well orderings and (as well as well-founded relations) the following way:

\begin{lemma}
  Let $(A, <)$ be a poset, then the following are equivalent:

  \begin{enumerate}
    \item $(A, <)$ is a well-ordering.
    \item No infinite sequences $a_0, a_1, \dots, a_n, \dots$ for $n < omega$ such that:
    \begin{center}
      $a_0 > a_1 > \dots > a_n > \dots$
    \end{center}
  \end{enumerate}
\end{lemma}

\subsection{Cardinal Arithmetic the Generalised Continuum Hypothesis}

Now let us discuss the cardinal exponentiation operator.

\begin{lemma}~\label{card1}
  Let $\lambda$ be infinite and $2 \leq \kappa \leq \lambda$, then $\kappa^{\lambda} = 2^{\lambda}$.
\end{lemma}

\begin{proof}
  $2^{\lambda} \leq \kappa^{\lambda} \leq (2^{\kappa})^{\lambda} = 2^{\kappa \cdot \lambda} = 2^{\lambda}$.
\end{proof}

The evaluation of $\kappa^{\lambda}$ is more complicated when $\lambda < \kappa$.
If $2^{\lambda} \geq \kappa$, then we have $\kappa^{\lambda} = 2^{\lambda}$ since $\kappa \leq (2^{\lambda})^{\lambda} = 2^{\lambda}$.
But if $2^{\lambda} < \kappa$, the only thing we can conclude:
\begin{center}
  $\kappa \leq \kappa^{\lambda} \leq 2^{\kappa}$
\end{center}
which is already known by Cantor's theorem.

Let $\lambda$ be a cardinal and let $A$ be a set such that $|A| \geq \lambda$, we let:
\begin{center}
  $[A]^{\lambda} = \{ X \in 2^A \: | \: |X| = \lambda \}$
\end{center}

\begin{lemma}~\label{cardfact1}
  If $|A| = \kappa \geq \lambda$, then the set $[A]^{\lambda}$ has cardinality $\kappa^{\lambda}$.
\end{lemma}

\begin{proof}
  On the one hand every function $f : \lambda \to A$ is a subset of $\lambda \times A$ and $|f| = \lambda$. Thus:
  \begin{center}
    $\kappa^{\lambda} \leq |[\lambda \times A]^{\lambda}| = |[A]^{\lambda}|$
  \end{center}
  On the other hand, there is a one-to-one function $F : [A]^{\lambda} \to A^{\lambda}$. 
  If $X \in [A]^{\lambda}$, let $F(X)$ be some function $f$ on $\lambda$ whose range is $X$.
\end{proof}

Let $\lambda$ be a limit cardinal, let:
\begin{center}
  $\kappa^{<\lambda} = \sup \{\kappa^{\mu} \: | \: \text{$\mu$ is a cardinal such that $\mu < \lambda$}\}$
\end{center}
We also define $\kappa^{<\lambda^{+}}$ for successors $\lambda^{+}$.

Let $\kappa$ be an infinite cardinal and $|A| \geq \kappa$, let:
\begin{center}
  $[A]^{<\kappa} = \{ X \in 2^{A} \: | \: |X| < \kappa \}$
\end{center}
Clearly, the cardinality of $[A]^{<\kappa}$ is $|A|^{<\kappa}$.

\subsection{Infinite Sums and Products}

Let $\{ \kappa_i \: | \: i \in I \}$ be an indexed family of cardinals, define:
\begin{center}
  $\sum \limits_{i \in I} \kappa_i = |\bigcup \limits_{i \in I} X_i|$
\end{center}
where each for $i \in I$ $|X_i| = \kappa_i$. Note that, by the Axiom of Choice, the definition of sum does not depend
on the choice of $\{ X_i \: | \: i \in I \}$.

Let $\lambda, \kappa$ be cardinals and let $\kappa_i = \kappa$, then:
\begin{center}
  $\sum \limits_{i < \lambda} \kappa_i = \lambda \cdot \kappa$
\end{center}

More generally, we have:
\begin{lemma}~\label{cardfact2}
  Let $\lambda$ be an infinite cardinal and $\kappa_i > 0$ for each $i < \lambda$, then:
  \begin{center}
    $\sum \limits_{i < \lambda} \kappa_i = \lambda \cdot \sup \limits_{i < \lambda} \kappa_i$
  \end{center}
\end{lemma}
\begin{proof}
  Let $\kappa = \sup \limits_{i < \lambda} \kappa_i$ and $\sigma = \sum \limits_{i < \lambda} \kappa_i$.
  On the one hand, we have $\forall i < \lambda \:\: \kappa_i \leq \kappa$, so
  \begin{center}
    $\sum \limits_{i < \lambda} \kappa_i \leq \lambda \cdot \kappa$
  \end{center}

  On the other hand, since $\kappa_1 \geq 1$ for each $i$, we have
  \begin{center}
  $\lambda = \sum \limits_{i < \lambda} 1 \leq \sigma$
  \end{center}
  $\sigma \geq \kappa_i$ for each $i$, so we have
  \begin{center}
    $\sigma \geq \sup \limits_{i < \lambda} \kappa_i = \kappa$
  \end{center}
  So $\sigma \geq \lambda \cdot \kappa$.
\end{proof}

Let $\{ X_i \: | \: i \in I \}$ be an indexed family of sets, we let:
\begin{center}
  $\prod \limits_{i \in I} X_i = \{ f \: | \: \text{$f$ is a function on $I$ such that $\forall i \in I \:\: f(i) \in X_i$} \}$
\end{center}
If each of $X_i$'s is non-empty, then the whole product is non-empty and this is equivalent to the axiom of choice.

Let $\{ \kappa_i \: | \: i \in I \}$ be a family of cardinals, define:
\begin{center}
  $\prod \limits_{i \in I} \kappa_i = |\prod \limits_{i \in I} X_i|$
\end{center}
where for each $i$ $X_i$ is a set of cardinality of $\kappa_i$. 
As in the case of sum, assuming the axiom of choice, one can show that the definition of product
does not depend on the choice of $X_i$'s.

If $\kappa_i = \kappa$ for each $i \in I$ and $I$ has cardinality $\lambda$, then:
\begin{center}
  $\prod \limits_{i \in I} \kappa_i = \lambda$
\end{center}

If $I$ is a disjoint union $I = \bigcup \limits_{j \in J} A_j$, then:
\begin{center}
  $\prod \limits_{i \in I} \kappa_i = \prod \limits_{j \in J} (\prod \limits_{i \in A_j} \kappa_i)$
\end{center}

If $\kappa_i \geq 2$ for each $i \in I$, then:
\begin{center}
  $\sum \limits_{i \in I} \kappa_i \leq \prod \limits_{i \in I} \kappa_i$
\end{center}
If $I$ is finite, then the inequality is self-evident. Assume $I$ is infinite. We have:
\begin{center}
  $\prod \limits_{i \in I} \kappa_i \geq \prod \limits_{i \in I} 2 = 2^{|I|} > |I|$
\end{center}
We show that $\sum_i \kappa_i \leq |I| \cdot \prod_i \kappa_i$.

Let $\{ X_i \: | \: i \in I \}$ be a disjoint family such that for each $i \in I$ $|X_i| = \kappa_i$.
Assign each $x \in \bigcup_i X_i$ to a pair $(i, f)$ such that $x \in X_i$ and $f \in \prod_i X_i$
such that $f(i) = x$.

\begin{lemma}
  Let $\lambda$ be an infinite cardinal and let $\langle \kappa_i \: : \: i < \lambda \rangle$ 
  be a non-descreasing sequence of ordinals, then
  \begin{center}
    $\prod \limits_{i \in I} \kappa_i = (\sup \limits_{i \in I} \kappa_i)^{\lambda}$
  \end{center}
\end{lemma}
\begin{proof}
  Let $\kappa = \sup_i \kappa_i$. Since $\kappa_i \leq \kappa$ for each $i < \lambda$, we have:
  \begin{center}
    $\prod \limits_{i \in I} \kappa_i \leq \prod \limits_{i \in I} \kappa = \kappa^{\lambda}$
  \end{center}
  Let us show $\kappa^{\lambda} \leq \prod \limits_{i \in I} \kappa_i$.

  Consider a partition of $\lambda$ into $\lambda$ disjoint sets $A_j$, each of which has cardinality $\lambda$:
  \begin{center}
    $\lambda = \bigcup \limits_{j < \lambda} A_j$
  \end{center}

  For each $j < \lambda$ we have:
  \begin{center}
    $\kappa = \sup \limits_{i \in A_j} \kappa_i \leq \prod \limits_{i \in A_j} \kappa_i$
  \end{center}
  And thus:
  \begin{center}
    $\prod \limits_{i \in I} \kappa_i = \prod \limits_{j < \lambda} (\prod \limits_{i \in A_j} \kappa_i) \geq \prod \limits_{j < \lambda} \kappa = \kappa^{\lambda}$
  \end{center}
\end{proof}

\begin{theorem} {\bf (K\"{o}nig)}

  Assume $\kappa_i < \lambda_i$ for each $i \in I$, then:
  \begin{center}
    $\sum \limits_{i \in I} \kappa_i < \prod \limits_{i \in I} \lambda_i$
  \end{center}
\end{theorem}

\begin{proof}
  Let us show $\Sigma_i \kappa_i \ngeq \Pi_i \lambda_i$. 
  Let $\{ T_i \: | \: i \in I\}$ be an indexed family such that $|T_i| = \lambda_i$.
  It sufficies to show that if we have a family $\{ Z_i \: | \: i \in I\}$ of subsets of $T = \Pi_i T_i$
  such that $|Z_i| < \kappa_i$ for each $i$, then $\cup_i Z_i \neq T$.

  For every $i \in I$, let $S_i$ be the projection of $Z_i$ into the $i$-th coordinate:
  \begin{center}
    $S_i = \{ f(i) \: | \: f \in Z_i \}$
  \end{center}

  As far as $|Z_i| < |T_i|$, we have $S_i \subset T_i$ and $S_i \neq T_i$ for each $i \in I$.
  Let $f \in T$ be a function such that $f(i) \notin S_i$. $f$ does not belong to any $Z_i$, so $\cup_i Z_i \neq T$.
\end{proof}

\begin{col}
  $\kappa < 2^{\kappa}$
\end{col}

\begin{proof}
  $\sum \limits_{i < \kappa} 1 < \prod \limits_{i < \kappa} 2$.
\end{proof}

\begin{col}~\label{cofinality:aleph}
  For each $\alpha$ $\aleph_{\alpha} < \operatorname{cf}(2^{\aleph_{\alpha}})$.
\end{col}

\begin{proof}
  Let us show that if for each $i < \omega_{\alpha}$ $\kappa_i < 2^{\aleph_{\alpha}}$,
  then $\Sigma_{i < \omega_{\alpha}} \kappa_i < 2^{\aleph_{\alpha}}$.
  Let $\lambda_i = 2^{\aleph_{\alpha}}$.
  \begin{center}
    $\sum \limits_{i < \omega_{\alpha}} \kappa_i < \prod \limits_{i < \omega_{\alpha}} \lambda_i = (2^{\aleph_{\alpha}})^{\aleph_{\alpha}} = 2^{\aleph_{\alpha}}$
  \end{center}
\end{proof}

\begin{col}
  For all $\alpha, \beta$ $\aleph_{\beta} < \operatorname{cf}(\aleph_{\alpha}^{\aleph_{\beta}})$.
\end{col}

\begin{proof}
  We show that if $\kappa_i < \aleph_{\alpha}^{\aleph_{\beta}}$ for each 
  $i < \omega_{\beta}$, then $\Sigma_{i < \omega_{\beta}} \kappa_i < \aleph_{\alpha}^{\aleph_{\beta}}$.
  Let $\lambda_i = \aleph_{\alpha}^{\aleph_{\beta}}$, then
  \begin{center}
    $\sum \limits_{i < \omega_{\beta}} \kappa_i < \prod \limits_{i < \omega_{\beta}} = \aleph_{\alpha}^{\aleph_{\beta}}$
  \end{center}
\end{proof}

\begin{col}
  Let $\kappa$ be an infinite cardinal, then $\kappa < \kappa^{\operatorname{cf} \kappa}$
\end{col}

\begin{proof}
  Let $i < \operatorname{cf} \kappa$ and $\kappa_i < \kappa$ be such that $\kappa = \Sigma_{i < \operatorname{cf} \kappa} \kappa_i$.
  \begin{center}
    $\kappa = \sum \limits_{i < \operatorname{cf} \kappa} \kappa_i < \prod \limits_{i < \operatorname{cf} \kappa} \kappa = \kappa^{\operatorname{cf} \kappa}$.
  \end{center}
\end{proof}

\subsection{The Continuum Function}

Cantor's theorem claims that $\aleph_{\alpha} < 2^{\aleph_{\alpha}}$, so $\aleph_{\alpha+1} \leq 2^{\aleph_{\alpha}}$
for each $\alpha$. The \emph{Generalised Continuum Hypothesis} (GCH) is the statement
\begin{center}
  $2^{\aleph_{\alpha}} = \aleph_{\alpha+1}$
\end{center}
for each $\alpha$. GCH is independent of ZFC, but ZFC + GCH proves the following properties of
cardinal exponentiation:
\begin{theorem} Assume GCH. Let $\kappa$ and $\lambda$ be infinite cardinals, then:

  \begin{enumerate}
    \item If $\kappa \leq \lambda$, then $\kappa^{\lambda} = \lambda^{+}$.
    \item If $\operatorname{cf} \kappa \leq \lambda < \kappa$, then $\kappa^{\lambda} = \kappa^{+}$.
    \item If $\lambda < \operatorname{cf} \kappa$, then $\kappa^{\lambda} = \kappa$.
  \end{enumerate}
\end{theorem}

\begin{proof}
$ $

\begin{enumerate}
  \item By Lemma~\ref{card1} we have $\kappa^{\lambda} = 2^{\lambda}$, but $2^{\lambda} = \lambda^{+}$.
  \item Combine Lemma~\ref{cardfact1} and Lemma~\ref{cardfact2}.
  \item By Lemma~\ref{boundedaleph} we have:
  \begin{center}
    $\kappa^{\lambda} = \{ \alpha^{\lambda} \: | \: \alpha < \kappa \}$
  \end{center}
  so:
  \begin{center}
    $|\alpha^{\lambda}| \leq 2^{|\alpha| \cdot \lambda} = (|\alpha| \cdot \lambda)^{+} \leq \kappa$
  \end{center}
\end{enumerate}
\end{proof}

The \emph{beth function} is defined by induction:
\begin{enumerate}
  \item $\beth_0 = \aleph_0$
  \item $\beta_{\alpha + 1} = 2^{\beta_{\alpha}}$
  \item $\beta_{\alpha} = \sup \{ \beta_{\beta} \: | \: \beta < \alpha \}$ if $\alpha$ is limit ordinal.
\end{enumerate}

So we can reword GCH as $\beta_{\alpha} = \aleph_{\alpha}$ for all $\alpha$.

Now we study the behaviour of the continuum function $\kappa \mapsto 2^{\kappa}$ assuming no GCH.

\begin{theorem}~\label{continuum:func} Let $\kappa, \lambda$ be cardinals, then

  \begin{enumerate}
    \item If $\kappa < \lambda$, then $2^{\kappa} \leq 2^{\lambda}$.
    \item $\kappa < \operatorname{cf} 2^{\kappa}$
    \item~\label{continuum:func:3} If $\kappa$ is a limit cardinal, then $2^{\kappa} = (2^{<\kappa})^{\operatorname{cf} \kappa}$
  \end{enumerate}
\end{theorem}

\begin{proof}
  
  $ $

  \begin{enumerate}
    \item Fairly obvious.
    \item Corollary~\ref{cofinality:aleph}.
    \item Let $\kappa = \Sigma_{i < \operatorname{cf} \kappa} \kappa_i$ where each $\kappa_i < \kappa$ for each $i$.
    We have
    \begin{center}
      $2^{\kappa} = 2^{\Sigma_{i < \operatorname{cf} \kappa} \kappa_i} = \prod \limits_{i < \operatorname{cf} \kappa} 2^{\kappa_i} \leq \prod \limits_{i < \operatorname{cf} \kappa} 2^{< \kappa} = (2^{< \kappa})^{\operatorname{cf} \kappa} \leq (2^{\kappa})^{\operatorname{cf} \kappa} \leq 2^{\kappa}$
    \end{center}
  \end{enumerate}
\end{proof}

\begin{col}
  Let $\kappa$ be a singular cardinal. 
  Assume the continuum function is eventually constant below $\kappa$, with value $\lambda$, 
  then $2^{\kappa} = \lambda$.
\end{col}

\begin{proof}
  If $\kappa$ is singular and it satisfies the assumption of the statement, then
  there is $\nu$ such that $\operatorname{cf} \kappa \leq \nu < \kappa$ and that
  $2^{< \kappa} = \lambda = 2^{\nu}$. Thus:
  \begin{center}
    $2^{\kappa} = (2^{<\kappa})^{\operatorname{cf} \kappa} = 2^{\nu}$.
  \end{center}
\end{proof}

The \emph{gimel function} is the function:
\begin{center}
  $\gimel(\kappa) = \kappa^{\operatorname{cf} \kappa}$
\end{center}

If $\kappa$ is a limit cardinal and the continuum function below $\kappa$ is not eventually constant,
then the cardinal $\lambda = 2^{<\kappa}$ is a limit of a non-decreasing sequence:
\begin{center}
  $\lambda = 2^{<\kappa} = \lim \limits_{\alpha \to \kappa} 2^{|\alpha|}$
\end{center}
Then, by Lemma~\ref{confin1}, $\operatorname{cf} \lambda = \operatorname{cf} \kappa$.
Thus, by Theorem~\ref{continuum:func}(\ref{continuum:func:3}), we have:
\begin{center}
$2^{\kappa} = (2^{<\kappa})^{\operatorname{cf} \kappa} = 2^{\operatorname{cf} \lambda}$
\end{center}

If $\kappa$ is regular, then $\kappa = \operatorname{cf} \kappa$ and, since 
$\kappa^{\kappa} = 2^{\kappa}$ we have:
\begin{center}
  $2^{\kappa} = \kappa^{\operatorname{cf} \kappa}$
\end{center}

So we can specify the behaviour of the continuum function in terms of the gimel function.
\begin{col}
  $ $

  \begin{enumerate}
    \item If $\kappa$ is a successor cardinal, then $2^{\kappa} = \gimel(\kappa)$.
    \item If $\kappa$ is a limit cardinal and $\lambda x. 2^x$ below $\kappa$ is eventually constant, then
    $2^{\kappa} = 2^{<\kappa} \cdot \gimel(\kappa)$.
    \item If $\kappa$ is a limit cardinal and $\lambda x. 2^x$ below $\kappa$ is not eventually constant, then
    $2^{\kappa} = \gimel( 2^{<\kappa})$.
  \end{enumerate}
\end{col}

\subsection{Cardinal Exponentiation}

Let $\kappa$ be a regular cardinal and let $\lambda < \kappa$, then every function $f : \lambda \to \kappa$ is bounded,
i.e., $\sup \{ f(\xi) \: | \: \xi < \lambda \} < \kappa$. Thus:
\begin{center}
  $\kappa^{\lambda} = \bigcup \limits_{\alpha < \kappa} \alpha^{\lambda}$
\end{center}
that is,
\begin{center}
  $\kappa^{\lambda} = \sum \limits_{\alpha < \kappa} |\alpha|^{\lambda}$
\end{center}
If $\kappa$ is a successor cardinal, then we obtain the \emph{Hausdorff formula}:
\begin{center}
  $\aleph_{\alpha + 1}^{\beta} = \aleph_{\alpha}^{\aleph_{\beta}} \cdot \aleph_{\alpha + 1}$
\end{center}

We can compute $\kappa^{\lambda}$ using the following fact.
If $\kappa$ is a limit cardinal, we use use the notation:
\begin{center}
  $\lim \limits_{\alpha \to \kappa} \alpha^{\lambda} := \sup \{ \mu^{\lambda} \: | \: \text{$\mu$ is a cardinal and $\mu < \kappa$}\}$
\end{center}

\begin{lemma}~\label{limitcard:lim}
  Let $\kappa$ be a limit cardinal and assume that $\operatorname{cf} \kappa \leq \lambda$, then
  \begin{center}
    $\kappa^{\lambda} = (\lim \limits_{\alpha \to \kappa} \alpha^{\lambda})^{\operatorname{cf} \kappa}$
  \end{center}
\end{lemma}
\begin{proof}
  Let $\kappa = \Sigma_{i < \operatorname{cf} \kappa} \kappa_i$, where $\kappa_i < \kappa$ for each $i$.
  We have:
  \begin{center}
    $\kappa^{\lambda} \leq (\prod \limits_{i < \operatorname{cf} \kappa} \kappa_i)^{\lambda} = 
    \prod \limits_{i < \operatorname{cf} \kappa} \kappa_i^{\lambda} \leq 
    \prod \limits_{i < \operatorname{cf} \kappa} (\lim \limits_{\alpha \to \kappa} \alpha^{\lambda})^{\operatorname{cf} \kappa} \leq (\kappa^{\lambda})^{\operatorname{cf} \kappa} = \kappa^{\lambda}$
  \end{center}
\end{proof}

\begin{theorem}~\label{cardexp}

  Let $\lambda$ be an infinite cardinal, then for all infinite cardinals $\kappa$, 
  the value of $\kappa^{\lambda}$ is computed as follows:

  \begin{enumerate}
    \item $\kappa \leq \lambda$ implies $\kappa^{\lambda} = 2^{\lambda}$.
    \item If there exists $\mu < \kappa$ such that $\kappa \leq \mu^{\lambda}$,
    then $\kappa^{\lambda} = \mu^{\lambda}$.
    \item Assume $\kappa > \lambda$ and if for all $\mu < \kappa$ $\mu^{\lambda} < \kappa$, then:
    \begin{enumerate}
      \item $\operatorname{cf} \kappa > \lambda$ implies $\kappa^{\lambda} = \kappa$.
      \item $\operatorname{cf} \kappa \leq \lambda$ implies $\kappa^{\lambda} = \kappa^{\operatorname{cf} \kappa}$.
    \end{enumerate}
  \end{enumerate}
\end{theorem}

\begin{proof}
  $ $

  \begin{enumerate}
    \item Follows from Lemma~\ref{card1}.
    \item $\mu^{\lambda} \leq \kappa^{\lambda} \leq (\mu^{\lambda})^{\lambda} = \mu^{\lambda}$.
    \item If $\kappa$ is a successor cardinal, then apply the Hausdorff formula.
    If $\kappa$ is a limit cardinal. We have $\kappa = \lim_{\alpha \to \kappa} \alpha^{\lambda}$.

    If $\operatorname{cf} \kappa > \lambda$, then every $f : \lambda \to \kappa$ is bounded and we have:
    \begin{center}
      $\kappa^{\lambda} = \lim \limits_{\alpha \to \kappa} \alpha^{\lambda} = \kappa$.
    \end{center}

    If $\operatorname{cf} \kappa \leq \lambda$, then, by Lemma~\ref{limitcard:lim}, we have:
    \begin{center}
      $\kappa^{\lambda} = (\lim \limits_{\alpha \to \kappa} \alpha^{\lambda})^{\operatorname{cf} \kappa} = \kappa^{\operatorname{cf} \kappa}$
    \end{center}
  \end{enumerate}
\end{proof}

Theorem~\ref{cardexp} allows defining all cardinal exponentiation in terms of the gimel function:
\begin{col}
  Let $\kappa$ and $\lambda$ be cardinals, then the value of $\kappa^{\lambda}$ 
  is either $2^{\lambda}$, or $\kappa$ or $\gimel(\mu)$ for some $\mu$ such that 
  $\operatorname{cf} \mu \leq \lambda < \mu$.
\end{col}

\begin{proof}
  Assume $\kappa^{\lambda} > 2^{\lambda} \cdot \kappa$. 
  Let $\mu$ be the least cardinal such that $\mu^{\lambda} = \kappa^{\lambda}$, so, by Theorem~\ref{cardexp},
  $\mu^{\lambda} = \mu^{\operatorname{cf} \mu}$.
\end{proof}

A cardinal $\kappa$ is a \emph{strong limit} cardinal if
\begin{center}
  $\forall \lambda < \kappa \:\: 2^{\lambda} < \kappa$
\end{center}

Every strong limit cardinal is a limit cardinal, and, assuming the generalised continuum hypothesis, the converse is also true.
If $\kappa$ is a strong limit cardinal, then
\begin{center}
  $\forall \lambda, \nu < \kappa \:\: \lambda^{\nu} < \kappa$
\end{center}
$\aleph_0$ is the smallest strong limit cardinal. Also, strong limit cardinals form a proper class: 
if $\alpha$ is an arbitrary cardinal, then the cardinal
\begin{center}
  $\kappa = \{ \alpha, 2^{\alpha}, 2^{2^{\alpha}}, \dots \}$
\end{center}
(of cofinality $\omega$) is a strong limit cardinal.

Also, if $\kappa$ is a strong limit cardinal, then $2^{\kappa} = \kappa^{\operatorname{cf} \kappa}$.
A cardinal $\kappa$ is \emph{strongly inaccessible} if $\kappa > \aleph_0$, if $\kappa$ is strong limit and regular.
Every strongly inaccessible cardinal is strongly inaccessible, and the converse is true assuming the generalised continuum hypothesis.
Generally, inaccessibility describes the impossibility of being obtained from smaller cardinals by usual set-theoretic operations:
\begin{center}
  $|X| < \kappa \Rightarrow 2^{|X|} < \kappa$.

  $|S| < \kappa$ and $|X| < \kappa$ for each $X \in S$, then $|\cup S| < \kappa$.
\end{center}

\subsection{The Singular Cardinal Hypothesis}

The \emph{Singular Cardinal Hypothesis} (SCH) states that
\begin{center}
  If $\kappa$ is singular, then $2^{\operatorname{cf} \kappa} < \kappa$ implies $2^{\operatorname{cf} \kappa} = \kappa^{+}$.
\end{center}
The singular cardinal hypothesis follows from the generalised continuum hypothesis. 
Indeed, if $\kappa \leq 2^{\operatorname{cf} \kappa}$, then $\kappa^{\kappa} = 2^{\operatorname{cf} \kappa}$.
If $2^{\operatorname{cf} \kappa} < \kappa$, then $\kappa^{+}$ is the least possible value of $\kappa^{\operatorname{cf} \kappa}$.

The singular cardinal hypothesis allows determining cardinal exponentiation by the values of
the continuum function on regular cardinals.

\begin{theorem}
  Assume SCH holds, then:

  \begin{enumerate}
    \item If $\kappa$ is a singular cardinal, then:
    \begin{enumerate}
      \item If the continuum function is eventually constant below $\kappa$, then $2^{\kappa} = 2^{< \kappa}$.
      \item $2^{\kappa} = (2^{< \kappa})^+$ otherwise.
    \end{enumerate}
    \item If $\kappa$ and $\lambda$ are infinite cardinals, then:
    \begin{enumerate}
      \item If $\kappa \leq 2^{\lambda}$, then $\kappa^{\lambda} = 2^{\lambda}$.
      \item If $2^{\lambda} < \kappa$, then $\lambda < \operatorname{cf} \kappa$ implies $\kappa = \kappa^{\lambda}$.
      \item If $2^{\lambda} < \kappa$, then $\operatorname{cf} \kappa \leq \lambda$ implies $ \kappa^{\lambda} = \kappa^+$.
    \end{enumerate}
  \end{enumerate}
\end{theorem}

\section{The Axiom of Regularity}

The \emph{Axiom of Regularity} states that the membership relation on any family of sets is well-founded:
\begin{center}
  $\forall S (S \neq \emptyset \to \exists s \in S \: S \cap x = \emptyset)$
\end{center}
that is, no infinite sequences are allowed:
\begin{center}
  $x_0 \ni x_1 \ni x_2 \ni \dots$
\end{center}
neither are cycles:
\begin{center}
  $x_0 \ni x_1 \ni x_2 \ni \dots \ni x_n \ni x_0$
\end{center}

Thus the Axiom of Regularity prevents some sets from existing. This is of interest for metamathematics of set theory, in
particular, we can classify all sets with respect to ranks and arrange them in a cumulative hierarchy.

Recall that a set $A$ is \emph{transitive} if $x \in A$ implies $x \subseteq A$.

\begin{lemma}
  Let $S$ be a set, then there exists a transitive set $T \supset S$.
\end{lemma}
\begin{proof}
  By induction:

  \begin{enumerate}
    \item $S_0 = S$
    \item $S_{n + 1} = \bigcup S_n$
    \item $T = \bigcup \limits_{n < \omega} S_n$
  \end{enumerate}
\end{proof}

$\operatorname{TC}(S)$ is the \emph{transitive closure} of $S$, that is, the minimal transitive set extending $S$.

\begin{lemma}
  Let $C$ be a non-empty class, then $C$ has an $\in$-minimal element.
\end{lemma}

\begin{proof}
  Let $S$ be a set from $C$. If $S \cap C = \emptyset$, then $S$ is minimal.
  Otherwise take $X = T \cap C$ where $T = \operatorname{TC}(S)$ and $X \neq \emptyset$.
  Then $X$ has a minimal $x$ such that $x \cap X = \emptyset$, then $x \cap C = \emptyset$.
\end{proof}

\subsection{The Cumulative Hierarchy of Sets}

We define by transfinite induction:

\begin{enumerate}
  \item $\mathcal{V}_0 = \emptyset$
  \item $\mathcal{V}_{\alpha + 1} = 2^{\mathcal{V}_\alpha}$
  \item $\mathcal{V}_{\alpha} = \bigcup \limits_{\beta < \alpha} \mathcal{V}_{\beta}$
\end{enumerate}

By induction, one can show the following:
\begin{enumerate}
  \item Each $\mathcal{V}_{\alpha}$ is transitive.
  \item $\alpha < \beta$ implies $\mathcal{V}_{\alpha} \subset \mathcal{V}_{\beta}$.
  \item $\alpha \subset \mathcal{V}_{\alpha}$.
\end{enumerate}

\begin{lemma}
  For every $x$ there exists $\alpha$ such that $x \in \mathcal{V}_{\alpha}$:
\begin{center}
  $\bigcup \limits_{\alpha} \mathcal{V}_{\alpha} = \mathcal{V}$
\end{center}
where $V = \{ x \: | \: x = x\}$.
\end{lemma}

\begin{proof}
Let $C$ be the class of all $x$ that no $\alpha$ exists such that $x \in \mathcal{V}_{\alpha}$.
If $C$ is non-empty, then $C$ has an $\in$-minimal element $x$.
That, $x \in C$ and $z \in \cup_{\alpha} \mathcal{V}_{\alpha}$ for some $\alpha$ for each $z \in x$.
Hence $x \subset \cup_{\alpha \in \operatorname{Ord}} \mathcal{V}_{\alpha}$.
By Replacement, there exists $\gamma$ such that $x \subset \cup_{\alpha < \gamma} \mathcal{V}_{\alpha}$, so 
$x \in \mathcal{V}_{\gamma + 1}$. So $C$ cannot be empty.
\end{proof}

Since every $x$ belongs to some $\mathcal{V}_{\alpha}$ for some $\alpha$, we can define \emph{the rank of $x$}:
\begin{center}
  $\operatorname{rank}(x) = \text{the smallest ordinal $\alpha$ such that $x \in \mathcal{V}_{\alpha + 1}$}$
\end{center}
Thus each $V_{\alpha}$ is a collection of sets having lower ranks and we have:
\begin{enumerate}
  \item $x \in y$ implies $\operatorname{rank}(x) < \operatorname{rank}(y)$.
  \item $\operatorname{rank}(\alpha)=\alpha$.
\end{enumerate}

The rank function is often needed when we deal with equivalence classes for equivalence relation on a proper class.
Let $C$ be a class, let
\begin{center}
  $\hat{C} = \{ x \in C \: | \: \forall z \in C \: \operatorname{rank}(x) \leq \operatorname{rank}(x) \}$
\end{center}
Note that $\hat{C}$ is always set and $\hat{C}$ is non-empty whenever $C$ is non-empty.

Let $\equiv$ be an equivalence relation on $C$. Apply the definition above to each equivalence class and define
\begin{center}
  $[x] = \{ y \in C \: | \: y \equiv x \land \forall z \in C \: (z \equiv x \to \operatorname{rank}(y) \leq \operatorname{rank}(z)) \}$
\end{center}
and
\begin{center} 
$C/_{\equiv} = \{ [x] \: | \: x \in C \}$
\end{center}

One can use the same to prove the \emph{Collection Principle}:
\begin{center}
  $\forall X \: \exists Y \:\: (\forall u \in X) [\exists v \varphi(u, v, p) \to (\exists v \in Y) \varphi(u, v, p)]$
\end{center}
where $p$ is a parameter.

We can formulate the collection principle the following way. 
Let $C_u$ be a collection of classes for $u \in X$, where $X$ is a set, then
there exists a set $Y$ such that for every $u \in X$
\begin{center}
  $C_u \neq \emptyset \Rightarrow C_u \cap Y = \emptyset$
\end{center}

To prove the collection principle, we let
\begin{center}
  $Y = \bigcup \limits_{u \in X} \hat{C}_u$
\end{center}
where $C_u = \{ v \: | \: \varphi(u, v, p) \}$, that is, 
\begin{center}
  $v \in Y \leftrightarrow \exists u \in X (\varphi(u, v, p) \: \& \: \forall z (\varphi(u, z, p) \to \operatorname{rank} v \leq \operatorname{rank} z))$
\end{center}

By Replacement, $Y$ is a set.

\subsection{$\in$-induction}

\begin{theorem}~\label{in:ind}
  Let $T$ be a transitive class and let $\Phi$ be a property such that:
  \begin{enumerate}
    \item $\Phi(\emptyset)$
    \item $x \in T \: \& \: \forall z \in x \: \Phi(z) \Rightarrow \Phi(x)$
  \end{enumerate}
  then every element of $T$ satisfies $\Phi$.
\end{theorem}

\begin{proof}
  Let $C$ be the class of all $x \in T$ such that $\Phi$ is not the case for $x$.
  If $C$ is non-empty, then either $\neg \Phi(\emptyset)$ or there
  exists $x \in T$ such that there exists $z \in x$ such that $\Phi(z)$ and $\neg \Phi(x)$.
\end{proof}

\begin{theorem}~\label{in:rec}
  Let $T$ be a non-empty transitive class and let $G$ be a function. Then there exists a unique function $F$ on $T$ such that
  \begin{center}
  $\forall x \in T \:\: F(x) = G(F \upharpoonright x)$
  \end{center}
\end{theorem}

\begin{proof}
  Let $x \in T$, we let $F(x) = y$ if and only if there exists a function $f$ such that 
  $\operatorname{dom}(f)$ is a transitive subset $T$ and 
  \begin{enumerate}
    \item $\forall z \in \operatorname{dom}(f) \:\: f(z) = G(f \upharpoonright z)$
    \item $f(x) = y$
  \end{enumerate}

  The uniqueness is proved by $\in$-induction.
\end{proof}

\begin{col}
  Let $A$ be a class, there is a unique class $B$ such that
  \begin{center}
    $B = \{ x \in A \:| \: x \subset B \}$
  \end{center}
\end{col}

\begin{proof}
  Let
\begin{center}
  $F(x) =
  \begin{cases}
    1, \:\: \text{if $x \in A$ and $F(z) = 1$ for all $z \in x$} \\
    0, \:\: \text{otherwise}
  \end{cases}$
\end{center}

Let $B = \{ x \: |\: F(x) = 1 \}$. The uniqueness is proved by $\in$-induction.
\end{proof}

In such case we say that each $x \in B$ is \emph{hereditarily} in $A$.

The Axiom of Regularity also implies that the universe does not admit non-trivial automorphisms.

\begin{theorem}~\label{trivialautomors}
  Let $T_1$ and $T_2$ be transitive classes and let $\pi$ be an $\in$-automorphism
  of $T_1$ onto $T_2$, i.e. $\pi$ is one-to-one and
  \begin{center}
    $u \in v \leftrightarrow \pi u \in \pi v$
  \end{center}

  Then $T_1 = T_2$ and $\pi u = u$ for every $u \in T_1$.
\end{theorem}
\begin{proof}
  One can show by $\in$-induction that $\pi x = x$ for each $x \in T_1$.
  Assume $\pi z = z$ for each $z \in x$ and let $y = \pi x$.

  We have $x \subset y$, then, as far as $z \in x$, we have $z = \pi z \in \pi x = y$.

  We also have $y \subset x$. Let $t \in y$. Provided $y \subset T_2$, there is
  $z \in T_1$ such that $\pi z = t$. Since $\pi z \in y$, we have $z \in x$ and so
  $t = \pi z = z$. Thus $t \in x$.
  Therefore $\pi x = x$ for each $x \in T_1$ and $T_1 = T_2$.
\end{proof}

\subsection{Well-Founded Relations}

Let $E$ be a binary relation on a class $P$. Let $x \in P$, we let the \emph{extension} of $x$:
\begin{center}
  $\operatorname{ext}_E(x) = \{ z \in P \: | \: z E x \}$
\end{center}

\begin{definition}
  A relation $E$ on $P$ is \emph{well-founded} if
  \begin{enumerate}
    \item Every non-empty set $x \subset P$ has an $E$-minimal element. 
    \item For all $x \in P$ $\operatorname{ext}_E(x)$ is a set.
  \end{enumerate}
\end{definition}

\begin{lemma}
  Let $E$ be a well-founded relation on a class $P$, then every class $C \subset P$ has an $E$-minimal element.
\end{lemma}

\begin{proof}
  We need some $x \in C$ such that $\operatorname{ext}_E(x) \cap x = \emptyset$.
  Let $S \in C$ be arbitrary assume $\operatorname{ext}_E(S) \cap C \neq \emptyset$.
  We let $X = T \cap C$ where
  \begin{enumerate}
    \item $S_0 = \operatorname{ext}_E(S)$ 

    \item $S_{n + 1} = \bigcup \limits_{n} \{ \operatorname{ext}_E(z) \: |\: z \in S_n \}$

    \item $T = \bigcup \limits_{n < \omega} S_n$.
  \end{enumerate}
\end{proof}

The following two theorems are proved similarly to Theorem~\ref{in:ind} and Theorem~\ref{in:rec} respecitvely.

\begin{theorem}
  Let $E$ be a well-founded relation on $P$ and let $\Phi$ be a property such that
  \begin{enumerate}
    \item Every $E$-minimal element of $P$ satisfies $\Phi$.
    \item IF $x \in P$ and if for each $z$ such that $z E x$ $\Phi(z)$ is the case, then $\Phi(x)$ holds.
  \end{enumerate}

  Then $\Phi$ holds for every element of $P$.
\end{theorem}

\begin{theorem}
  Let $E$ be a well-founded relation on $P$. Let $G$ be a function on $\mathcal{V} \times \mathcal{V}$, 
  then there exists a unique function $F$ on $P$ such that for each $x \in P$
  \begin{center}
    $F(x) = G(x, F \upharpoonright \operatorname{ext}_E(x))$.
  \end{center}
\end{theorem}

\begin{example} {\bf (The Rank Function) }

  Let us define, by induction, for all $x \in P$
  \begin{center}
    $\rho(x) = \sup \{ \rho(z) + 1 \: | \: z E x \}$.
  \end{center}
  The codomain of $\rho$ is either a particular ordinal or the class of all ordinals. One has
  \begin{center}
    $\forall x, y \in P \:\: (x E y \to \rho(x) < \rho(y))$.
  \end{center}
\end{example}

\begin{example} {\bf (The Transitive Collapse)}

  By induction, let
  \begin{center}
    $\forall x \in P \:\: \pi(x) = \{ \pi(z) \: | \: z E x \}$.
  \end{center}

  The range of $\pi$ is a transitive class such that
  \begin{center}
    $\forall x, y \in P \:\: (x E y \to \pi(x) \in \pi(y))$
  \end{center}
\end{example}

$\pi$ is one-to-one whenever $E$ is extensional.

\begin{definition}
  A well-founded relation $E$ on a class $P$ is \emph{extensional} if
  \begin{center}
    $\forall x, y \in P \:\: x \neq y \to \operatorname{ext}_E(x) \neq \operatorname{ext}_E(y)$
  \end{center}
\end{definition}
A class $M$ is \emph{extensional} if the membership relation on $M$ is extensional, that is,
\begin{center}
  $\forall x, y \in M \:\: x \neq y \to x \cap M \neq y \cap M$.
\end{center}

\begin{theorem} {\bf (Mostowski's Collapsing Theorem)}

  \begin{enumerate}
    \item Let $E$ be a well-founded relation and extensional relation on a class $P$, then
    there exists a unique transitive class $M$ and a unique isomorphism $\pi : (P, E) \cong (M, \in)$.
    \item Every extensional class $P$ is isomorphic to some transitive class $M$.
    \item~\label{most:iii} In the case of the previous item, if $T \subset P$ is transitive, 
    then $\pi x = x$ for every $x \in T$.
  \end{enumerate}
\end{theorem}

\begin{proof} Let us show the general existence of an isomorphism.

$E$ is well-founded, so we can define $\pi$ by well-founded induction.
That is, take any $x \in P$, then $\pi(x)$ can be defined by $\pi(z)$'s where $z E x$:
\begin{center}
  $\pi(x) = \{ \pi(z) \: | \: z E x \}$
\end{center}
In particular, if $E$ is the membership relation, then $\pi(x)$ becomes
\begin{center}
  $\pi(x) = \{ \pi(z) \: | \: z \in x \cap P \}$.
\end{center}
$\pi$ maps $P$ onto the class $M = \pi(P)$, which turns out to be transitive by the definition of $\pi$.
 
We use extensionality of $E$ in order to show that $\pi$ is one-to-one.
Let $z \in M$ be of least rank such that $z = \pi(x) = \pi(y)$ for some different $x, y \in P$.
$x \neq y$ implies $\operatorname{ext}_E(x) \neq \operatorname{ext}_E(y)$. In other words,
there is some $u \in \operatorname{ext}_E(x)$ such that $u \notin \operatorname{ext}_E(y)$.

Let $t = \pi(u)$. Since $t \in z = \pi(y)$, there is $v \in \operatorname{ext}_E(y)$ such that
$t = \pi(v)$. Thus we have $t = \pi(u) = \pi(v)$ for different $u, v$ of less rank that $z$ has since $t \in z$. Contradiction.

Now it follows that
\begin{center}
  $x E y \leftrightarrow \pi(x) E \pi(y)$
\end{center}
because

  $\begin{array}{lll}
    & xEy \leftrightarrow & \\
    & \:\:\:\: \text{By the definition of $\pi$}& \\
    & \pi(x) E \pi(y) \to & \\
    & \:\:\:\: \text{By the definition of $\pi$}& \\
    & \exists z \: z E y \: \& \: \pi(x) = \pi(z) \to & \\
    & \:\:\:\: \text{As far as $\pi$ is one-to-one}& \\
    & x = z \land x E y & \\
  \end{array}$

The uniqueness of the isomorphism as well as the transitive class $M = \pi(P)$
by Theorem~\ref{trivialautomors}. Let $\pi_1$ and $\pi_2$ be two isomorphisms of $P$ and $M_1$ and $M_2$,
then $\pi_2 \circ \pi_1^{-1} : M_1 \cong M_2$ and therefore $\pi_2 \circ \pi_1^{-1}$ is the identity mapping.
Thus $M_1 = M_2$.

To show (\ref{most:iii}), let $T \subset P$ be transitive. 
Observe $x \subset P$ for every $x \in T$ and $x \cap P = x$ and we have
\begin{center}
  $\pi(x) = \{ \pi(z) \: | \: z \in x \}$
\end{center}
for all $x \in T$. By $\in$-induction one can show that $\pi(x) = x$ for each $x \in T$.
\end{proof}

\subsection{The Bernays-G\"{o}del Axiomatic Set Theory}

In the Bernays-G\"{o}del set theory we consider two types of objects: \emph{sets} (denoted with lowercase letters) and 
\emph{classes} (denoted with uppercase letters).

\begin{enumerate}
  \item Extensionality: $\forall u (u \in X \leftrightarrow u \in Y) \to X = Y$.
  \item Every set is a class.
  \item If $X \in Y$, then $X$ is a set.
  \item Pairing: if $x, y$ are sets, so is $\{x, y\}$.
  \item Let $\varphi$ be a formula where only set variables are quantified, then
  \begin{center}
    $\forall X_1 \dots \forall X_n \exists Y \:\: Y = \{ x \: | \: \varphi(x, X_1, \dots, X_n )\}$
  \end{center}
  \item Infinite: there exists an infinite set.
  \item Union: for every set $x$ $\cup x$ exists.
  \item Powerset: for every set $x$ the powerset $P(x)$ exists.
  \item Replacement: if a class $F$ is a function and $x$ is a set, then $\{ F(z) \: | \: z \in x \}$ is a set.
  \item Regularity.
  \item Choice: any family of non-empty sets has a choice function.
\end{enumerate}

\subsection{Some Exercises}

\begin{exercise}
  $\operatorname{rank}(x) = \sup \{ \operatorname{rank}(z) + 1 \: | \: z \in x \}$
\end{exercise}

\begin{proof}
  If $x = \emptyset$, then 
  \begin{center}
    $\operatorname{rank}(\emptyset) = 1 = \sup \{ \operatorname{rank}(\emptyset) + 1 \: \}$
  \end{center}

  If there is $z \in x$, then assume that 
  \begin{center}
    $\operatorname{rank}(z) = \sup \{ \operatorname{rank}(z') + 1 \: | \: z' \in z \}$
  \end{center}

  On the other hand, we have
  \begin{center}
    $\operatorname{rank}(x) = \text{the least $\alpha$ such that $x \in V_{\alpha + 1}$}$
  \end{center}
  which is $\cup \operatorname{rank}(z) = \sup \{ \operatorname{rank}(z) + 1 \: | \: z \in x \}$.
\end{proof}

\begin{exercise}
  $|\mathcal{V}_{\omega + \alpha}| = \beth_{\alpha}$
\end{exercise}

\begin{proof}

  \begin{enumerate}
    \item $\alpha = 0$, then let us show $|\mathcal{V}_{\omega}| = \aleph_0 = \beth_0$.

    Recall that 
    \begin{center}
      $\mathcal{V}_{\omega} = \bigcup \limits_{n < \omega} \mathcal{V}_n$
    \end{center}
    In turn each of $\mathcal{V}_n$ is finite and we have
    \begin{center}
      $\mathcal{V}_0 \subset \mathcal{V}_1 \subset \dots \mathcal{V}_n \subset \dots$ for $n < \omega$.
    \end{center}
    so the whole union $\cup_{n < \omega} \mathcal{V}_n$ is countable.
    \item Assume $\alpha = \beta + 1$, we have to show that $|\mathcal{V}_{\omega + \beta + 1}| = \beth_{\beta + 1} = 2^{\beth_{\beta}}$

    Assume we have already showed that $|\mathcal{V}_{\omega + \beta}| = \beth_{\beta}$. But then
    
    \begin{center}
    $|\mathcal{V}_{\omega + (\beta + 1)}| = |\mathcal{V}_{(\omega + \beta) + 1}| = |P(\mathcal{V}_{\omega + \beta})| = 2^{\beth_{\beta}}$.
    \end{center}
    \item Let $\alpha = \sup \{\beta \: | \: \beta < \alpha \}$
    and assume $\mathcal{V}_{\omega + \beta} = \beth_{\beta}$ for each $\beta < \omega$.

    We have

    $\begin{array}{lll}
      & |\mathcal{V}_{\omega + \alpha}| = & \\
      & |\mathcal{V}_{\omega + \sup \{ \beta \: | \: \beta < \alpha\}}| = & \\
      & |\mathcal{V}_{\sup \{ \omega + \beta \: | \: \beta < \alpha \}} |= & \\
      & |\bigcup \limits_{\gamma} \mathcal{V}_{\gamma} | = & \\
      & \sup \{ \beth_{\beta} \: | \: \beta < \alpha\} = \beth_{\alpha}.&
    \end{array}$
  \end{enumerate}
\end{proof}


\section{Filters, Ultrafilters and Boolean Algebras}

\bibliographystyle{alpha}
\bibliography{Text}


\end{document}