\documentclass[8pt]{article}
\usepackage{graphicx} % Required for inserting images
\usepackage{amsthm}
\usepackage{amsmath}
\usepackage{amsfonts}
\usepackage{amsopn}
\usepackage{comment}
\usepackage{amssymb}
\usepackage{hyperref}
\usepackage{bussproofs}
\usepackage[all, 2cell]{xy}
\usepackage[all]{xy}
\usepackage{rotating}
\usepackage{lscape}
\usepackage{minted}


\theoremstyle{definition}
\newtheorem{definition}{Definition}[section]

\theoremstyle{definition}
\newtheorem{theorem}{Theorem}[section]

\theoremstyle{definition}
\newtheorem{claim}{Claim}[section]

\theoremstyle{definition}
\newtheorem{ex}{Example}[section] 

\theoremstyle{definition}
\newtheorem{cons}{Construction}[section] 

\theoremstyle{definition}
\newtheorem{rem}{Remark}[section] 


\theoremstyle{definition}
\newtheorem{prop}{Proposition}[section]

\theoremstyle{definition}
\newtheorem{lemma}{Lemma}[section]

\theoremstyle{definition}
\newtheorem{fact}{Fact}[section]

\theoremstyle{definition}
\newtheorem{remark}{Remark}[section]

\theoremstyle{definition}
\newtheorem{notation}{Notation}[section]

\theoremstyle{definition}
\newtheorem{example}{Example}[section]

\theoremstyle{definition}
\newtheorem{col}{Corollary}[section]

\theoremstyle{question}
\newtheorem{question}{Question}

\let\strokeL\L
\renewcommand\L{\mathbf{L}}

\title{Some Notes on Set Theory, Pt 1}
\author{Daniel Rogozin}
\date{ }

\begin{document}

\maketitle

\tableofcontents

\newpage

\section{Cardinals}

An ordinal number $\alpha$ is a \emph{cardinal number} if no $\beta < \alpha$ such that $|\alpha| = |\beta|$.
Further, we shall use $\kappa, \lambda, \mu$ to denote cardinal numbers.

Let $W$ be a well-ordered set, then there exists an ordinal $\alpha$ such that $|W| = |\alpha|$, so we let:
\begin{center}
  $|W| = \text{the least ordinal $\alpha$ such that $|W| = |\alpha|$}$
\end{center}

An \emph{aleph} is an infinite cardinal number. 

Let $\alpha$ be an ordinal, then $\alpha^+$ is the least cardinal bigger than $\alpha$.

\begin{lemma}
  $ $

  \begin{enumerate}
    \item For every $\alpha$ there is a cardinal number $\kappa$ such that $\kappa > \alpha$.
    \item Let $X$ be a set of cardinal, then $\sup X$ is a cardinal.
  \end{enumerate}
\end{lemma}

\begin{proof}
  $ $

  \begin{enumerate}
    \item Let $X$ be a set, let 
    \begin{center}
      $h(X) = \text{the least $\alpha$ such that no injection from $\alpha$ into $X$}$
    \end{center}

    Consider $X \times X$, so $2^{X \times X}$ is the set of relations on $X$ and there 
    are well-orderings of subsets of $X$ amongst all relations in $2^{X \times X}$, so consider the set
    \begin{center}
      $Y = \{ R \subseteq Y \times Y \: | \: Y \subseteq X \: \& \: \text{$Y$ is a well-ordering }\}$
    \end{center}

    So there is a set of ordinals:
    \begin{center}
      $Ord(Y) = \{ \alpha \in \operatorname{Ord} \: | \: \exists R \in Y \: \text{$\alpha$ is the order type of $Y$}  \}$
    \end{center}
    Note that $Ord(Y)$ is a set and take the least element ordinal $\beta$ does not belong to $Ord(Y)$. So $h(X) = \beta$.
    To be more precise, we have:
    \begin{center}
      $\beta = \sup Ord(Y)$
    \end{center}

    Then $|\alpha| < h(\alpha)$ for each ordinal $\alpha$.

    \item Let $\alpha = \sup X$. Let $f$ be a one-to-one function from $\alpha$ onto some $\beta < \alpha$.
    Let $\kappa$ be a cardinal such that $\beta < \kappa \leq \alpha$, then $|\kappa| = |\{ f(\xi) \: | \: \xi < \kappa \}| \leq \beta$, 
    so contradiction and $\alpha$ is a cardinal.
  \end{enumerate}
\end{proof}

The enumeration of all alephs is defined by transfinite induction:
\begin{itemize}
  \item $\aleph_0 = \omega$
  \item $\aleph_{\alpha + 1} =\aleph_{\alpha}^+ = \omega_{\alpha+1}$
  \item If $\beta$ is a limit ordinal, then $\aleph_{\beta} = \omega_{\beta} = \sup \{ \omega_{\alpha} \: | \: \alpha < \beta \}$.
\end{itemize}

A cardinal of the form $\aleph_{\alpha + 1}$ is a \emph{successor} cardinal, 
a cardinal $\aleph_{\beta}$ for limit $\beta$ is a \emph{limit cardinal}.

\subsection{The ordering of $\alpha \times \alpha$}

Define a well-ordering of the class $\operatorname{Ord} \times \operatorname{Ord}$ the following way:

\begin{center}
  $(\alpha, \beta) < (\gamma, \delta)$ iff either $\max(\alpha, \beta) < \max(\gamma, \delta)$ or \\ $\max(\alpha, \beta) = \max(\gamma, \delta)$ and $\alpha < \gamma$ or \\ $\max(\alpha, \beta) = \max(\gamma, \delta)$ and $\alpha = \gamma$ and $\beta < \delta$.
\end{center}

Then $<$ is a well-ordering and linear relation on $\operatorname{Ord}$. 
Moreover, $\alpha \times \alpha$ is the initial segment of $(\operatorname{Ord} \times \operatorname{Ord}, <)$ given by $(0, \alpha)$.

We let:

\begin{center}
  $\Gamma(\alpha, \beta) = \text{the order type of } \{ (\xi, \eta) \: | \: (\xi, \eta) < (\alpha, \beta) \}$
\end{center}

$\Gamma$ is also one-to-one:
\begin{center}
  $(\alpha, \beta) < (\gamma, \delta)$ iff $\Gamma(\alpha, \beta) < \Gamma(\gamma, \delta)$
\end{center}

$\Gamma$ is increasing and continuous and $\Gamma(\alpha \times \alpha) = \alpha$ for arbitrarily large $\alpha$.

\begin{theorem}
  $\aleph_{\alpha} \cdot \aleph_{\alpha} = \aleph_{\alpha}$
\end{theorem}

\begin{proof}
  Let us show that $\Gamma(\omega_{\alpha} \times \omega_{\alpha}) = \omega_{\alpha}$. 
  \begin{enumerate}
    \item If $\alpha = 0$, then $\Gamma(\omega \times \omega) = \omega$.
    \item Let $\alpha$ be the least ordinal such that $\Gamma(\omega_{\alpha} \times \omega_{\alpha}) \neq \omega_{\alpha}$.
    Let $\beta, \gamma$ be ordinals such that $\Gamma(\beta, \gamma) = \omega_{\alpha}$.
    Take $\delta < \omega_{\alpha}$ such that $\delta > \beta, \gamma$. 
    $\delta \times \delta$ is the initial segment of $\operatorname{Ord}^2$ and it contains $(\beta, \gamma)$. 
    So $\Gamma(\delta \times \delta) \supset \omega_{\alpha} = \Gamma(\beta, \gamma)$.
    Thus $|\delta \times \delta| \geq \aleph_{\alpha}$. 
    But $|\delta \times \delta| = |\delta| \cdot |\delta| = |\delta|$. But $|\delta| < \aleph_{\alpha}$ by the assumption 
    of minimality of $\alpha$. Contradiction.
  \end{enumerate}
\end{proof}

As a corollary:
\begin{center}
  $\aleph_{\alpha} + \aleph_{\beta} = \aleph_{\alpha} \cdot \aleph_{\beta} = \max(\aleph_{\alpha}, \aleph_{\beta})$
\end{center}

\subsection{Cofinality}

Let $\alpha, \beta > 0$ be limit ordinals. An increasing $\beta$-sequence 
$\langle \alpha_{\xi} : \xi < \beta \rangle$ is \emph{cofinal} in $\alpha$ if $\lim_{\xi \to \beta} \alpha_{\xi} = \alpha$.
A subset $X \subseteq \alpha$ is \emph{cofinal} in $\alpha$ whenever $\sup X = \alpha$.

Let $\alpha > 0$ be a limit ordinal, the \emph{cofinality} of $\alpha$ is:
\begin{center}
  $\operatorname{cf} \alpha = \text{the least ordinal $\beta$ such that $\exists$ 
  $\langle \alpha_{\xi} : \xi < \beta \rangle$ such that $\lim_{\xi \to \beta} \alpha_{\xi} = \alpha$}$
\end{center}

  Note that for each $\alpha$ $\operatorname{cf} \alpha$ is a limit ordinal and $\operatorname{cf} \alpha \leq \alpha$.

\begin{lemma}
  For each $\alpha$ $\operatorname{cf} (\operatorname{cf} \alpha) \leq \operatorname{cf} \alpha$.
\end{lemma}

\begin{proof}
  Let $\langle \alpha_{\xi} : \xi < \beta \rangle$ be cofinal in $\alpha$ and let $\langle \xi_{\nu} : \nu < \gamma \rangle$ be cofinal in $\beta$.

  Consider $\langle \alpha_{\xi_{\nu}} : \nu < \gamma \rangle$, then
  \begin{center}
    $\lim \limits_{\nu < \gamma} \alpha_{\xi_{\nu}} = \alpha$
  \end{center}
  since the limit of a subsequence equals the limit of a sequence as in usual real analysis or topology.
\end{proof}

\begin{lemma}~\label{confin1}
  Let $\alpha$ be a non-zero limit ordinal, then

  \begin{enumerate}
    \item~\label{confin1.1} If $A \subseteq \alpha$ and $\sup A = \alpha$, the order-type of $A$ is at least $\operatorname{cf} \alpha$.
    \item Let $\beta_0 \leq \beta_1 \leq \dots \leq \beta_{\xi} \leq \dots $ for $\xi < \gamma$ 
    be a non-decreasing sequence of ordinals such that $\lim_{\xi \to \gamma} = \alpha$, then $\operatorname{cf} \gamma = \alpha$.
  \end{enumerate}
\end{lemma}

\begin{proof}

  \begin{enumerate}
    \item The order-type of $A$ is the length of the increasing enumeration of $A$, 
    the limit of which (as an increasing sequence) is $\alpha$.
    \item If $\gamma = \lim_{\nu \to \operatorname{cf} \gamma} \xi_{\nu}$,
    then $\alpha = \lim_{\nu \to \operatorname{cf} \gamma} \beta_{\xi_{\nu}}$,
    and the non-decreasing sequence $\langle \beta_{\xi_{\nu}} : \nu < \operatorname{cf} \gamma\rangle$
    has an increasing sequence of the length at most $\operatorname{cf} \gamma$ and it has the same limit, 
    so $\operatorname{cf} \alpha \leq \operatorname{cf} \gamma$.

    To show $\operatorname{cf} \gamma \leq \operatorname{cf} \alpha$, 
    assume $\alpha = \lim_{\nu \to \operatorname{cf} \alpha} \alpha_{\nu}$.
    Take $\nu < \operatorname{cf} \alpha$, 
    let $\xi_{\nu}$ be the least $\xi$ greater than all $\xi_{\iota}$ for $\iota < \nu$
    such that $\beta_{\xi} > \alpha_{\nu}$.
    We have $\alpha = \lim_{\nu \to \operatorname{cf} \alpha} \beta_{\xi_{\nu}}$, so
    $\gamma = \lim_{\nu \to \operatorname{cf} \alpha} \xi_{\nu}$, so the inequation is proved.
  \end{enumerate}
\end{proof}

An infinite cardinal $\aleph_{\alpha}$ is \emph{regular} if $\operatorname{cf} \omega_{\alpha} = \omega_{\alpha}$.
$\aleph_{\alpha}$ is \emph{singular} if $\operatorname{cf} \omega_{\alpha} < \omega_{\alpha}$.

\begin{lemma}
  Let $\alpha$ be a limit ordinal, then $\operatorname{cf} \alpha$ is a regular cardinal.
\end{lemma}

\begin{proof}
  If $\alpha$ is not a cardinal, then there exists an ordinal $\beta < \alpha$ such that 
  $|\beta| = |\alpha|$, 
  then we construct a cofinal sequence in $\alpha$ of length $|\beta|$, then $\operatorname{cf} \alpha = |\beta|$ 
  and $\operatorname{cf} \alpha < \alpha$.
\end{proof}

Let $\kappa$ be a limit ordinal, a subset $X \subset \kappa$ is \emph{bounded} if $\sup X < \kappa$ and 
\emph{unbounded} if $\sup X = \kappa$.

\begin{lemma}~\label{boundedaleph} Let $\kappa$ be an aleph, then:

  \begin{enumerate}
    \item If $X \subset \kappa$ and $|X| < \operatorname{cf} \kappa$, then $X$ is bounded.
    \item If $\lambda$ < $\operatorname{cf} \kappa$ and $f : \lambda \to \kappa$, 
    then $\operatorname{Im}f$ is bounded in $\kappa$.
  \end{enumerate}
\end{lemma}

\begin{proof}

  \begin{enumerate}
    \item Let $X$ be such subset of $\kappa$ and assume $X$ is unbounded, so $\sup X = \kappa$.
    By~\ref{confin1.1} of Lemma~\ref{confin1}, the order-type of $X$ is at least $\operatorname{cf} \kappa$, which contradicts to 
    $|X| < \operatorname{cf} \kappa$, so $X$ is bounded.
    \item  Follows from the first item.
  \end{enumerate}
\end{proof}

\begin{lemma}~\label{reg} {\bf (Hausdorff)}

  Let $\kappa$ be a cardinal, then the following are equivalent:

  \begin{enumerate}
    \item $\kappa$ is singular.
    \item There is a cardinal $\lambda < \kappa$ and a family $\{ S_{\xi} \: | \: \xi < \lambda \}$ such that
    each $S_{\xi} \subset \kappa$, $|S_{\xi}| < \kappa$ and $\kappa = \bigcup \limits_{\xi < \lambda} S_{\xi}$.
  \end{enumerate}

\end{lemma}

\begin{proof}
  $ $

  \begin{enumerate}
    \item (1) $\Rightarrow$ (2).

    If $\kappa$ is singular, then there is an increasing sequence 
    $\langle \alpha_{\xi} \: : \: \xi < \operatorname{cf} \kappa \rangle$, so a family of required subsets is
    actually a family of those $\alpha_{\xi}$'s and $\lambda = \operatorname{cf} \kappa$ which is strictly less than $\kappa$
    since $\kappa$ is singular.
    \item (2) $\Rightarrow$ (1).
    
    Let $\lambda$ be the least cardinal such that $\lambda < \kappa$ and there exists a family
    $\{ S_{\xi} \: | \: \xi < \lambda \}$ where each $S_{\xi} \subset \kappa$, $|S_{\xi}| < \kappa$ and
    \begin{center}
      $\kappa = \bigcup \limits_{\xi < \lambda} S_{\xi}$
    \end{center}

    For each $\xi < \lambda$, let $\beta_{\xi}$ be the order-type of 
    $\cup_{\nu < \xi} S_{\nu}$. The sequence $\langle \beta_{\xi} : \xi < \lambda \rangle$ is non-decreasing and 
    each $\beta_{\xi} < \kappa$ for all $\xi < \lambda$ since $\lambda$ is minimal.

    Let us show that $\lim \limits_{\xi \to \kappa} \beta_{\xi} = \kappa$ to show that 
    $\operatorname{cf} \kappa \leq \lambda$.

    Assume $\beta = \lim \limits_{\xi \to \kappa} \beta_{\xi}$. There is a one-to-one mapping 
    $f : \bigcup \limits_{\xi < \beta} S_{\xi} \to \lambda \times \beta$ such that:
    \begin{center}
      $f : \alpha \mapsto (\xi, \gamma)$
    \end{center}
    where $\xi$ is the least ordinal such that $\alpha \in S_{\xi}$ and $\gamma$ is the order-type of $S_{\xi} \cap \gamma$.

    We have $\lambda < \kappa$ and $|\lambda \times \beta| = \lambda \cdot |\beta|$, then $\kappa = \beta$.
  \end{enumerate}
\end{proof}

\begin{theorem}~\label{expcofin}
  Let $\kappa$ be an infinite cardinal, then $\kappa < \kappa^{\operatorname{cf} \kappa}$.
\end{theorem}

\begin{proof}

  Let $F$ be a collection of $\kappa$ functions from $\operatorname{cf} \kappa$ to $\kappa$:
  \begin{center}
    $F = \{ f_{\alpha} : \operatorname{cf} \kappa \to \kappa \: | \: \alpha < \kappa \}$
  \end{center}

  Let us construct $f$ that does not belong to $F$.

  We have $\kappa = \lim_{\xi < \operatorname{cf} \kappa} \alpha_{\xi}$, for $\xi < \operatorname{cf} \kappa$ we let:
  \begin{center}
    $f(\xi) = \text{least $\gamma$ such that $\gamma \neq \forall \alpha < \alpha_{\xi} \: f_{\alpha} \neq \gamma$}$
  \end{center}

  Such $\gamma$ does exist and $f$ is different from all the $f_{\alpha}$.
\end{proof}

An uncountable cardinal $\kappa$ is \emph{weakly inaccessible} if it is limit and regular, but we cannot prove the
existence of weakly inaccessible cardinals in ZFC.

\section{Real Numbers and The Baire Space}

The \emph{continuum} is the cardinality of $\mathbb{R}$ denoted as $\mathfrak{c}$.

\begin{theorem} {\bf (Cantor)}

  $\aleph_0 < \mathfrak{c}$.
\end{theorem}

\begin{proof}
  One can think of it as a consequence of Theorem~\ref{expcofin}.
\end{proof}

\begin{definition} The \emph{Continuum Hypothesis} (CH) is the following statement:

  \begin{center}
    $\aleph_1 = \mathfrak{c}$.
  \end{center}
\end{definition}

Let $(P, <)$ be an ordered set, a subset $D \subset P$ is a \emph{dense} 
subset of $P$ if $a < b$ in $P$ implies $a < d$ and $d < b$ for some $d \in D$.

\begin{theorem}
  $(\mathbb{R}, <)$ is the unique complete linear ordering that has a 
  countable dense subset isomorphic to $(\mathbb{Q}, <)$.
\end{theorem}

\begin{proof}

  Let $C$ and $C'$ be two complete dense linear orderings and let 
  $P$ and $P'$ be dense in $C$ and $C'$ respectively. 
  Let $f : P \cong P'$, so $f$ can be extended to $f^* : C \cong C'$ by letting:

  \begin{center}
    $f^* : x \mapsto \sup \{ f(p) \: | \: p \in P \: \& \: p \leq x \}$
  \end{center}

  That is, ${(.)}^*$ is functorial.
\end{proof}

The existence of $(\mathbb{R}, <)$ follows from the following general statement:
\begin{theorem}
  Let $(P, <)$ be a dense unbounded linear ordering, then there exists a complete dense unbounded linear ordering 
  $(C, \prec)$ such that:

  \begin{enumerate}
    \item $(P, <)$ embeds to $(C, \prec)$.
    \item $P$ is dense in $C$.
  \end{enumerate}
\end{theorem}

\begin{proof}
  Recall that a \emph{Dedekind cut} in $P$ is a pair $(A, B)$ of disjoint subsets of $P$ such that:
  \begin{enumerate}
    \item $A \cup B = P$.
    \item $\forall a \in A \: \forall b \in B \: a < b$.
    \item $A$ has no greatest element.
  \end{enumerate}

  Let $C$ be the set of all Dedekind cuts in $P$. We let $(A_1, B_1) \preceq (A_2, B_2)$
  if $A_1 \subset A_2$ and $B_2 \subset B_1$. $(C, \preceq)$ is complete.

  Let $\{ C_i \: | \: i \in I \} \neq \emptyset$ be a bounded subset of $C$, then 
  $(\bigcup \limits_i A_i, \bigcap \limits_i B_i)$ is its supremum.

  Let $p \in P$, let
  \begin{center}
    $A_p = \{ x \in P \: | \: x < p \}$

    $B_p = \{ x \in P \: | \: x \geq p \}$
  \end{center}

  Then $(\{ (A_p, B_p) \: | \: p \in P \}, \preceq) \cong (P, <)$ and is dense in $C$.
\end{proof}

$\mathbb{Q}$ is dense in $\mathbb{R}$, so every open interval $(a, b)$ contains some rational number.
Then if $S$ is a disjoint collection of open intervals, then $S$ is at most countable.

Let $P$ be a dense linearly ordered set, if every disjoint collection of open intervals is at most countable, then we say
that $P$ satisfies the \emph{countable chain condition}.

\emph{{\bf (Suslin's Problem)} Let $P$ be a dense linearly ordered set satisfying the countable chain condition. Is $P$ isomorphic to $(\mathbb{R}, <)$?}

Note that neither Suslin's Problem nor its negation can be decided in ZFC.

\subsection{Topology of $\mathbb{R}$}

The real line is equipped with the natural topology induced by the metric $d(a, b) = |b - a|$ coincides with
the order topology on $(\mathbb{R}, <)$. $\mathbb{R}$ is also a complete separable metric space.

Every open set in $\mathbb{R}$ is the union of intervals with rational endpoints, so there are continuum many open sets 
(and the same observation holds for open sets as well).

A subset $P$ is \emph{perfect} is it has no isolated points.

\begin{theorem}
  Every perfect set $P$ has cardinality $\mathfrak{c}$.
\end{theorem}

\begin{proof}
  We construct a one-to-one function $F$ from $\{ 0, 1\}^{\omega}$ to $P$. 
  Let $S$ be the set of all finite binary sequences and let $s \in S$.
  
  By induction on $len(s)$ one can find closed intervals $I_s$ 
  such that for each $n < \omega$ and for each $s \in S$ such that $len(s) = n$:
  \begin{enumerate}
    \item $I_s \cap P$ is perfect,
    \item the diameter of $I_s$ is $\leq 1/2$,
    \item $I_{0:s}, I_{1:s} \subset I_s$ and $I_{0:s} \cap I_{1:s} = \emptyset$
  \end{enumerate}

  Take $f \in \{ 0, 1\}^{\omega}$, the set $P \cap \bigcap \limits_{n < \omega} I_{f \upharpoonright n}$ has exactly one element, so let:
  \begin{center}
    $F : f \mapsto \bigcap \limits_{n < \omega} I_{f \upharpoonright n}$
  \end{center}
\end{proof}

\begin{theorem}~\label{cantor-bendixon} ({\bf Cantor-Bendixon})

 If $F$ is an uncountable closed set, then $F = P \cup S$, where $P$ is perfect and $S$ is at most countable.

\end{theorem}

\begin{proof}

\item Let $F \subset \mathbb{R}$, let
\begin{center}
  $F^{'} = \text{the set of all limit points of $F$}$
\end{center}
$F^{'}$ is also called the \emph{derived set} of $F$. $F'$ is closed and obviously a subset of $A$. 

We let:
\begin{enumerate}
  \item $F_0 = A$.
  \item $F_{\alpha+1} = F^{'}_{\alpha}$.
  \item $F_{\alpha} = \bigcap \limits_{\gamma < \alpha} F_{\gamma}$ if $\alpha > 0$ is a limit ordinal.
\end{enumerate}

Since $F_0 \supset F_1 \supset \dots \supset F_{\alpha} \supset$, so we have an ordinal $\theta$ such that 
$F_{\theta} = F_{\theta + 1}$ (otherwise we could map the proper class of ordinals onto some set).
We let $P = F_{\alpha}$. If $P$ is nonempty, then $P$ is also perfect. 

Let us show that $F - P$ is at most countable. 
Let $\langle J_k : k < \omega \rangle$ be an enumeration of rational intervals. We have
\begin{center}
$F - P = \bigcup \limits_{\alpha < \theta} (F_{\alpha} - F_{\alpha + 1})$
\end{center}
So if $a \in F - P$, then there exists $\alpha < \theta$ such that $a \in F_{\alpha} - F_{\alpha + 1}$, that is,
$a$ is an isolated point of $F_{\alpha}$. We let $k_a$ be the least $k$ such that $a$ 
is the only point of $F_{\alpha}$ in $J_k$. 

If $\alpha \leq \beta$ and $a \neq b$ and $b$ is isolated in $F_{\beta}$, then $b \notin J_{k_a}$, so $k_a \neq k_b$, so
the mapping $a \mapsto k_a$ is one-to-one.

\end{proof}

\begin{col}
  If $C \subseteq \mathbb{R}$ is closed, then either $|C| = 2^{\aleph_0}$ or $|C| \leq \aleph_0$.
\end{col}

A set $A \subset \mathbb{R}$ is \emph{nowhere dense} if $\operatorname{Int}\operatorname{Cl} A = \emptyset$.
The following theorem shows that $\mathbb{R}$ is not of the \emph{first category}, that is,
$\mathbb{R}$ is not the union of a countable family of nowhere dense sets.

\begin{theorem} ({\bf The Baire Category Theorem})

  Let $\{ D_i \: | \: i < \omega \}$ be a countable family of dense open subsets of $\mathbb{R}$, then
  $D = \bigcap \limits_{i < \omega} D_i$ is dense in $\mathbb{R}$.
\end{theorem}

\begin{proof}
  We show that $D \cap I \neq \emptyset$ for each open interval $I$.

  Note that each finite intersection $D_0 \cap D_1 \cap \dots \cap D_n$ is dense and open for each $n < \omega$.
  Let $\langle J_k \: : \: k < \omega \rangle$ be an enumeration of rational intervals.
  
  Let $I_0 := I$ and for each $n$ $I_{n + 1} = J_k = (q_k, r_k)$ 
  where $k$ is the smallest index such that $[q_k, r_k] \subset I_n \cap D_n$.

  Take $a = \lim_{k \to \infty} q_k$, then $a \in I \cap D$.
\end{proof}

\subsection{The Baire Space}

The \emph{Baire Space} is the space $\mathcal{N} = \omega^{\omega}$ of infinite sequences of natural numbers
with the topology defined the following way.
Let $s$ be a finite sequence $s = \langle a_k \: : \: k < n \rangle$, we let:
\begin{center}
  $O(s) = \{ f \in \mathcal{N} \: | \: s \subset f \} = \{ \langle c_k \: | \: k < \omega \rangle \: | \: \forall k < n \: c_k = a_k \}$
\end{center}
All those $O(s)$'s form the open basis for $\mathcal{N}$.

The Baire space is separable and metrisable. The metric is defined as $d(f, g) = 1/2^{n + 1}$ 
where $n$ is the smallest natural number such that $f(n) \neq g(n)$. 
We also have separability since the set of all eventually constant sequences is dense in $\mathcal{N}$.

Every infinite sequence $\langle a_k \: : \: k < \omega \rangle$ defines a continued fraction $1/(a_0 + 1/(a_1 + 1/(a_2 + \dots)))$,
so we have a continuous bijection between infinite sequences and irrational points of the open interval $(0, 1)$. 
Moreover, the Baire space is homeomorphic to the space of irrational numbers.

Now we describe the characterisation of perfect sets in the Baire space.

Let $\operatorname{Seq}$ be the set of all finite sequences in $\mathcal{N}$. 
A \emph{tree} is a set $T \subset \operatorname{Seq}$ satisfying:
\begin{center}
  If $t \in T$ and there exists $n < \omega$ such that $s = t \upharpoonright n$, then $s \in T$.
\end{center}

Let $T$ be a tree, let $[T]$ be the set of all infinite paths through $T$:
\begin{center}
  $[T] = \{ f \in \mathbb{N} \: | \: \forall n < \omega \: f \upharpoonright n \in T \}$
\end{center}

For each $T$, the set $[T]$ is closed in the Baire space. Let $f \in \mathcal{N}$
such that $f \notin [T]$. Then there exists $n < \omega$ such that $s = f \upharpoonright n \notin T$, 
so the open neighbourhood of $f$ $O(s) = \{ g \in \mathcal{N} \: | \: g \supset s \}$. Thus $[T]$ is closed.

Conversely, let $F$ be closed in $\mathcal{N}$, then the set
\begin{center}
  $T_F = \{ s \in \operatorname{Seq} \: | \: \exists f \in F \: s \subset f \}$
\end{center}
is a tree and one can verify that $[T_F] = F$. If $f \in \mathcal{N}$ such that $f \upharpoonright n \in T$
for each $n < \omega$, then for each $n$ there is some $g \in F$ such that 
$g \upharpoonright n = f \upharpoonright n$, so $f \in F$ since $F$ is closed.

If $f$ is an isolated point of a closed set $F$ in $\mathcal{N}$, then there is $n \in \mathbb{N}$ such that no
$g \in F$ such that $g \neq f$ and $g \upharpoonright n = f \upharpoonright n$, so we have no branching starting from the
$n$-th position.

So we have the notion of a perfect set $P$ in the Baire space. A tree $T$ is \emph{perfect} if $t \in T$, then there
exist incomparable $t_1, t_2 \supset t$ such that both of them are in $T$ and neither $t_1 \subset t_2$ nor $t_2 \subset t_1$.

\begin{theorem}
  A closed set $F \subset \mathcal{N}$ is perfect iff the tree $T_F$ is perfect.
\end{theorem}

Let us discuss the Cantor-Bendixon analysis of closed subsets of the Baire space. Let $T$ be a tree, define:
\begin{center}
  $T^{'} = \{ t \in T \: | \: \exists t_1, t_2 \supset t \: (t_1, t_2 \in T \: \& \: \neg (t_1 \subset t_2 \lor t_2 \subset t_1))\}$
\end{center}

Then a set $T$ is perfect iff $T = T^{'} \neq \emptyset$.

$[T] - [T^{'}]$ is at most countable: take $f \in [T]$ such that $f \notin [T^{'}]$.
Take $s_f = f \upharpoonright n$ where $n < \omega$ is the smallest index such that $f \upharpoonright n \notin T'$.
If $f, g \in [T] - [T']$, then $s_f \neq s_g$ by the definition of $T'$, so the mapping $f \mapsto s_f$ is one-to-one.

Now let:
\begin{center}
  $T_0 = T$
  
  $T_{\alpha + 1} = T^{'}_{\alpha}$

  $T_{\alpha} = \bigcap \limits_{\beta < \alpha} T_{\beta}$ if $\alpha > 0$ is limit.
\end{center}

We have $T_0 \supset T_1 \supset \dots \supset T_{\alpha} \supset \dots$. $T_0$ is at most countable, so there is
$\theta < \omega_1$ at which the sequence stabilises. If $T_{\theta} \neq \emptyset$, then $T_{\theta}$ is perfect.

One can verify that:
\begin{center}
  $[\bigcap \limits_{\beta < \alpha} T_{\beta}] = \bigcap \limits_{\beta < \alpha} [T_{\beta}]$
\end{center}
so we have
\begin{center}
  $[T] - [T_{\theta}] = \bigcup \limits_{\beta < \alpha} ([T_{\alpha} - T^{'}_{\alpha}])$
\end{center}
and the set $[T] - [T_{\theta}]$ is at most countable. So we have a version of Theorem~\ref{cantor-bendixon} for the Baire space.

\section{The Axiom of Choice}

Recall that the axiom of choice (AC) says that if we have a family of sets $S$ such that $\emptyset \notin S$, then
we have a \emph{choice function} on $S$ such that $f(X) \in X$.

In some cases we can show the existence of a choice function without using the axiom of choice. 
For example, for families of a complete lattice, the choice function can return the supremum or infimum of each set
belonging to a family.

Using the axiom of choice one can also show that every infinite set has cardinality equal to $\aleph_{\alpha}$ for some $\alpha$.

\begin{theorem}{({\bf Zermelo})}

  Every set can be well-ordered.
\end{theorem}

\begin{proof}
  Let $A$ be a set. It is sufficient to construct a transfinite sequence 
  $\langle a_{\alpha} \: : \: \alpha < \theta \rangle$ that enumerates $A$. We do that by induction and by using
  the choice function $f$ on non-empty subsets of $A$. For $\alpha$ we let:
  \begin{center}
    $a_{\alpha} = f (A - \{ a_{\xi} \: | \: \xi < \alpha \})$
  \end{center}
  whenever $A - \{ a_{\xi} \: | \: \xi < \alpha \}$ is non-empty. 
  Let $\theta$ be the smallest ordinal such that $A = \{ a_{\alpha} \: | \: \alpha < \theta \}$.
  Thus $\langle a_{\alpha} \: : \: \alpha < \theta \rangle$ enumerates $A$.
\end{proof}

As it is well-known, Zermelo's theorem implies the axiom of choice.
Let $S$ be a family of sets such that $\emptyset \notin S$.
By Zermelo's theorem, we can well-order $\cup S$, so let $f(X)$ be the smallest element of $X$.

Note that Zermelo's theorem also implies that $\mathbb{R}$ can be well ordered and also that $2^{\aleph_0}$ is an aleph
and $2^{\aleph_0} \geq \aleph_1$.

Another important consequence of the axiom of choice:
\begin{theorem}
  The union of a countable family of countable sets is countable.
\end{theorem}

\begin{proof}~\label{countableunion}
  Let $A_n$ be a countable set for each $n < \omega$. For each $n$ let us choose an enumeration 
  $\langle a_{n, k} \: : \: k < \omega \rangle$ of $A_n$. So we have a projection of $\mathbb{N} \times \mathbb{N}$ onto
  $\bigcup \limits_{n < \omega} A_n$ by mapping $(n, k) \mapsto a_{n,k}$.
\end{proof}

In fact, the theorem above can be generalised the following way:
\begin{theorem}~\label{union}
  $|S| \leq S \cdot \sup \{ |X| \: | \: X \in S \}$.
\end{theorem}

\begin{proof}
  Let $\kappa = |S|$ and $\lambda = \sup \{ |X| \: | \: X \in S \}$.
  We have $S = \{ X_{\alpha} \: | \: \alpha < \kappa \}$ and for each $\alpha < \kappa$ we choose an 
  enumeration $X_{\alpha} = \{ a_{\alpha, \beta} \: | \: \beta < \lambda_{\alpha} \}$ where $\lambda_{\alpha} = |X_{\alpha}|$.
  Clearly that $\lambda_{\alpha} \leq \lambda$ for each $\alpha < \kappa$.
  So we have a projection of $\kappa \times \lambda$ onto $\cup S$ by mapping $(\alpha, \beta) \mapsto a_{\alpha, \beta}$.
\end{proof}

\begin{col}
  For every $\alpha$ $\aleph_{\alpha + 1}$ is a regular cardinal.
\end{col}

\begin{proof}
  If $\aleph_{\alpha + 1}$ were singular for some $\alpha$, then 
  $\omega_{\alpha + 1}$ would be the union of at most $\aleph_{\alpha}$ 
  sets of cardinality $\aleph_{\alpha}$ by Lemma~\ref{reg}, which
  would mean that $\aleph_{\alpha + 1} = \aleph_{\alpha}$ by Theorem~\ref{union}.
  Contradiction.
\end{proof}

Let $(P, <)$ be a poset, an element $a \in P$ is \emph{maximal} if no $b \in P$ such that $b > a$.
Let $X$ be a non-empty subset of $P$, then $c$ is the \emph{upper bound} of $X$ if $c \geq X$. 
$X$ is a \emph{chain} in $P$ if any two elements of $X$ are comparable.

\begin{theorem} {\bf (Zorn)}

  Let $(P, <)$ be a poset such that every chain $C$ has an upper bound, then $P$ has a maximal element.
\end{theorem}

\begin{proof}
  Let $f$ be a choice function on non-empty subsets of $P$. We construct a chain $C$ leading to a maximal element.

  Construct the following elements by induction:
  \begin{center}
    $a_{\alpha} = \text{an element of $P$ such that $a_{\alpha} > a_{\xi}$ for every $\xi > \alpha$ if it exists}$
  \end{center}

  If $\alpha > 0$ is a limit ordinal, then $C_{\alpha}$ is a chain in $P$ and $a_{\alpha}$ does exist.
  Eventually, there is $\theta$ such that no $a_{\theta+1} > a_{\theta}$. Thus $a_{\theta}$ is maximal.
\end{proof}

As it is known, Zorn's lemma implies the axiom of choice.
Let $S$ be a family of non-empty sets, then we check that the set
$\{ f \: | \: f \text{is a choice function on some $S' \subset S$}\}$ ordered by inclusion
satisfies the condition of Zorn's lemma, so a maximal element of that poset is a choice function on $S$.


There is a weaker version of the axiom of choice for countable families of non-empty sets. 
The countable AC implies Theorem~\ref{countableunion} and regularity of $\aleph_1$, but the
countable AC is too weak to show that $\mathbb{R}$ can be well-ordered.

There is a stronger version of the countable AC.
\begin{definition} {\bf (The Principle of Dependent Choice (DC))}

  Let $R$ be a binary relation on $A$ such that for all $x \in A$ there exists $y \in A$ such that $y R x$, then there is
  a sequence $a_0, a_1, \dots, a_n, \dots$ for $n < \omega$ such that:
  \begin{center}
    $\forall n < \omega \: (a_{n + 1} R a_n)$
  \end{center}
\end{definition}

The Principle of Dependent Choices allows characterising well orderings and (as well as well-founded relations) the following way:

\begin{lemma}
  Let $(A, <)$ be a poset, then the following are equivalent:

  \begin{enumerate}
    \item $(A, <)$ is a well-ordering.
    \item No infinite sequences $a_0, a_1, \dots, a_n, \dots$ for $n < omega$ such that:
    \begin{center}
      $a_0 > a_1 > \dots > a_n > \dots$
    \end{center}
  \end{enumerate}
\end{lemma}

\subsection{Cardinal Arithmetic the Generalised Continuum Hypothesis}

We shall be discussing the cardinal exponentiation operator for the rest of the section.

\begin{lemma}~\label{card1}
  Let $\lambda$ be infinite and $2 \leq \kappa \leq \lambda$, then $\kappa^{\lambda} = 2^{\lambda}$.
\end{lemma}

\begin{proof}
  $2^{\lambda} \leq \kappa^{\lambda} \leq (2^{\kappa})^{\lambda} = 2^{\kappa \cdot \lambda} = 2^{\lambda}$.
\end{proof}

The evaluation of $\kappa^{\lambda}$ is more complicated when $\lambda < \kappa$.
If $2^{\lambda} \geq \kappa$, then we have $\kappa^{\lambda} = 2^{\lambda}$ since $\kappa \leq (2^{\lambda})^{\lambda} = 2^{\lambda}$.
But if $2^{\lambda} < \kappa$, the only thing we can conclude:
\begin{center}
  $\kappa \leq \kappa^{\lambda} \leq 2^{\kappa}$
\end{center}
which is already known by Cantor's theorem.

Let $\lambda$ be a cardinal and let $A$ be a set such that $|A| \geq \lambda$, we let:
\begin{center}
  $[A]^{\lambda} = \{ X \in 2^A \: | \: |X| = \lambda \}$
\end{center}

\begin{lemma}~\label{cardfact1}
  If $|A| = \kappa \geq \lambda$, then the set $[A]^{\lambda}$ has cardinality $\kappa^{\lambda}$.
\end{lemma}

\begin{proof}
  On the one hand every function $f : \lambda \to A$ is a subset of $\lambda \times A$ and $|f| = \lambda$. Thus:
  \begin{center}
    $\kappa^{\lambda} \leq |[\lambda \times A]^{\lambda}| = |[A]^{\lambda}|$
  \end{center}
  On the other hand, there is a one-to-one function $F : [A]^{\lambda} \to A^{\lambda}$. 
  If $X \in [A]^{\lambda}$, let $F(X)$ be some function $f$ on $\lambda$ whose range is $X$.
\end{proof}

Let $\lambda$ be a limit cardinal, let:
\begin{center}
  $\kappa^{<\lambda} = \sup \{\kappa^{\mu} \: | \: \text{$\mu$ is a cardinal such that $\mu < \lambda$}\}$
\end{center}
We also define $\kappa^{<\lambda^{+}}$ for successors $\lambda^{+}$.

Let $\kappa$ be an infinite cardinal and $|A| \geq \kappa$, let:
\begin{center}
  $[A]^{<\kappa} = \{ X \in 2^{A} \: | \: |X| < \kappa \}$
\end{center}
Clearly, the cardinality of $[A]^{<\kappa}$ is $|A|^{<\kappa}$.

\subsection{Infinite Sums and Products}

Let $\{ \kappa_i \: | \: i \in I \}$ be an indexed family of cardinals, define:
\begin{center}
  $\sum \limits_{i \in I} \kappa_i = |\bigcup \limits_{i \in I} X_i|$
\end{center}
where each for $i \in I$ $|X_i| = \kappa_i$. Note that, by the Axiom of Choice, the definition of sum does not depend
on the choice of $\{ X_i \: | \: i \in I \}$.

Let $\lambda, \kappa$ be cardinals and let $\kappa_i = \kappa$, then:
\begin{center}
  $\sum \limits_{i < \lambda} \kappa_i = \lambda \cdot \kappa$
\end{center}

More generally, we have:
\begin{lemma}~\label{cardfact2}
  Let $\lambda$ be an infinite cardinal and $\kappa_i > 0$ for each $i < \lambda$, then:
  \begin{center}
    $\sum \limits_{i < \lambda} \kappa_i = \lambda \cdot \sup \limits_{i < \lambda} \kappa_i$
  \end{center}
\end{lemma}
\begin{proof}
  Let $\kappa = \sup \limits_{i < \lambda} \kappa_i$ and $\sigma = \sum \limits_{i < \lambda} \kappa_i$.
  On the one hand, we have $\forall i < \lambda \:\: \kappa_i \leq \kappa$, so
  \begin{center}
    $\sum \limits_{i < \lambda} \kappa_i \leq \lambda \cdot \kappa$
  \end{center}

  On the other hand, since $\kappa_1 \geq 1$ for each $i$, we have
  \begin{center}
  $\lambda = \sum \limits_{i < \lambda} 1 \leq \sigma$
  \end{center}
  $\sigma \geq \kappa_i$ for each $i$, so we have
  \begin{center}
    $\sigma \geq \sup \limits_{i < \lambda} \kappa_i = \kappa$
  \end{center}
  So $\sigma \geq \lambda \cdot \kappa$.
\end{proof}

Let $\{ X_i \: | \: i \in I \}$ be an indexed family of sets, we let:
\begin{center}
  $\prod \limits_{i \in I} X_i = \{ f \: | \: \text{$f$ is a function on $I$ such that $\forall i \in I \:\: f(i) \in X_i$} \}$
\end{center}
If each of $X_i$'s is non-empty, then the whole product is non-empty and this is equivalent to the axiom of choice.

Let $\{ \kappa_i \: | \: i \in I \}$ be a family of cardinals, define:
\begin{center}
  $\prod \limits_{i \in I} \kappa_i = |\prod \limits_{i \in I} X_i|$
\end{center}
where for each $i$ $X_i$ is a set of cardinality of $\kappa_i$. 
As in the case of sum, assuming the axiom of choice, one can show that the definition of product
does not depend on the choice of $X_i$'s.

If $\kappa_i = \kappa$ for each $i \in I$ and $I$ has cardinality $\lambda$, then:
\begin{center}
  $\prod \limits_{i \in I} \kappa_i = \lambda$
\end{center}

If $I$ is a disjoint union $I = \bigcup \limits_{j \in J} A_j$, then:
\begin{center}
  $\prod \limits_{i \in I} \kappa_i = \prod \limits_{j \in J} (\prod \limits_{i \in A_j} \kappa_i)$
\end{center}

If $\kappa_i \geq 2$ for each $i \in I$, then:
\begin{center}
  $\sum \limits_{i \in I} \kappa_i \leq \prod \limits_{i \in I} \kappa_i$
\end{center}
If $I$ is finite, then the inequality is self-evident. Assume $I$ is infinite. We have:
\begin{center}
  $\prod \limits_{i \in I} \kappa_i \geq \prod \limits_{i \in I} 2 = 2^{|I|} > |I|$
\end{center}
We show that $\sum_i \kappa_i \leq |I| \cdot \prod_i \kappa_i$.

Let $\{ X_i \: | \: i \in I \}$ be a disjoint family such that for each $i \in I$ $|X_i| = \kappa_i$.
Assign each $x \in \bigcup_i X_i$ to a pair $(i, f)$ such that $x \in X_i$ and $f \in \prod_i X_i$
such that $f(i) = x$.

\begin{lemma}
  Let $\lambda$ be an infinite cardinal and let $\langle \kappa_i \: : \: i < \lambda \rangle$ 
  be a non-descreasing sequence of ordinals, then
  \begin{center}
    $\prod \limits_{i \in I} \kappa_i = (\sup \limits_{i \in I} \kappa_i)^{\lambda}$
  \end{center}
\end{lemma}
\begin{proof}
  Let $\kappa = \sup_i \kappa_i$. Since $\kappa_i \leq \kappa$ for each $i < \lambda$, we have:
  \begin{center}
    $\prod \limits_{i \in I} \kappa_i \leq \prod \limits_{i \in I} \kappa = \kappa^{\lambda}$
  \end{center}
  Let us show $\kappa^{\lambda} \leq \prod \limits_{i \in I} \kappa_i$.

  Consider a partition of $\lambda$ into $\lambda$ disjoint sets $A_j$, each of which has cardinality $\lambda$:
  \begin{center}
    $\lambda = \bigcup \limits_{j < \lambda} A_j$
  \end{center}

  For each $j < \lambda$ we have:
  \begin{center}
    $\kappa = \sup \limits_{i \in A_j} \kappa_i \leq \prod \limits_{i \in A_j} \kappa_i$
  \end{center}
  And thus:
  \begin{center}
    $\prod \limits_{i \in I} \kappa_i = \prod \limits_{j < \lambda} (\prod \limits_{i \in A_j} \kappa_i) \geq \prod \limits_{j < \lambda} \kappa = \kappa^{\lambda}$
  \end{center}
\end{proof}

\begin{theorem} {\bf (K\"{o}nig)}

  Assume $\kappa_i < \lambda_i$ for each $i \in I$, then:
  \begin{center}
    $\sum \limits_{i \in I} \kappa_i < \prod \limits_{i \in I} \lambda_i$
  \end{center}
\end{theorem}

\begin{proof}
  Let us show $\Sigma_i \kappa_i \ngeq \Pi_i \lambda_i$. 
  Let $\{ T_i \: | \: i \in I\}$ be an indexed family such that $|T_i| = \lambda_i$.
  It sufficies to show that if we have a family $\{ Z_i \: | \: i \in I\}$ of subsets of $T = \Pi_i T_i$
  such that $|Z_i| < \kappa_i$ for each $i$, then $\cup_i Z_i \neq T$.

  For every $i \in I$, let $S_i$ be the projection of $Z_i$ into the $i$-th coordinate:
  \begin{center}
    $S_i = \{ f(i) \: | \: f \in Z_i \}$
  \end{center}

  As far as $|Z_i| < |T_i|$, we have $S_i \subset T_i$ and $S_i \neq T_i$ for each $i \in I$.
  Let $f \in T$ be a function such that $f(i) \notin S_i$. $f$ does not belong to any $Z_i$, so $\cup_i Z_i \neq T$.
\end{proof}

\begin{col}
  $\kappa < 2^{\kappa}$
\end{col}

\begin{proof}
  $\sum \limits_{i < \kappa} 1 < \prod \limits_{i < \kappa} 2$.
\end{proof}

\begin{col}~\label{cofinality:aleph}
  For each $\alpha$ $\aleph_{\alpha} < \operatorname{cf}(2^{\aleph_{\alpha}})$.
\end{col}

\begin{proof}
  Let us show that if for each $i < \omega_{\alpha}$ $\kappa_i < 2^{\aleph_{\alpha}}$,
  then $\Sigma_{i < \omega_{\alpha}} \kappa_i < 2^{\aleph_{\alpha}}$.
  Let $\lambda_i = 2^{\aleph_{\alpha}}$.
  \begin{center}
    $\sum \limits_{i < \omega_{\alpha}} \kappa_i < \prod \limits_{i < \omega_{\alpha}} \lambda_i = (2^{\aleph_{\alpha}})^{\aleph_{\alpha}} = 2^{\aleph_{\alpha}}$
  \end{center}
\end{proof}

\begin{col}
  For all $\alpha, \beta$ $\aleph_{\beta} < \operatorname{cf}(\aleph_{\alpha}^{\aleph_{\beta}})$.
\end{col}

\begin{proof}
  We show that if $\kappa_i < \aleph_{\alpha}^{\aleph_{\beta}}$ for each 
  $i < \omega_{\beta}$, then $\Sigma_{i < \omega_{\beta}} \kappa_i < \aleph_{\alpha}^{\aleph_{\beta}}$.
  Let $\lambda_i = \aleph_{\alpha}^{\aleph_{\beta}}$, then
  \begin{center}
    $\sum \limits_{i < \omega_{\beta}} \kappa_i < \prod \limits_{i < \omega_{\beta}} = \aleph_{\alpha}^{\aleph_{\beta}}$
  \end{center}
\end{proof}

\begin{col}
  Let $\kappa$ be an infinite cardinal, then $\kappa < \kappa^{\operatorname{cf} \kappa}$
\end{col}

\begin{proof}
  Let $i < \operatorname{cf} \kappa$ and $\kappa_i < \kappa$ be such that $\kappa = \Sigma_{i < \operatorname{cf} \kappa} \kappa_i$.
  \begin{center}
    $\kappa = \sum \limits_{i < \operatorname{cf} \kappa} \kappa_i < \prod \limits_{i < \operatorname{cf} \kappa} \kappa = \kappa^{\operatorname{cf} \kappa}$.
  \end{center}
\end{proof}

\subsection{The Continuum Function}

Cantor's theorem claims that $\aleph_{\alpha} < 2^{\aleph_{\alpha}}$, so $\aleph_{\alpha+1} \leq 2^{\aleph_{\alpha}}$
for each $\alpha$. The \emph{Generalised Continuum Hypothesis} (GCH) is the statement
\begin{center}
  $2^{\aleph_{\alpha}} = \aleph_{\alpha+1}$
\end{center}
for each $\alpha$. GCH is independent of ZFC, but ZFC + GCH proves the following properties of
cardinal exponentiation:
\begin{theorem} Assume GCH. Let $\kappa$ and $\lambda$ be infinite cardinals, then:

  \begin{enumerate}
    \item If $\kappa \leq \lambda$, then $\kappa^{\lambda} = \lambda^{+}$.
    \item If $\operatorname{cf} \kappa \leq \lambda < \kappa$, then $\kappa^{\lambda} = \kappa^{+}$.
    \item If $\lambda < \operatorname{cf} \kappa$, then $\kappa^{\lambda} = \kappa$.
  \end{enumerate}
\end{theorem}

\begin{proof}
$ $

\begin{enumerate}
  \item By Lemma~\ref{card1} we have $\kappa^{\lambda} = 2^{\lambda}$, but $2^{\lambda} = \lambda^{+}$.
  \item Combine Lemma~\ref{cardfact1} and Lemma~\ref{cardfact2}.
  \item By Lemma~\ref{boundedaleph} we have:
  \begin{center}
    $\kappa^{\lambda} = \{ \alpha^{\lambda} \: | \: \alpha < \kappa \}$
  \end{center}
  so:
  \begin{center}
    $|\alpha^{\lambda}| \leq 2^{|\alpha| \cdot \lambda} = (|\alpha| \cdot \lambda)^{+} \leq \kappa$
  \end{center}
\end{enumerate}
\end{proof}

The \emph{beth function} is defined by induction:
\begin{enumerate}
  \item $\beth_0 = \aleph_0$
  \item $\beta_{\alpha + 1} = 2^{\beta_{\alpha}}$
  \item $\beta_{\alpha} = \sup \{ \beta_{\beta} \: | \: \beta < \alpha \}$ if $\alpha$ is limit ordinal.
\end{enumerate}

So we can reword GCH as $\beta_{\alpha} = \aleph_{\alpha}$ for all $\alpha$.

Now we study the behaviour of the continuum function $\kappa \mapsto 2^{\kappa}$ assuming no GCH.

\begin{theorem}~\label{continuum:func} Let $\kappa, \lambda$ be cardinals, then

  \begin{enumerate}
    \item If $\kappa < \lambda$, then $2^{\kappa} \leq 2^{\lambda}$.
    \item $\kappa < \operatorname{cf} 2^{\kappa}$
    \item~\label{continuum:func:3} If $\kappa$ is a limit cardinal, then $2^{\kappa} = (2^{<\kappa})^{\operatorname{cf} \kappa}$
  \end{enumerate}
\end{theorem}

\begin{proof}
  
  $ $

  \begin{enumerate}
    \item Fairly obvious.
    \item Corollary~\ref{cofinality:aleph}.
    \item Let $\kappa = \Sigma_{i < \operatorname{cf} \kappa} \kappa_i$ where each $\kappa_i < \kappa$ for each $i$.
    We have
    \begin{center}
      $2^{\kappa} = 2^{\Sigma_{i < \operatorname{cf} \kappa} \kappa_i} = \prod \limits_{i < \operatorname{cf} \kappa} 2^{\kappa_i} \leq \prod \limits_{i < \operatorname{cf} \kappa} 2^{< \kappa} = (2^{< \kappa})^{\operatorname{cf} \kappa} \leq (2^{\kappa})^{\operatorname{cf} \kappa} \leq 2^{\kappa}$
    \end{center}
  \end{enumerate}
\end{proof}

\begin{col}
  Let $\kappa$ be a singular cardinal. 
  Assume the continuum function is eventually constant below $\kappa$, with value $\lambda$, 
  then $2^{\kappa} = \lambda$.
\end{col}

\begin{proof}
  If $\kappa$ is singular and it satisfies the assumption of the statement, then
  there is $\nu$ such that $\operatorname{cf} \kappa \leq \nu < \kappa$ and that
  $2^{< \kappa} = \lambda = 2^{\nu}$. Thus:
  \begin{center}
    $2^{\kappa} = (2^{<\kappa})^{\operatorname{cf} \kappa} = 2^{\nu}$.
  \end{center}
\end{proof}

The \emph{gimel function} is the function:
\begin{center}
  $\gimel(\kappa) = \kappa^{\operatorname{cf} \kappa}$
\end{center}

If $\kappa$ is a limit cardinal and the continuum function below $\kappa$ is not eventually constant,
then the cardinal $\lambda = 2^{<\kappa}$ is a limit of a non-decreasing sequence:
\begin{center}
  $\lambda = 2^{<\kappa} = \lim \limits_{\alpha \to \kappa} 2^{|\alpha|}$
\end{center}
Then, by Lemma~\ref{confin1}, $\operatorname{cf} \lambda = \operatorname{cf} \kappa$.
Thus, by Theorem~\ref{continuum:func}(\ref{continuum:func:3}), we have:
\begin{center}
$2^{\kappa} = (2^{<\kappa})^{\operatorname{cf} \kappa} = 2^{\operatorname{cf} \lambda}$
\end{center}

If $\kappa$ is regular, then $\kappa = \operatorname{cf} \kappa$ and, since 
$\kappa^{\kappa} = 2^{\kappa}$ we have:
\begin{center}
  $2^{\kappa} = \kappa^{\operatorname{cf} \kappa}$
\end{center}

So we can specify the behaviour of the continuum function in terms of the gimel function.
\begin{col}
  $ $

  \begin{enumerate}
    \item If $\kappa$ is a successor cardinal, then $2^{\kappa} = \gimel(\kappa)$.
    \item If $\kappa$ is a limit cardinal and $\lambda x. 2^x$ below $\kappa$ is eventually constant, then
    $2^{\kappa} = 2^{<\kappa} \cdot \gimel(\kappa)$.
    \item If $\kappa$ is a limit cardinal and $\lambda x. 2^x$ below $\kappa$ is not eventually constant, then
    $2^{\kappa} = \gimel( 2^{<\kappa})$.
  \end{enumerate}
\end{col}

\subsection{Cardinal Exponentiation}

Let $\kappa$ be a regular cardinal and let $\lambda < \kappa$, then every function $f : \lambda \to \kappa$ is bounded,
i.e., $\sup \{ f(\xi) \: | \: \xi < \lambda \} < \kappa$. Thus:
\begin{center}
  $\kappa^{\lambda} = \bigcup \limits_{\alpha < \kappa} \alpha^{\lambda}$
\end{center}
that is,
\begin{center}
  $\kappa^{\lambda} = \sum \limits_{\alpha < \kappa} |\alpha|^{\lambda}$
\end{center}
If $\kappa$ is a successor cardinal, then we obtain the \emph{Hausdorff formula}:
\begin{center}
  $\aleph_{\alpha + 1}^{\beta} = \aleph_{\alpha}^{\aleph_{\beta}} \cdot \aleph_{\alpha + 1}$
\end{center}

We can compute $\kappa^{\lambda}$ using the following fact.
If $\kappa$ is a limit cardinal, we use use the notation:
\begin{center}
  $\lim \limits_{\alpha \to \kappa} \alpha^{\lambda} := \sup \{ \mu^{\lambda} \: | \: \text{$\mu$ is a cardinal and $\mu < \kappa$}\}$
\end{center}

\begin{lemma}
  Let $\kappa$ be a limit cardinal and assume that $\operatorname{cf} \kappa \leq \lambda$, then
  \begin{center}
    $\kappa^{\lambda} = (\lim \limits_{\alpha \to \kappa} \alpha^{\lambda})^{\operatorname{cf} \kappa}$
  \end{center}
\end{lemma}
\begin{proof}
  Let $\kappa = \Sigma_{i < \operatorname{cf} \kappa} \kappa_i$, where $\kappa_i < \kappa$ for each $i$.
  We have:
  \begin{center}
    $\kappa^{\lambda} \leq (\prod \limits_{i < \operatorname{cf} \kappa} \kappa_i)^{\lambda} = 
    \prod \limits_{i < \operatorname{cf} \kappa} \kappa_i^{\lambda} \leq 
    \prod \limits_{i < \operatorname{cf} \kappa} (\lim \limits_{\alpha \to \kappa} \alpha^{\lambda})^{\operatorname{cf} \kappa} \leq (\kappa^{\lambda})^{\operatorname{cf} \kappa} = \kappa^{\lambda}$
  \end{center}
\end{proof}

\begin{comment}
\begin{theorem}

  Let $\lambda$ be an infinite cardinal, then for all infinite cardinals $\kappa$, the value of $\kappa^{\lambda}$ is computed as follows:

  \begin{enumerate}
    \item $\kappa \leq \lambda$ implies $\kappa^{\lambda} = 2^{\lambda}$.
    \item 
    \item 
  \end{enumerate}
\end{theorem}
\end{comment}

\section{The Axiom of Regularity}

\bibliographystyle{alpha}
\bibliography{Text}


\end{document}