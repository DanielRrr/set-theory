\documentclass[8pt]{article}
\usepackage{graphicx} % Required for inserting images
\usepackage{amsthm}
\usepackage{amsmath}
\usepackage{amsfonts}
\usepackage{amsopn}
\usepackage{comment}
\usepackage{amssymb}
\usepackage{hyperref}
\usepackage{bussproofs}
\usepackage[all, 2cell]{xy}
\usepackage[all]{xy}
\usepackage{rotating}
\usepackage{lscape}
\usepackage{minted}


\theoremstyle{definition}
\newtheorem{definition}{Definition}[section]

\theoremstyle{definition}
\newtheorem{theorem}{Theorem}[section]

\theoremstyle{definition}
\newtheorem{claim}{Claim}[section]

\theoremstyle{definition}
\newtheorem{ex}{Example}[section] 

\theoremstyle{definition}
\newtheorem{cons}{Construction}[section] 

\theoremstyle{definition}
\newtheorem{rem}{Remark}[section] 


\theoremstyle{definition}
\newtheorem{prop}{Proposition}[section]

\theoremstyle{definition}
\newtheorem{lemma}{Lemma}[section]

\theoremstyle{definition}
\newtheorem{fact}{Fact}[section]

\theoremstyle{definition}
\newtheorem{remark}{Remark}[section]

\theoremstyle{definition}
\newtheorem{notation}{Notation}[section]

\theoremstyle{definition}
\newtheorem{example}{Example}[section]

\theoremstyle{col}
\newtheorem{col}{Corollary}

\theoremstyle{question}
\newtheorem{question}{Question}

\let\strokeL\L
\renewcommand\L{\mathbf{L}}

\title{Some Notes on Set Theory, Pt 1}
\author{Daniel Rogozin}
\date{ }

\begin{document}

\maketitle

\section{Cardinals}

An ordinal number $\alpha$ is a \emph{cardinal number} if no $\beta < \alpha$ such that $|\alpha| = |\beta|$.
Further, we shall use $\kappa, \lambda, \mu$ to denote cardinal numbers.

Let $W$ be a well-ordered set, then there exists an ordinal $\alpha$ such that $|W| = |\alpha|$, so we let:
\begin{center}
  $|W| = \text{the least ordinal $\alpha$ such that $|W| = |\alpha|$}$
\end{center}

An \emph{aleph} is an infinite cardinal number. 

Let $\alpha$ be an ordinal, then $\alpha^+$ is the least cardinal bigger than $\alpha$.

\begin{lemma}
  $ $

  \begin{enumerate}
    \item For every $\alpha$ there is a cardinal number $\kappa$ such that $\kappa > \alpha$.
    \item Let $X$ be a set of cardinal, then $\sup X$ is a cardinal.
  \end{enumerate}
\end{lemma}

\begin{proof}
  $ $

  \begin{enumerate}
    \item Let $X$ be a set, let 
    \begin{center}
      $h(X) = \text{the least $\alpha$ such that no injection from $\alpha$ into $X$}$
    \end{center}

    Consider $X \times X$, so $2^{X \times X}$ is the set of relations on $X$ and there 
    are well-orderings of subsets of $X$ amongst all relations in $2^{X \times X}$, so consider the set
    \begin{center}
      $Y = \{ R \subseteq Y \times Y \: | \: Y \subseteq X \: \& \: \text{$Y$ is a well-ordering }\}$
    \end{center}

    So there is a set of ordinals:
    \begin{center}
      $Ord(Y) = \{ \alpha \in \operatorname{Ord} \: | \: \exists R \in Y \: \text{$\alpha$ is the order type of $Y$}  \}$
    \end{center}
    Note that $Ord(Y)$ is a set and take the least element ordinal $\beta$ does not belong to $Ord(Y)$. So $h(X) = \beta$.
    To be more precise, we have:
    \begin{center}
      $\beta = \sup Ord(Y)$
    \end{center}

    Then $|\alpha| < h(\alpha)$ for each ordinal $\alpha$.

    \item Let $\alpha = \sup X$. Let $f$ be a one-to-one function from $\alpha$ onto some $\beta < \alpha$.
    Let $\kappa$ be a cardinal such that $\beta < \kappa \leq \alpha$, then $|\kappa| = |\{ f(\xi) \: | \: \xi < \kappa \}| \leq \beta$, 
    so contradiction and $\alpha$ is a cardinal.
  \end{enumerate}
\end{proof}

The enumeration of all alephs is defined by transfinite induction:
\begin{itemize}
  \item $\aleph_0 = \omega$
  \item $\aleph_{\alpha + 1} =\aleph_{\alpha}^+ = \omega_{\alpha+1}$
  \item If $\beta$ is a limit ordinal, then $\aleph_{\beta} = \omega_{\beta} = \sup \{ \omega_{\alpha} \: | \: \alpha < \beta \}$.
\end{itemize}

A cardinal of the form $\aleph_{\alpha + 1}$ is a \emph{successor} cardinal, 
a cardinal $\aleph_{\beta}$ for limit $\beta$ is a \emph{limit cardinal}.

\subsection{The ordering of $\alpha \times \alpha$}

Define a well-ordering of the class $\operatorname{Ord} \times \operatorname{Ord}$ the following way:

\begin{center}
  $(\alpha, \beta) < (\gamma, \delta)$ iff either $\max(\alpha, \beta) < \max(\gamma, \delta)$ or \\ $\max(\alpha, \beta) = \max(\gamma, \delta)$ and $\alpha < \gamma$ or \\ $\max(\alpha, \beta) = \max(\gamma, \delta)$ and $\alpha = \gamma$ and $\beta < \delta$.
\end{center}

Then $<$ is a well-ordering and linear relation on $\operatorname{Ord}$. 
Moreover, $\alpha \times \alpha$ is the initial segment of $(\operatorname{Ord} \times \operatorname{Ord}, <)$ given by $(0, \alpha)$.

We let:

\begin{center}
  $\Gamma(\alpha, \beta) = \text{the order type of } \{ (\xi, \eta) \: | \: (\xi, \eta) < (\alpha, \beta) \}$
\end{center}

$\Gamma$ is also one-to-one:
\begin{center}
  $(\alpha, \beta) < (\gamma, \delta)$ iff $\Gamma(\alpha, \beta) < \Gamma(\gamma, \delta)$
\end{center}

$\Gamma$ is increasing and continuous and $\Gamma(\alpha \times \alpha) = \alpha$ for arbitrarily large $\alpha$.

\begin{theorem}
  $\aleph_{\alpha} \cdot \aleph_{\alpha} = \aleph_{\alpha}$
\end{theorem}

\begin{proof}
  Let us show that $\Gamma(\omega_{\alpha} \times \omega_{\alpha}) = \omega_{\alpha}$. 
  \begin{enumerate}
    \item If $\alpha = 0$, then $\Gamma(\omega \times \omega) = \omega$.
    \item Let $\alpha$ be the least ordinal such that $\Gamma(\omega_{\alpha} \times \omega_{\alpha}) \neq \omega_{\alpha}$.
    Let $\beta, \gamma$ be ordinals such that $\Gamma(\beta, \gamma) = \omega_{\alpha}$.
    Take $\delta < \omega_{\alpha}$ such that $\delta > \beta, \gamma$. 
    $\delta \times \delta$ is the initial segment of $\operatorname{Ord}^2$ and it contains $(\beta, \gamma)$. 
    So $\Gamma(\delta \times \delta) \supset \omega_{\alpha} = \Gamma(\beta, \gamma)$.
    Thus $|\delta \times \delta| \geq \aleph_{\alpha}$. 
    But $|\delta \times \delta| = |\delta| \cdot |\delta| = |\delta|$. But $|\delta| < \aleph_{\alpha}$ by the assumption 
    of minimality of $\alpha$. Contradiction.
  \end{enumerate}
\end{proof}

As a corollary:
\begin{center}
  $\aleph_{\alpha} + \aleph_{\beta} = \aleph_{\alpha} \cdot \aleph_{\beta} = \max(\aleph_{\alpha}, \aleph_{\beta})$
\end{center}

\subsection{Cofinality}

Let $\alpha, \beta > 0$ be limit ordinals. An increasing $\beta$-sequence 
$\langle \alpha_{\xi} : \xi < \beta \rangle$ is \emph{cofinal} in $\alpha$ if $\lim_{\xi \to \beta} \alpha_{\xi} = \alpha$.
A subset $X \subseteq \alpha$ is \emph{cofinal} in $\alpha$ whenever $\sup X = \alpha$.

Let $\alpha > 0$ be a limit ordinal, the \emph{cofinality} of $\alpha$ is:
\begin{center}
  $\operatorname{cf} \alpha = \text{the least ordinal $\beta$ such that $\exists$ 
  $\langle \alpha_{\xi} : \xi < \beta \rangle$ such that $\lim_{\xi \to \beta} \alpha_{\xi} = \alpha$}$
\end{center}

  Note that for each $\alpha$ $\operatorname{cf} \alpha$ is a limit ordinal and $\operatorname{cf} \alpha \leq \alpha$.

\begin{lemma}
  For each $\alpha$ $\operatorname{cf} (\operatorname{cf} \alpha) \leq \operatorname{cf} \alpha$.
\end{lemma}

\begin{proof}
  Let $\langle \alpha_{\xi} : \xi < \beta \rangle$ be cofinal in $\alpha$ and let $\langle \xi_{\nu} : \nu < \gamma \rangle$ be cofinal in $\beta$.

  Consider $\langle \alpha_{\xi_{\nu}} : \nu < \gamma \rangle$, then
  \begin{center}
    $\lim \limits_{\nu < \gamma} \alpha_{\xi_{\nu}} = \alpha$
  \end{center}
  since the limit of a subsequence equals the limit of a sequence as in usual real analysis or topology.
\end{proof}

\begin{lemma}
  Let $\alpha$ be a non-zero limit ordinal, then

  \begin{enumerate}
    \item If $A \subseteq \alpha$ and $\sup A = \alpha$, the order-type of $A$ is at least $\operatorname{cf} \alpha$.
    \item Let $\beta_0 \leq \beta_1 \leq \dots \leq \beta_{\xi} \leq \dots $ for $\xi < \gamma$ 
    be a non-decreasing sequence of ordinals such that $\lim_{\xi \to \gamma} = \alpha$, then $\operatorname{cf} \gamma = \alpha$.
  \end{enumerate}
\end{lemma}

\begin{proof}

  \begin{enumerate}
    \item The order-type of $A$ is the length of the increasing enumeration of $A$, 
    the limit of which (as an increasing sequence) is $\alpha$.
    \item If $\gamma = \lim_{\nu \to \operatorname{cf} \gamma} \xi_{\nu}$,
    then $\alpha = \lim_{\nu \to \operatorname{cf} \gamma} \beta_{\xi_{\nu}}$,
    and the non-decreasing sequence $\langle \beta_{\xi_{\nu}} : \nu < \operatorname{cf} \gamma\rangle$
    has an increasing sequence of the length at most $\operatorname{cf} \gamma$ and it has the same limit, 
    so $\operatorname{cf} \alpha \leq \operatorname{cf} \gamma$.

    To show $\operatorname{cf} \gamma \leq \operatorname{cf} \alpha$, 
    assume $\alpha = \lim_{\nu \to \operatorname{cf} \alpha} \alpha_{\nu}$.
    Take $\nu < \operatorname{cf} \alpha$, 
    let $\xi_{\nu}$ be the least $\xi$ greater than all $\xi_{\iota}$ for $\iota < \nu$
    such that $\beta_{\xi} > \alpha_{\nu}$.
    We have $\alpha = \lim_{\nu \to \operatorname{cf} \alpha} \beta_{\xi_{\nu}}$, so
    $\gamma = \lim_{\nu \to \operatorname{cf} \alpha} \xi_{\nu}$, so the inequation is proved.
  \end{enumerate}
\end{proof}

An infinite cardinal $\aleph_{\alpha}$ is \emph{regular} if $\operatorname{cf} \omega_{\alpha} = \omega_{\alpha}$.
$\aleph_{\alpha}$ is \emph{singular} if $\operatorname{cf} \omega_{\alpha} < \omega_{\alpha}$.

\section{The Baire Space}

\section{The Axiom of Choice}

\section{Cardinal Arithmetic via the Generalised Continuum Hypothesis}

\bibliographystyle{alpha}
\bibliography{Text}


\end{document}